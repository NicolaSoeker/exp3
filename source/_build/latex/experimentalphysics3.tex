%% Generated by Sphinx.
\def\sphinxdocclass{report}
\documentclass[letterpaper,10pt,english]{sphinxmanual}
\ifdefined\pdfpxdimen
   \let\sphinxpxdimen\pdfpxdimen\else\newdimen\sphinxpxdimen
\fi \sphinxpxdimen=.75bp\relax

\PassOptionsToPackage{warn}{textcomp}
\usepackage[utf8]{inputenc}
\ifdefined\DeclareUnicodeCharacter
% support both utf8 and utf8x syntaxes
  \ifdefined\DeclareUnicodeCharacterAsOptional
    \def\sphinxDUC#1{\DeclareUnicodeCharacter{"#1}}
  \else
    \let\sphinxDUC\DeclareUnicodeCharacter
  \fi
  \sphinxDUC{00A0}{\nobreakspace}
  \sphinxDUC{2500}{\sphinxunichar{2500}}
  \sphinxDUC{2502}{\sphinxunichar{2502}}
  \sphinxDUC{2514}{\sphinxunichar{2514}}
  \sphinxDUC{251C}{\sphinxunichar{251C}}
  \sphinxDUC{2572}{\textbackslash}
\fi
\usepackage{cmap}
\usepackage[T1]{fontenc}
\usepackage{amsmath,amssymb,amstext}
\usepackage{babel}



\usepackage{times}
\expandafter\ifx\csname T@LGR\endcsname\relax
\else
% LGR was declared as font encoding
  \substitutefont{LGR}{\rmdefault}{cmr}
  \substitutefont{LGR}{\sfdefault}{cmss}
  \substitutefont{LGR}{\ttdefault}{cmtt}
\fi
\expandafter\ifx\csname T@X2\endcsname\relax
  \expandafter\ifx\csname T@T2A\endcsname\relax
  \else
  % T2A was declared as font encoding
    \substitutefont{T2A}{\rmdefault}{cmr}
    \substitutefont{T2A}{\sfdefault}{cmss}
    \substitutefont{T2A}{\ttdefault}{cmtt}
  \fi
\else
% X2 was declared as font encoding
  \substitutefont{X2}{\rmdefault}{cmr}
  \substitutefont{X2}{\sfdefault}{cmss}
  \substitutefont{X2}{\ttdefault}{cmtt}
\fi


\usepackage[Bjarne]{fncychap}
\usepackage{sphinx}

\fvset{fontsize=\small}
\usepackage{geometry}


% Include hyperref last.
\usepackage{hyperref}
% Fix anchor placement for figures with captions.
\usepackage{hypcap}% it must be loaded after hyperref.
% Set up styles of URL: it should be placed after hyperref.
\urlstyle{same}

\addto\captionsenglish{\renewcommand{\contentsname}{Course Information:}}

\usepackage{sphinxmessages}
\setcounter{tocdepth}{1}


% Jupyter Notebook code cell colors
\definecolor{nbsphinxin}{HTML}{307FC1}
\definecolor{nbsphinxout}{HTML}{BF5B3D}
\definecolor{nbsphinx-code-bg}{HTML}{F5F5F5}
\definecolor{nbsphinx-code-border}{HTML}{E0E0E0}
\definecolor{nbsphinx-stderr}{HTML}{FFDDDD}
% ANSI colors for output streams and traceback highlighting
\definecolor{ansi-black}{HTML}{3E424D}
\definecolor{ansi-black-intense}{HTML}{282C36}
\definecolor{ansi-red}{HTML}{E75C58}
\definecolor{ansi-red-intense}{HTML}{B22B31}
\definecolor{ansi-green}{HTML}{00A250}
\definecolor{ansi-green-intense}{HTML}{007427}
\definecolor{ansi-yellow}{HTML}{DDB62B}
\definecolor{ansi-yellow-intense}{HTML}{B27D12}
\definecolor{ansi-blue}{HTML}{208FFB}
\definecolor{ansi-blue-intense}{HTML}{0065CA}
\definecolor{ansi-magenta}{HTML}{D160C4}
\definecolor{ansi-magenta-intense}{HTML}{A03196}
\definecolor{ansi-cyan}{HTML}{60C6C8}
\definecolor{ansi-cyan-intense}{HTML}{258F8F}
\definecolor{ansi-white}{HTML}{C5C1B4}
\definecolor{ansi-white-intense}{HTML}{A1A6B2}
\definecolor{ansi-default-inverse-fg}{HTML}{FFFFFF}
\definecolor{ansi-default-inverse-bg}{HTML}{000000}

% Define an environment for non-plain-text code cell outputs (e.g. images)
\makeatletter
\newenvironment{nbsphinxfancyoutput}{%
    % Avoid fatal error with framed.sty if graphics too long to fit on one page
    \let\sphinxincludegraphics\nbsphinxincludegraphics
    \nbsphinx@image@maxheight\textheight
    \advance\nbsphinx@image@maxheight -2\fboxsep   % default \fboxsep 3pt
    \advance\nbsphinx@image@maxheight -2\fboxrule  % default \fboxrule 0.4pt
    \advance\nbsphinx@image@maxheight -\baselineskip
\def\nbsphinxfcolorbox{\spx@fcolorbox{nbsphinx-code-border}{white}}%
\def\FrameCommand{\nbsphinxfcolorbox\nbsphinxfancyaddprompt\@empty}%
\def\FirstFrameCommand{\nbsphinxfcolorbox\nbsphinxfancyaddprompt\sphinxVerbatim@Continues}%
\def\MidFrameCommand{\nbsphinxfcolorbox\sphinxVerbatim@Continued\sphinxVerbatim@Continues}%
\def\LastFrameCommand{\nbsphinxfcolorbox\sphinxVerbatim@Continued\@empty}%
\MakeFramed{\advance\hsize-\width\@totalleftmargin\z@\linewidth\hsize\@setminipage}%
\lineskip=1ex\lineskiplimit=1ex\raggedright%
}{\par\unskip\@minipagefalse\endMakeFramed}
\makeatother
\newbox\nbsphinxpromptbox
\def\nbsphinxfancyaddprompt{\ifvoid\nbsphinxpromptbox\else
    \kern\fboxrule\kern\fboxsep
    \copy\nbsphinxpromptbox
    \kern-\ht\nbsphinxpromptbox\kern-\dp\nbsphinxpromptbox
    \kern-\fboxsep\kern-\fboxrule\nointerlineskip
    \fi}
\newlength\nbsphinxcodecellspacing
\setlength{\nbsphinxcodecellspacing}{0pt}

% Define support macros for attaching opening and closing lines to notebooks
\newsavebox\nbsphinxbox
\makeatletter
\newcommand{\nbsphinxstartnotebook}[1]{%
    \par
    % measure needed space
    \setbox\nbsphinxbox\vtop{{#1\par}}
    % reserve some space at bottom of page, else start new page
    \needspace{\dimexpr2.5\baselineskip+\ht\nbsphinxbox+\dp\nbsphinxbox}
    % mimick vertical spacing from \section command
      \addpenalty\@secpenalty
      \@tempskipa 3.5ex \@plus 1ex \@minus .2ex\relax
      \addvspace\@tempskipa
      {\Large\@tempskipa\baselineskip
             \advance\@tempskipa-\prevdepth
             \advance\@tempskipa-\ht\nbsphinxbox
             \ifdim\@tempskipa>\z@
               \vskip \@tempskipa
             \fi}
    \unvbox\nbsphinxbox
    % if notebook starts with a \section, prevent it from adding extra space
    \@nobreaktrue\everypar{\@nobreakfalse\everypar{}}%
    % compensate the parskip which will get inserted by next paragraph
    \nobreak\vskip-\parskip
    % do not break here
    \nobreak
}% end of \nbsphinxstartnotebook

\newcommand{\nbsphinxstopnotebook}[1]{%
    \par
    % measure needed space
    \setbox\nbsphinxbox\vbox{{#1\par}}
    \nobreak % it updates page totals
    \dimen@\pagegoal
    \advance\dimen@-\pagetotal \advance\dimen@-\pagedepth
    \advance\dimen@-\ht\nbsphinxbox \advance\dimen@-\dp\nbsphinxbox
    \ifdim\dimen@<\z@
      % little space left
      \unvbox\nbsphinxbox
      \kern-.8\baselineskip
      \nobreak\vskip\z@\@plus1fil
      \penalty100
      \vskip\z@\@plus-1fil
      \kern.8\baselineskip
    \else
      \unvbox\nbsphinxbox
    \fi
}% end of \nbsphinxstopnotebook

% Ensure height of an included graphics fits in nbsphinxfancyoutput frame
\newdimen\nbsphinx@image@maxheight % set in nbsphinxfancyoutput environment
\newcommand*{\nbsphinxincludegraphics}[2][]{%
    \gdef\spx@includegraphics@options{#1}%
    \setbox\spx@image@box\hbox{\includegraphics[#1,draft]{#2}}%
    \in@false
    \ifdim \wd\spx@image@box>\linewidth
      \g@addto@macro\spx@includegraphics@options{,width=\linewidth}%
      \in@true
    \fi
    % no rotation, no need to worry about depth
    \ifdim \ht\spx@image@box>\nbsphinx@image@maxheight
      \g@addto@macro\spx@includegraphics@options{,height=\nbsphinx@image@maxheight}%
      \in@true
    \fi
    \ifin@
      \g@addto@macro\spx@includegraphics@options{,keepaspectratio}%
    \fi
    \setbox\spx@image@box\box\voidb@x % clear memory
    \expandafter\includegraphics\expandafter[\spx@includegraphics@options]{#2}%
}% end of "\MakeFrame"-safe variant of \sphinxincludegraphics
\makeatother

\makeatletter
\renewcommand*\sphinx@verbatim@nolig@list{\do\'\do\`}
\begingroup
\catcode`'=\active
\let\nbsphinx@noligs\@noligs
\g@addto@macro\nbsphinx@noligs{\let'\PYGZsq}
\endgroup
\makeatother
\renewcommand*\sphinxbreaksbeforeactivelist{\do\<\do\"\do\'}
\renewcommand*\sphinxbreaksafteractivelist{\do\.\do\,\do\:\do\;\do\?\do\!\do\/\do\>\do\-}
\makeatletter
\fvset{codes*=\sphinxbreaksattexescapedchars\do\^\^\let\@noligs\nbsphinx@noligs}
\makeatother



\title{Experimental Physics 3}
\date{Dec 28, 2020}
\release{1}
\author{Frank Cichos}
\newcommand{\sphinxlogo}{\vbox{}}
\renewcommand{\releasename}{Release}
\makeindex
\begin{document}

\pagestyle{empty}
\sphinxmaketitle
\pagestyle{plain}
\sphinxtableofcontents
\pagestyle{normal}
\phantomsection\label{\detokenize{index::doc}}
\begin{figure}[htbp]
\centering

\noindent\sphinxincludegraphics[width=8711\sphinxpxdimen,height=1893\sphinxpxdimen]{{CompSoft_banner}.png}
\end{figure}



In this Experimental Physics 3 course, we will dive into to basic experiments and mathematical descriptions related to light propagation, electromagnetic waves and its material counter part of matter waves. In particular we will have a look at
\begin{itemize}
\item {} 
Geometrical Optics

\item {} 
Wave Optics1

\item {} 
Electromagnetic Waves

\item {} 
Matter Waves and Quantum Mechanics

\end{itemize}

The fields of optics and quantum mechanics are nowadays very active research areas with a dynamically devloping field of optical technologies, high resolution microscopy and quantum information. All this builds on the fundations that are tackled in this course. To back up the lectures I further recommend \sphinxstylestrong{books} that are listed in the \sphinxstylestrong{resources section} of the website.


\chapter{Glassbow Challenge Pictures}
\label{\detokenize{index:glassbow-challenge-pictures}}
The awesome fotos of the Glassbow Challenge by \sphinxstylestrong{Tomasz Niewiadomski, Shashank Shetty Kalavara, Thomas de Paula Barbosa, Theo Häußler}.

\noindent\sphinxincludegraphics[width=0.300\linewidth]{{GlB1}.png}

\noindent\sphinxincludegraphics[width=0.300\linewidth]{{GlB2}.jpg}

\noindent\sphinxincludegraphics[width=0.300\linewidth]{{GlB7}.jpg}


\section{This Website}
\label{\detokenize{course-info/website:this-website}}\label{\detokenize{course-info/website::doc}}
This website will contain all the information that are required for our \sphinxstylestrong{Experimental Physics 3} course. It is not yet complete, but it will be updated each week and you will find new lectures every week.
All of the lectures will be made available as Jupyter notebooks and videos.
You will be guided from here to several resources that you can use to learn programming in Python.


\section{Course Schedule}
\label{\detokenize{course-info/schedule:course-schedule}}\label{\detokenize{course-info/schedule::doc}}
The course will be held in two weekly lectures, starting \sphinxstylestrong{26.10.2020} online.

\begin{DUlineblock}{0em}
\item[] \sphinxstylestrong{Monday 11:15 am \textendash{} 12:45 pm}
\item[] \sphinxstylestrong{Thursday 11:15 am \textendash{} 12:45 pm}
\end{DUlineblock}

The course and the material is available online on this website. You may come back to study whenever it is suitable for you.

The seminar will be held online by Alex and Nic. Please also use the discussion forum to ask questions and leave comments.


\section{Assignments}
\label{\detokenize{course-info/assignments:assignments}}\label{\detokenize{course-info/assignments::doc}}
All of the assignments of the lecture will be handled in \sphinxhref{https://moodle2.uni-leipzig.de}{moodle}. An assigment will be handed out every week starting the 26.October 2020 in the afternoon. The assignment will be due the week after before the lecture until 11:00 o’clock. All assignments will be corrected and the solution to the individual problems will be discussed in the seminar one week later.

\sphinxstylestrong{To take part in the final exam you need 50\% (sharp!) of the total point possible.}

The video below tells you how to access the assignments and how to submit the solutions.




\section{Exams}
\label{\detokenize{course-info/exam:exams}}\label{\detokenize{course-info/exam::doc}}
This course will end with a \sphinxstylestrong{written exam} of \sphinxstylestrong{180 min} duration. To take part in the exam you need to qualify with 50 \% of the possible points of the exercises handed out weekly.


\section{Resources}
\label{\detokenize{course-info/resources:resources}}\label{\detokenize{course-info/resources::doc}}

\subsection{Books}
\label{\detokenize{course-info/resources:books}}
The course will be mainly build on a number of excellent books on electrodynamics and optics as well as on the basics of quantum mechanics.
\begin{enumerate}
\sphinxsetlistlabels{\arabic}{enumi}{enumii}{}{.}%
\item {} 
\sphinxstylestrong{Optics and Electrodynamics}
\begin{itemize}
\item {} 
Demtröder: Electrodynamics and Optics, Springer, 2019

\item {} 
Saleh/Teich: Fundamentals of Photonics, Wiley, 2007

\item {} 
Jackson: Classical Electrodynamics, Wiley, 1998

\item {} 
Hecht: Optics, Pearson, 2016

\item {} 
My handwritten lecture notes for download \sphinxcode{\sphinxupquote{pdf}}

\end{itemize}

\item {} 
\sphinxstylestrong{Quantum Mechanics}
\begin{itemize}
\item {} 
Demtröder: Atoms, Molecules and Photons, Springer, 2010

\item {} 
Haken, Wolf: The Physics of Atoms and Quanta: Introduction to Experiments and Theory, Springer, 2005

\item {} 
Harnwel, Livingood: Experimental Atomic Physics, McGraw\sphinxhyphen{}Hill Book Company, Inc, 1933

\end{itemize}

\end{enumerate}


\subsection{Molecular Nanophotonics Group}
\label{\detokenize{course-info/resources:molecular-nanophotonics-group}}
Besides the books, you may also want to have a look at the following websites maintained by the group
\begin{itemize}
\item {} 
\sphinxhref{https://home.uni-leipzig.de/~physik/sites/mona/}{Molecular Nanophotonics Group Website}

\item {} 
\sphinxhref{https://fcichos.github.com/website/}{Computer\sphinxhyphen{}based Physical Modeling Website @ MONA}

\end{itemize}


\section{Instructors}
\label{\detokenize{course-info/instructor:instructors}}\label{\detokenize{course-info/instructor::doc}}\begin{itemize}
\item {} 
Prof. Dr. Frank Cichos
\begin{itemize}
\item {} 
Linnéstr. 5, 04103 Leipzig

\item {} 
Office: 322

\item {} 
Phone: +0341 97 32571

\item {} 
Email: \sphinxstyleemphasis{lastname@physik.uni\sphinxhyphen{}leipzig.de}

\end{itemize}

\end{itemize}

\begin{DUlineblock}{0em}
\item[] 
\end{DUlineblock}
\begin{itemize}
\item {} 
Alexander Fischer
\begin{itemize}
\item {} 
Linnéstr. 5, 04103 Leipzig

\item {} 
Office: 333a

\item {} 
Phone: +0341 97 32570

\item {} 
Email: \sphinxstyleemphasis{firstname.lastname@physik.uni\sphinxhyphen{}leipzig.de}

\end{itemize}

\end{itemize}

\begin{DUlineblock}{0em}
\item[] 
\end{DUlineblock}
\begin{itemize}
\item {} 
Nicola Söker
\begin{itemize}
\item {} 
Linnéstr. 5, 04103 Leipzig

\item {} 
Office: 102a

\item {} 
Phone: +0341 97 32575

\item {} 
Email: \sphinxstyleemphasis{firstname.lastname@studserv.uni\sphinxhyphen{}leipzig.de}

\end{itemize}

\end{itemize}


\section{Overview}
\label{\detokenize{lectures/Intro/overview:overview}}\label{\detokenize{lectures/Intro/overview::doc}}
The Experimental Physics 3 course is introducing you to topics related to electromagnetic waves, optics and matter waves.
We will cover experiments, some basic mathematical description and also interactive visualizations in our lecture. Below
you find the planned contents of our course. Besides the physical contents, this website also contains a number of additional
interactive features. These features are provided by so\sphinxhyphen{}called Jupyter notebooks and the contained Python code. If you want to know more about those possibilities, have a look at the sections on Jupyter notebooks in the Course Introduction.



\begin{DUlineblock}{0em}
\item[] 
\end{DUlineblock}


\subsection{Lecture Contents}
\label{\detokenize{lectures/Intro/overview:lecture-contents}}\begin{enumerate}
\sphinxsetlistlabels{\arabic}{enumi}{enumii}{}{.}%
\item {} \begin{description}
\item[{Ray Optics}] \leavevmode
\begin{DUlineblock}{0em}
\item[] 1.1.Reflection
\item[] 1.2.Refraction, Total internal reflection, Rainbow challenge
\item[] 1.3.Mirrors, Lenses, Prisms
\item[] 1.4.Optical instruments
\item[]
\begin{DUlineblock}{\DUlineblockindent}
\item[] 1.4.1.Telescope
\item[] 1.4.2.Microscope
\end{DUlineblock}
\item[] 1.5.Dispersion
\item[] 1.6.Imaging errors
\item[]
\begin{DUlineblock}{\DUlineblockindent}
\item[] 1.6.1.Spherical aberration
\item[] 1.6.2.Coma
\item[] 1.6.3.Astigmatism
\item[] 1.6.4.Chromatic aberration
\end{DUlineblock}
\end{DUlineblock}

\end{description}

\item {} \begin{description}
\item[{Wave Optics}] \leavevmode
\begin{DUlineblock}{0em}
\item[] 2.1.Wave equation
\item[]
\begin{DUlineblock}{\DUlineblockindent}
\item[] 2.1.1.Plane waves
\item[] 2.1.2.Spherical waves
\end{DUlineblock}
\item[] 2.2.Interference
\item[]
\begin{DUlineblock}{\DUlineblockindent}
\item[] 2.2.1.Coherence
\item[] 2.2.2.Interferometers
\end{DUlineblock}
\item[] 2.3.Huygens principle
\item[]
\begin{DUlineblock}{\DUlineblockindent}
\item[] 2.3.1.Diffraction
\item[] 2.3.2.Single and double slit
\item[] 2.3.3.Diffraction grating
\item[] 2.3.4.Optical resolution
\end{DUlineblock}
\end{DUlineblock}

\end{description}

\item {} \begin{description}
\item[{Electromagnetic Waves}] \leavevmode
\begin{DUlineblock}{0em}
\item[] 3.1.Electromagnetic spectrum
\item[] 3.2.Plane and spherical electromagnetic waves
\item[] 3.3.Energy transport and Poynting vector
\item[] 3.4.Polarization
\item[] 3.5.Reflection and transmission
\item[] 3.6.Total internal reflection
\item[] 3.7.Fresnel formulas
\item[] 3.8.Hertz dipole
\end{DUlineblock}

\end{description}

\item {} \begin{description}
\item[{Foundations of Quantum Physics}] \leavevmode
\begin{DUlineblock}{0em}
\item[] 4.1.Particle properties of light
\item[]
\begin{DUlineblock}{\DUlineblockindent}
\item[] 4.1.1.Photo effect
\item[] 4.1.2.Black body radiation
\item[] 4.1.3.Photon gas
\item[] 4.1.4.Planck’s radiation law
\end{DUlineblock}
\item[] 4.2.Structure of matter
\item[]
\begin{DUlineblock}{\DUlineblockindent}
\item[] 4.2.1.Thomson model of the atom
\item[] 4.2.2.Rutherford scattering
\item[] 4.2.3.Rutherford and Bohr atom model
\end{DUlineblock}
\item[] 4.3.Matter waves
\item[]
\begin{DUlineblock}{\DUlineblockindent}
\item[] 4.3.1.Heisenberg uncertainty relation
\item[] 4.3.2.Wave function
\item[] 4.3.3.Probability interpretation of the wave function
\item[] 4.3.4.Schrödinger equation
\item[] 4.3.5.Quantum states
\item[] 4.3.6.Potential box
\item[] 4.3.7.Harmonic oscillator
\item[] 4.3.8.Tunneling
\item[] 4.3.9.Correspondence principle
\end{DUlineblock}
\end{DUlineblock}

\end{description}

\end{enumerate}

The following section was created from \sphinxcode{\sphinxupquote{source/notebooks/Intro/Introduction2Jupyter.ipynb}}.


\section{Introduction to Jupyter}
\label{\detokenize{notebooks/Intro/Introduction2Jupyter:Introduction-to-Jupyter}}\label{\detokenize{notebooks/Intro/Introduction2Jupyter::doc}}

\subsection{What is Jupyter Notebook?}
\label{\detokenize{notebooks/Intro/Introduction2Jupyter:What-is-Jupyter-Notebook?}}
A Jupyter Notebook is a web browser based \sphinxstylestrong{interactive computing environment} that enables users to create documents that include code to be executed, results from the executed code such as plots and images, and finally also an additional documentation in form of markdown text including equations in LaTeX.

These documents provide a \sphinxstylestrong{complete and self\sphinxhyphen{}contained record of a computation} that can be converted to various formats and shared with others using email, version control systems (like git/\sphinxhref{https://github.com}{GitHub}) or \sphinxhref{http://nbviewer.jupyter.org}{nbviewer.jupyter.org}.

The Jupyter Notebook combines three components:
\begin{itemize}
\item {} 
\sphinxstylestrong{Notebook editor}: An interactive application for writing and running code interactively and editing notebook documents. If you run Jupyter on desktop, you will be using Jupyter’s web application.

\item {} 
\sphinxstylestrong{Kernels}: Separate processes started by Jupyter on your server, that runs users’ code in a given language and returns output back to the notebook web application. The kernel also handles things like computations for interactive widgets, tab completion and introspection.

\item {} 
\sphinxstylestrong{Notebook documents}: Self\sphinxhyphen{}contained documents that contain a representation of all content visible in the notebook editor, including inputs and outputs of the computations, markdown text, equations, images, and rich media representations of objects. Each notebook document has its own kernel.

\end{itemize}


\subsection{Notebook editor}
\label{\detokenize{notebooks/Intro/Introduction2Jupyter:Notebook-editor}}
The Notebook editor is a web application running in your browser. It enables you to
\begin{itemize}
\item {} 
\sphinxstylestrong{Edit code} in individual cells

\item {} 
\sphinxstylestrong{Run code} in individuall cells in arbitrary order and display results of the computation in various formats (HTML, LaTeX, PNG, SVG, PDF)

\item {} 
Create and use \sphinxstylestrong{interactive JavaScript widgets}, which bind interactive user interface controls and visualizations to reactive kernel side computations.

\item {} 
Add \sphinxstylestrong{documentation text} using \sphinxhref{https://daringfireball.net/projects/markdown/}{Markdown} markup language, including LaTeX equations

\end{itemize}


\subsection{Kernels}
\label{\detokenize{notebooks/Intro/Introduction2Jupyter:Kernels}}
The Jupyter notebook is not bound to any specific programming language, but can be used for almost any type of language. Each Jupyter notebook starts a server application that is connected to a kernel that runs the code in the notebook. This kernel is dedicated to a specific programming language. Thus, installing different kernels \sphinxhref{https://github.com/jupyter/jupyter/wiki/Jupyter-kernels}{100+ languages} allows you to execute code in \sphinxstylestrong{Python}, \sphinxstylestrong{Julia}, \sphinxstylestrong{R}, \sphinxstylestrong{Ruby}, \sphinxstylestrong{Haskell},
\sphinxstylestrong{Scala}, and many others.

Yet, the default kernel runs Python code. The notebook provides a simple way for users to pick which of these kernels is used for a given notebook. Each of these kernels communicate with the notebook editor using JSON over the ZeroMQ/WebSockets message protocol that is described \sphinxhref{https://jupyter-client.readthedocs.io/en/latest/messaging.html\#messaging}{here}. Most users do not need to know about these details, but it helps to understand that “kernels run code”.


\subsection{Notebook documents}
\label{\detokenize{notebooks/Intro/Introduction2Jupyter:Notebook-documents}}
Notebook documents, or notebooks, contain the \sphinxstylestrong{inputs and outputs} of an interactive session as well as \sphinxstylestrong{documentation text} that accompanies the code but is not meant for execution.

A notebook is just a \sphinxstylestrong{file on your server’s filesystem with a \textasciigrave{}\textasciigrave{}.ipynb\textasciigrave{}\textasciigrave{} extension}. This allows you to share your notebook easily.

Notebooks consist of a \sphinxstylestrong{linear sequence of cells}. There are three basic cell types:
\begin{itemize}
\item {} 
\sphinxstylestrong{Code cells:} Input and output of live code that is run in the kernel.

\item {} 
\sphinxstylestrong{Markdown cells:} Narrative text with embedded LaTeX equations.

\item {} 
\sphinxstylestrong{Raw cells:} Unformatted text that is included, without modification, when notebooks are converted to different formats using \sphinxcode{\sphinxupquote{nbconvert}}.

\end{itemize}

Internally, notebook documents are \sphinxhref{https://en.wikipedia.org/wiki/JSON}{JSON}\sphinxstylestrong{text files} with \sphinxstylestrong{binary data encoded in}\sphinxhref{http://en.wikipedia.org/wiki/Base64}{base64}. This allows them to be \sphinxstylestrong{read and manipulated programmatically} by any programming language.

\sphinxstylestrong{Notebooks can be exported} to different static formats including HTML, reStructeredText, LaTeX, PDF, and slide shows (\sphinxhref{http://lab.hakim.se/reveal-js/}{reveal.js}) using Jupyter’s \sphinxcode{\sphinxupquote{nbconvert}} utility.

Furthermore, any notebook document available from a \sphinxstylestrong{public URL on or GitHub can be shared} via \sphinxhref{http://nbviewer.jupyter.org}{nbviewer}. This service loads the notebook document from the URL and renders it as a static web page. The resulting web page may thus be shared with others \sphinxstylestrong{without their needing to install the Jupyter Notebook}.

The following section was created from \sphinxcode{\sphinxupquote{source/notebooks/Intro/NotebookEditor.ipynb}}.


\section{Notebook editor}
\label{\detokenize{notebooks/Intro/NotebookEditor:Notebook-editor}}\label{\detokenize{notebooks/Intro/NotebookEditor::doc}}
\sphinxincludegraphics[width=778\sphinxpxdimen,height=478\sphinxpxdimen]{{notebook}.png}

A Jupyter Notebook provides an interface with essentially two modes
\begin{itemize}
\item {} 
\sphinxstylestrong{edit mode} the mode where you edit a cell’s content.

\item {} 
\sphinxstylestrong{command mode} the mode where you execute the cells content.

\end{itemize}

In the more advanced version of JupyterLab, which we are using on myBinder, this will look like that


\subsection{Edit mode}
\label{\detokenize{notebooks/Intro/NotebookEditor:Edit-mode}}
Edit mode is indicated by a blue cell border and a prompt showing in the editor area:

\sphinxincludegraphics[width=526\sphinxpxdimen,height=45\sphinxpxdimen]{{edit_mode}.png}

When a cell is in edit mode, you can type into the cell, like a normal text editor.


\subsection{Command mode}
\label{\detokenize{notebooks/Intro/NotebookEditor:Command-mode}}
Command mode is indicated by a grey cell border:

\sphinxincludegraphics[width=521\sphinxpxdimen,height=46\sphinxpxdimen]{{command_mode}.png}


\subsection{Keyboard navigation}
\label{\detokenize{notebooks/Intro/NotebookEditor:Keyboard-navigation}}
If you have a hardware keyboard connected to your iOS device, you can use Jupyter keyboard shortcuts. The modal user interface of the Jupyter Notebook has been optimized for efficient keyboard usage. This is made possible by having two different sets of keyboard shortcuts: one set that is active in edit mode and another in command mode.

In edit mode, most of the keyboard is dedicated to typing into the cell’s editor. Thus, in edit mode there are relatively few shortcuts. In command mode, the entire keyboard is available for shortcuts, so there are many more. Most important ones are:
\begin{enumerate}
\sphinxsetlistlabels{\arabic}{enumi}{enumii}{}{.}%
\item {} 
Switch command and edit mods: \sphinxcode{\sphinxupquote{Enter}} for edit mode, and \sphinxcode{\sphinxupquote{Esc}} or \sphinxcode{\sphinxupquote{Control}} for command mode.

\item {} 
Basic navigation: \sphinxcode{\sphinxupquote{↑}}/\sphinxcode{\sphinxupquote{k}}, \sphinxcode{\sphinxupquote{↓}}/\sphinxcode{\sphinxupquote{j}}

\item {} 
Run or render currently selected cell: \sphinxcode{\sphinxupquote{Shift}}+\sphinxcode{\sphinxupquote{Enter}} or \sphinxcode{\sphinxupquote{Control}}+\sphinxcode{\sphinxupquote{Enter}}

\item {} 
Saving the notebook: \sphinxcode{\sphinxupquote{s}}

\item {} 
Change cell types: \sphinxcode{\sphinxupquote{y}} to make it a \sphinxstylestrong{code} cell, \sphinxcode{\sphinxupquote{m}} for \sphinxstylestrong{markdown} and \sphinxcode{\sphinxupquote{r}} for \sphinxstylestrong{raw}

\item {} 
Inserting new cells: \sphinxcode{\sphinxupquote{a}} to \sphinxstylestrong{insert above}, \sphinxcode{\sphinxupquote{b}} to \sphinxstylestrong{insert below}

\item {} 
Manipulating cells using pasteboard: \sphinxcode{\sphinxupquote{x}} for \sphinxstylestrong{cut}, \sphinxcode{\sphinxupquote{c}} for \sphinxstylestrong{copy}, \sphinxcode{\sphinxupquote{v}} for \sphinxstylestrong{paste}, \sphinxcode{\sphinxupquote{d}} for \sphinxstylestrong{delete} and \sphinxcode{\sphinxupquote{z}} for \sphinxstylestrong{undo delete}

\item {} 
Kernel operations: \sphinxcode{\sphinxupquote{i}} to \sphinxstylestrong{interrupt} and \sphinxcode{\sphinxupquote{0}} to \sphinxstylestrong{restart}

\end{enumerate}


\subsection{Running code}
\label{\detokenize{notebooks/Intro/NotebookEditor:Running-code}}
Code cells allow you to enter and run code. Run a code cell by pressing the \sphinxcode{\sphinxupquote{▶︎}} button in the bottom\sphinxhyphen{}right panel, or \sphinxcode{\sphinxupquote{Control}}+\sphinxcode{\sphinxupquote{Enter}} on your hardware keyboard.

{
\sphinxsetup{VerbatimColor={named}{nbsphinx-code-bg}}
\sphinxsetup{VerbatimBorderColor={named}{nbsphinx-code-border}}
\begin{sphinxVerbatim}[commandchars=\\\{\}]
\llap{\color{nbsphinxin}[1]:\,\hspace{\fboxrule}\hspace{\fboxsep}}\PYG{n}{v} \PYG{o}{=} \PYG{l+m+mi}{10}
\end{sphinxVerbatim}
}

{
\sphinxsetup{VerbatimColor={named}{nbsphinx-code-bg}}
\sphinxsetup{VerbatimBorderColor={named}{nbsphinx-code-border}}
\begin{sphinxVerbatim}[commandchars=\\\{\}]
\llap{\color{nbsphinxin}[5]:\,\hspace{\fboxrule}\hspace{\fboxsep}}\PYG{n+nb}{print}\PYG{p}{(}\PYG{n}{v}\PYG{p}{)}
\end{sphinxVerbatim}
}

{

\kern-\sphinxverbatimsmallskipamount\kern-\baselineskip
\kern+\FrameHeightAdjust\kern-\fboxrule
\vspace{\nbsphinxcodecellspacing}

\sphinxsetup{VerbatimColor={named}{white}}
\sphinxsetup{VerbatimBorderColor={named}{nbsphinx-code-border}}
\begin{sphinxVerbatim}[commandchars=\\\{\}]
100
\end{sphinxVerbatim}
}

There are a couple of keyboard shortcuts for running code:
\begin{itemize}
\item {} 
\sphinxcode{\sphinxupquote{Control}}+\sphinxcode{\sphinxupquote{Enter}} runs the current cell and enters command mode.

\item {} 
\sphinxcode{\sphinxupquote{Shift}}+\sphinxcode{\sphinxupquote{Enter}} runs the current cell and moves selection to the one below.

\item {} 
\sphinxcode{\sphinxupquote{Option}}+\sphinxcode{\sphinxupquote{Enter}} runs the current cell and inserts a new one below.

\end{itemize}


\subsection{Managing the kernel}
\label{\detokenize{notebooks/Intro/NotebookEditor:Managing-the-kernel}}
Code is run in a separate process called the \sphinxstylestrong{kernel}, which can be interrupted or restarted. You can see the kernel indicator in the top\sphinxhyphen{}right corner reporting the current kernel state: \sphinxcode{\sphinxupquote{⚪︎}} means the kernel is \sphinxstylestrong{ready} to execute code, and \sphinxcode{\sphinxupquote{⚫︎}} means the kernel is currently \sphinxstylestrong{busy}. Tapping kernel indicator will open the \sphinxstylestrong{kernel menu}, where you can reconnect, interrupt or restart the kernel.

Try running the following cell — kernel indicator will switch from \sphinxcode{\sphinxupquote{⚪︎}} to \sphinxcode{\sphinxupquote{⚫︎}}, i.e., reporting the kernel as “busy”. This means that you will not be able to run any new cells until the current execution finishes, or until the kernel is interrupted. You can then go to the kernel menu by tapping the kernel indicator and select “Interrupt”.

The following section was created from \sphinxcode{\sphinxupquote{source/notebooks/Intro/EditCells.ipynb}}.


\section{Entering code}
\label{\detokenize{notebooks/Intro/EditCells:Entering-code}}\label{\detokenize{notebooks/Intro/EditCells::doc}}
Entering code is pretty easy. You just have to click into a cell and type the commands you want to type. If you have multiple lines of code, just press \sphinxstylestrong{enter} at the end of the line and start a new one.
\begin{itemize}
\item {} 
\sphinxstylestrong{code blocks} Python identifies blocks of codes belonging together by its identation. This will become important if you write longer code in a cell later. To indent the block, you may use either \sphinxstyleemphasis{whitespaces} or \sphinxstyleemphasis{tabs}.

\item {} 
\sphinxstylestrong{comments} Comments can be added to annotate the code, such that you or someone else can understand the code.
\begin{itemize}
\item {} 
Comments in a single line are started with the \sphinxcode{\sphinxupquote{\#}} character in front of the comment.

\item {} 
Comments over multiple lines can be started with \sphinxcode{\sphinxupquote{\textquotesingle{}\textquotesingle{}\textquotesingle{}}}and end with the same \sphinxcode{\sphinxupquote{\textquotesingle{}\textquotesingle{}\textquotesingle{}}}. They are used as \sphinxcode{\sphinxupquote{docstrings}} to provide a help text.

\end{itemize}

\end{itemize}

{
\sphinxsetup{VerbatimColor={named}{nbsphinx-code-bg}}
\sphinxsetup{VerbatimBorderColor={named}{nbsphinx-code-border}}
\begin{sphinxVerbatim}[commandchars=\\\{\}]
\llap{\color{nbsphinxin}[2]:\,\hspace{\fboxrule}\hspace{\fboxsep}}\PYG{c+c1}{\PYGZsh{} typical function}

\PYG{k}{def} \PYG{n+nf}{function}\PYG{p}{(}\PYG{n}{x}\PYG{p}{)}\PYG{p}{:}
    \PYG{l+s+sd}{\PYGZsq{}\PYGZsq{}\PYGZsq{} function to calculate a function}
\PYG{l+s+sd}{    arguments:}
\PYG{l+s+sd}{        x ... float or integer value}
\PYG{l+s+sd}{    returns:}
\PYG{l+s+sd}{        y ... two times the integer value}
\PYG{l+s+sd}{    \PYGZsq{}\PYGZsq{}\PYGZsq{}}
    \PYG{n}{y}\PYG{o}{=}\PYG{l+m+mi}{2}\PYG{o}{*}\PYG{n}{x} \PYG{c+c1}{\PYGZsh{} don\PYGZsq{}t forget the identation of the block}
    \PYG{k}{return}\PYG{p}{(}\PYG{n}{y}\PYG{p}{)}
\end{sphinxVerbatim}
}

{
\sphinxsetup{VerbatimColor={named}{nbsphinx-code-bg}}
\sphinxsetup{VerbatimBorderColor={named}{nbsphinx-code-border}}
\begin{sphinxVerbatim}[commandchars=\\\{\}]
\llap{\color{nbsphinxin}[3]:\,\hspace{\fboxrule}\hspace{\fboxsep}}\PYG{n}{help}\PYG{p}{(}\PYG{n}{function}\PYG{p}{)}
\end{sphinxVerbatim}
}

{

\kern-\sphinxverbatimsmallskipamount\kern-\baselineskip
\kern+\FrameHeightAdjust\kern-\fboxrule
\vspace{\nbsphinxcodecellspacing}

\sphinxsetup{VerbatimColor={named}{white}}
\sphinxsetup{VerbatimBorderColor={named}{nbsphinx-code-border}}
\begin{sphinxVerbatim}[commandchars=\\\{\}]
Help on function function in module \_\_main\_\_:

function(x)
    function to calculate a function
    arguments:
        x {\ldots} float or integer value
    returns:
        y {\ldots} two times the integer value

\end{sphinxVerbatim}
}


\section{Entering Markdown}
\label{\detokenize{notebooks/Intro/EditCells:Entering-Markdown}}
Text can be added to Jupyter Notebooks using Markdown cells. This is extremely useful providing a complete documentation of your calculations or simulations. In this way, everything really becomes a notebook. You can change the cell type to Markdown by using the “Cell Actions” menu, or with a hardware keyboard shortcut \sphinxcode{\sphinxupquote{m}}. Markdown is a popular markup language that is a superset of HTML. Its specification can be found here:

\sphinxurl{https://daringfireball.net/projects/markdown/}

Markdown cells can either be \sphinxstylestrong{rendered} or \sphinxstylestrong{unrendered}.

When they are rendered, you will see a nice formatted representation of the cell’s contents.

When they are unrendered, you will see the raw text source of the cell. To render the selected cell, click the \sphinxcode{\sphinxupquote{▶︎}} button or \sphinxcode{\sphinxupquote{shift}}+ \sphinxcode{\sphinxupquote{enter}}. To unrender, select the markdown cell, and press \sphinxcode{\sphinxupquote{enter}} or just double click.


\subsection{Markdown basics}
\label{\detokenize{notebooks/Intro/EditCells:Markdown-basics}}
Below are some basic markdown examples, in its rendered form. If you which to access how to create specific appearances, double click the individual cells to put them into an unrendered edit mode.

You can make text \sphinxstyleemphasis{italic} or \sphinxstylestrong{bold}.

You can build nested itemized or enumerated lists:
\begin{itemize}
\item {} 
first item
\begin{itemize}
\item {} 
first subitem
\begin{itemize}
\item {} 
first subsubitem

\end{itemize}

\item {} 
second subitem \sphinxhyphen{} first subitem of second subitem \sphinxhyphen{} second subitem of second subitem

\end{itemize}

\item {} 
second item
\begin{itemize}
\item {} 
first subitem

\end{itemize}

\item {} 
third item
\begin{itemize}
\item {} 
first subitem

\end{itemize}

\end{itemize}

Now another list:
\begin{enumerate}
\sphinxsetlistlabels{\arabic}{enumi}{enumii}{}{.}%
\item {} 
Here we go
\begin{enumerate}
\sphinxsetlistlabels{\arabic}{enumii}{enumiii}{}{.}%
\item {} 
Sublist

\item {} 
Sublist

\end{enumerate}

\item {} 
There we go

\item {} 
Now this

\end{enumerate}

Here is a blockquote:
\begin{quote}

Beautiful is better than ugly. Explicit is better than implicit. Simple is better than complex. Complex is better than complicated. Flat is better than nested. Sparse is better than dense. Readability counts. Special cases aren’t special enough to break the rules. Although practicality beats purity. Errors should never pass silently. Unless explicitly silenced. In the face of ambiguity, refuse the temptation to guess. There should be one\textendash{} and preferably only one \textendash{}obvious way to do it.
Although that way may not be obvious at first unless you’re Dutch. Now is better than never. Although never is often better than \sphinxstyleemphasis{right} now. If the implementation is hard to explain, it’s a bad idea. If the implementation is easy to explain, it may be a good idea. Namespaces are one honking great idea \textendash{} let’s do more of those!
\end{quote}

And shorthand for links:

\sphinxhref{http://jupyter.org}{Jupyter’s website}


\subsection{Headings}
\label{\detokenize{notebooks/Intro/EditCells:Headings}}
You can add headings by starting a line with one (or multiple) \sphinxcode{\sphinxupquote{\#}} followed by a space, as in the following example:

\begin{sphinxVerbatim}[commandchars=\\\{\}]
\PYG{c+c1}{\PYGZsh{} Heading 1}
\PYG{c+c1}{\PYGZsh{} Heading 2}
\PYG{c+c1}{\PYGZsh{}\PYGZsh{} Heading 2.1}
\PYG{c+c1}{\PYGZsh{}\PYGZsh{} Heading 2.2}
\end{sphinxVerbatim}


\subsection{Embedded code}
\label{\detokenize{notebooks/Intro/EditCells:Embedded-code}}
You can embed code meant for illustration instead of execution in Python:

\begin{sphinxVerbatim}[commandchars=\\\{\}]
\PYG{k}{def} \PYG{n+nf}{f}\PYG{p}{(}\PYG{n}{x}\PYG{p}{)}\PYG{p}{:}
    \PYG{l+s+sd}{\PYGZdq{}\PYGZdq{}\PYGZdq{}a docstring\PYGZdq{}\PYGZdq{}\PYGZdq{}}
    \PYG{k}{return} \PYG{n}{x}\PYG{o}{*}\PYG{o}{*}\PYG{l+m+mi}{2}
\end{sphinxVerbatim}

or other languages:

\begin{sphinxVerbatim}[commandchars=\\\{\}]
\PYG{k}{if} \PYG{p}{(}\PYG{n}{i}\PYG{o}{=}\PYG{l+m+mi}{0}\PYG{p}{;} \PYG{n}{i}\PYG{o}{\PYGZlt{}}\PYG{n}{n}\PYG{p}{;} \PYG{n}{i}\PYG{o}{+}\PYG{o}{+}\PYG{p}{)} \PYG{p}{\PYGZob{}}
  \PYG{n}{printf}\PYG{p}{(}\PYG{l+s+s2}{\PYGZdq{}}\PYG{l+s+s2}{hello }\PYG{l+s+si}{\PYGZpc{}d}\PYG{l+s+se}{\PYGZbs{}n}\PYG{l+s+s2}{\PYGZdq{}}\PYG{p}{,} \PYG{n}{i}\PYG{p}{)}\PYG{p}{;}
  \PYG{n}{x} \PYG{o}{+}\PYG{o}{=} \PYG{l+m+mi}{4}\PYG{p}{;}
\PYG{p}{\PYGZcb{}}
\end{sphinxVerbatim}


\subsection{LaTeX equations}
\label{\detokenize{notebooks/Intro/EditCells:LaTeX-equations}}
Courtesy of MathJax, you can include mathematical expressions both inline: \(e^{i\pi} + 1 = 0\) and displayed:
\begin{equation*}
\begin{split}e^x=\sum_{i=0}^\infty \frac{1}{i!}x^i\end{split}
\end{equation*}
Inline expressions can be added by surrounding the latex code with \sphinxcode{\sphinxupquote{\$}}:

\begin{sphinxVerbatim}[commandchars=\\\{\}]
\PYGZdl{}e\PYGZca{}\PYGZob{}i\PYGZbs{}pi\PYGZcb{} + 1 = 0\PYGZdl{}
\end{sphinxVerbatim}

Expressions on their own line are surrounded by \sphinxcode{\sphinxupquote{\$\$}}:

\begin{sphinxVerbatim}[commandchars=\\\{\}]
\PYG{l+s+sb}{\PYGZdl{}\PYGZdl{}}\PYG{n+nb}{e}\PYG{n+nb}{\PYGZca{}}\PYG{n+nb}{x}\PYG{o}{=}\PYG{n+nv}{\PYGZbs{}sum}\PYG{n+nb}{\PYGZus{}}\PYG{n+nb}{\PYGZob{}}\PYG{n+nb}{i}\PYG{o}{=}\PYG{l+m}{0}\PYG{n+nb}{\PYGZcb{}}\PYG{n+nb}{\PYGZca{}}\PYG{n+nv}{\PYGZbs{}infty}\PYG{n+nb}{ }\PYG{n+nv}{\PYGZbs{}frac}\PYG{n+nb}{\PYGZob{}}\PYG{l+m}{1}\PYG{n+nb}{\PYGZcb{}}\PYG{n+nb}{\PYGZob{}}\PYG{n+nb}{i}\PYG{o}{!}\PYG{n+nb}{\PYGZcb{}}\PYG{n+nb}{x}\PYG{n+nb}{\PYGZca{}}\PYG{n+nb}{i}\PYG{l+s}{\PYGZdl{}\PYGZdl{}}
\end{sphinxVerbatim}


\subsection{Images}
\label{\detokenize{notebooks/Intro/EditCells:Images}}
Images may be also directly integrated into a Markdown block.

To include images use

\begin{sphinxVerbatim}[commandchars=\\\{\}]
![alternative text](url)
\end{sphinxVerbatim}

for example

\sphinxincludegraphics[width=200\sphinxpxdimen,height=200\sphinxpxdimen]{{particle}.png}


\subsection{Videos}
\label{\detokenize{notebooks/Intro/EditCells:Videos}}
To include videos, we use HTML code like

\begin{sphinxVerbatim}[commandchars=\\\{\}]
\PYG{o}{\PYGZlt{}}\PYG{n}{video} \PYG{n}{src}\PYG{o}{=}\PYG{l+s+s2}{\PYGZdq{}}\PYG{l+s+s2}{mov/movie.mp4}\PYG{l+s+s2}{\PYGZdq{}} \PYG{n}{width}\PYG{o}{=}\PYG{l+s+s2}{\PYGZdq{}}\PYG{l+s+s2}{320}\PYG{l+s+s2}{\PYGZdq{}} \PYG{n}{height}\PYG{o}{=}\PYG{l+s+s2}{\PYGZdq{}}\PYG{l+s+s2}{200}\PYG{l+s+s2}{\PYGZdq{}} \PYG{n}{controls} \PYG{n}{preload}\PYG{o}{\PYGZgt{}}\PYG{o}{\PYGZlt{}}\PYG{o}{/}\PYG{n}{video}\PYG{o}{\PYGZgt{}}
\end{sphinxVerbatim}

in the Markdown cell. This works with videos stored locally.





You can embed YouTube Videos as well by using the \sphinxcode{\sphinxupquote{IPython}} module.

{
\sphinxsetup{VerbatimColor={named}{nbsphinx-code-bg}}
\sphinxsetup{VerbatimBorderColor={named}{nbsphinx-code-border}}
\begin{sphinxVerbatim}[commandchars=\\\{\}]
\llap{\color{nbsphinxin}[9]:\,\hspace{\fboxrule}\hspace{\fboxsep}}\PYG{k+kn}{from} \PYG{n+nn}{IPython}\PYG{n+nn}{.}\PYG{n+nn}{display} \PYG{k+kn}{import} \PYG{n}{YouTubeVideo}
\PYG{n}{YouTubeVideo}\PYG{p}{(}\PYG{l+s+s1}{\PYGZsq{}}\PYG{l+s+s1}{QlLx32juGzI}\PYG{l+s+s1}{\PYGZsq{}}\PYG{p}{,}\PYG{n}{width}\PYG{o}{=}\PYG{l+m+mi}{600}\PYG{p}{)}
\end{sphinxVerbatim}
}

\hrule height -\fboxrule\relax
\vspace{\nbsphinxcodecellspacing}

\savebox\nbsphinxpromptbox[0pt][r]{\color{nbsphinxout}\Verb|\strut{[9]:}\,|}

\begin{nbsphinxfancyoutput}

\noindent\sphinxincludegraphics[width=480\sphinxpxdimen,height=360\sphinxpxdimen]{{notebooks_Intro_EditCells_26_0}.jpg}

\end{nbsphinxfancyoutput}


\section{Lecture Contents}
\label{\detokenize{lectures/L1/overview_1:lecture-contents}}\label{\detokenize{lectures/L1/overview_1::doc}}
This is our first lecture, in which we will have a look at the organization of the course and start off with the basic ingredients of \sphinxstylestrong{Geometrical Optics}.
While during Experimental Physics 2, you did already address electrical oscillators and the wave equation, we would like to start with Geometrical Optics here, as this is the simplest form of describing light propagation and it turns out to be a limit of electromagnetic optics. Many of the phenomena in optics can already be sufficiently understood in this limit.

\noindent\sphinxincludegraphics[width=600\sphinxpxdimen]{{slides}.png}

Lecture 1 slides for download \sphinxcode{\sphinxupquote{pdf}}

The following section was created from \sphinxcode{\sphinxupquote{source/notebooks/L1/Geometrical Optics.ipynb}}.


\section{Geometrical Optics}
\label{\detokenize{notebooks/L1/Geometrical Optics:Geometrical-Optics}}\label{\detokenize{notebooks/L1/Geometrical Optics::doc}}
Geometrical optics refers to a field of optics that describes light propagation in terms of rays. This is on one side borrowed from your daily experience, for example, with shadows. On a more scientific side, it will turn out that geometrical optics is the approximation in which the wavelength of light is much shorter.

\begin{sphinxadmonition}{note}{}\unskip
\sphinxstylestrong{Geometrical Optics}

Geometrical optics is an approximate description of light propagation in the limit of infinitely small wavelength, where all wave phenomena like diffraction can be neglected.
\end{sphinxadmonition}


\subsection{Assumptions}
\label{\detokenize{notebooks/L1/Geometrical Optics:Assumptions}}
As geometrical optics is not a rigorous description, we have to write down some postulates (things we do not understand yet) to describe light propagation in geometrical optics
\begin{itemize}
\item {} 
light rays emerge from a light source

\item {} 
light rays are detected by a detector

\item {} 
light propagates in straight line paths (rays) in a homogeneous medium

\item {} 
light\textendash{}matter interaction is characterized by a refractive index \(n\)

\item {} 
light bends to a curved path in inhomogeneous media (i.e., \(n(\vec{r})\))

\item {} 
rays may be reflected and refracted at interfaces between media

\end{itemize}

Let us have a look at some examples providing indications for the linear propagation of light.

\sphinxstylestrong{Laser}

\sphinxstylestrong{Pinhole Camera}


\begin{savenotes}\sphinxattablestart
\centering
\begin{tabulary}{\linewidth}[t]{|T|}
\hline
\sphinxstyletheadfamily 
\sphinxincludegraphics[width=0.800\linewidth]{{pinhole_camera}.png}
\\
\hline
\sphinxstylestrong{Fig.:} Schematics of a pinhole camera.
\\
\hline
\end{tabulary}
\par
\sphinxattableend\end{savenotes}


\begin{savenotes}\sphinxattablestart
\centering
\begin{tabulary}{\linewidth}[t]{|T|}
\hline
\sphinxstyletheadfamily 
\sphinxincludegraphics[width=0.400\linewidth]{{pinhole_cameraF}.png}
\\
\hline
\sphinxstylestrong{Fig.:} Image of an upright letter “F” on the screen of a pinhole camera in lecture 1.
\\
\hline
\end{tabulary}
\par
\sphinxattableend\end{savenotes}


\begin{savenotes}\sphinxattablestart
\centering
\begin{tabulary}{\linewidth}[t]{|T|}
\hline
\sphinxstyletheadfamily 
\sphinxincludegraphics[width=0.400\linewidth]{{core_shadow}.png}
\\
\hline
\sphinxstylestrong{Fig.:} Core shadow and partial shadows behind a disk when illuminated with two seperate light sources.
\\
\hline
\end{tabulary}
\par
\sphinxattableend\end{savenotes}

\sphinxstylestrong{Water Basin with Salt}


\begin{savenotes}\sphinxattablestart
\centering
\begin{tabulary}{\linewidth}[t]{|T|}
\hline
\sphinxstyletheadfamily 
\sphinxincludegraphics[width=0.400\linewidth]{{refractive_index_gradient}.png}
\\
\hline
\sphinxstylestrong{Fig.:} Bend rays in a refractive index gradient with a salt (NaCl) layer on the bottom.
\\
\hline
\end{tabulary}
\par
\sphinxattableend\end{savenotes}

The following section was created from \sphinxcode{\sphinxupquote{source/notebooks/L1/Reflection.ipynb}}.


\section{Reflection}
\label{\detokenize{notebooks/L1/Reflection:Reflection}}\label{\detokenize{notebooks/L1/Reflection::doc}}
The law of reflection is probably the most simple one. Yet the simplicity gives us the chance to define some basic objects which we will further use for the description of light rays and their propagation.


\subsection{Law of Reflection}
\label{\detokenize{notebooks/L1/Reflection:Law-of-Reflection}}
The sketch below shows the reflection of an incoming light ray (red) on an interface. This incoming light ray has an angle \(\theta_{1}\) with the axis (dashed line), which is perpendicular to the reflecting surface. As compared to X\sphinxhyphen{}ray diffraction, we measure the angle to the normal of the surface and not to the surface itself.


\begin{savenotes}\sphinxattablestart
\centering
\begin{tabulary}{\linewidth}[t]{|T|}
\hline
\sphinxstyletheadfamily 
\sphinxincludegraphics[width=0.490\linewidth]{{reflection}.png} \sphinxincludegraphics[width=0.500\linewidth]{{reflection_law}.png}
\\
\hline
\sphinxstylestrong{Fig.:} Law of reflection
\\
\hline
\end{tabulary}
\par
\sphinxattableend\end{savenotes}

The law of reflection tells us now, that the outgoing reflected ray is now leaving the surface under an angle \(\theta_2=\theta_1\). So both angles are the same for the reflection.

\begin{sphinxadmonition}{note}{}\unskip
\sphinxstylestrong{Law of Reflection}

If a ray is incident to a reflecting surface under an angle \(\theta_1\) it will be reflected towards under an angle \(\theta_2=\theta_1\) to the same side of the surface.
\end{sphinxadmonition}


\subsection{Fermat’s Principle}
\label{\detokenize{notebooks/L1/Reflection:Fermat_u2019s-Principle}}
The law of reflection can be actually obtained from a variational principle saying the light rays propagate along those path on which the propagation time is an extremum. This variational principle is called Fermat’s principle.


\begin{savenotes}\sphinxattablestart
\centering
\begin{tabulary}{\linewidth}[t]{|T|}
\hline
\sphinxstyletheadfamily 
\sphinxincludegraphics[width=0.600\linewidth]{{fermat}.png}
\\
\hline
\sphinxstylestrong{Fig.:} Sketch for deriving the law of reflection from Fermat’s principle
\\
\hline
\end{tabulary}
\par
\sphinxattableend\end{savenotes}

Consider now a light ray that should travel from point \(A\) to point \(C\) via a point \(B\) on the mirror surface. In general multiple paths are possible such as the one indicated in the above picture. Clearly this path is not satisfying our reflection law formulated above. Fermat’s principle now restricts the path length from \(A\) to \(C\) to be the one, which takes the least amount of time.

\begin{sphinxadmonition}{note}{}\unskip
\sphinxstylestrong{Fermat’s principle}

The path taken by a ray between two given points A, B is the path that can be traversed in the least time.

\sphinxstyleemphasis{More precise alternative:} A ray going in a certain particular path has the property that if we make a small change in the ray in any manner whatever, say in the location at which it comes to the mirror, or the shape of the curve, or anything, there will be no first\sphinxhyphen{}order change in the time; there will be only a second\sphinxhyphen{}order change in the time.
\end{sphinxadmonition}

So let us consider that contraints to the path length. The total length the light hast to travel via the three points is

\begin{equation}
l=l_{1}+l_{2}=\sqrt{(x-x_1)^2+y_1^2}+\sqrt{(x_2-x)^2+y_2^2}.
\end{equation}

The time that is required by the light to travel that distance \(l\) is then given by

\begin{equation}
t=\frac{l}{c},
\end{equation}

where \(c\) is the speed of light in the medium above the mirror. If this time should now be a minimum, we have to take the derivative of the time \(t\) with respect to the position \(x\) on the mirror and set that to zero, i.e.,
\begin{equation*}
\begin{split}\frac{\mathrm dt}{\mathrm dx}=0.\end{split}
\end{equation*}
This results in \begin{equation}
\frac{x-x_1}{\sqrt{(x-x_1)^2+y_{1}^2}}=\frac{x_2-x}{\sqrt{(x_2-x)^2+y_{2}^2}},
\end{equation}

which is actually

\begin{equation}
\frac{x-x_1}{l_1}=\frac{x_2-x}{l_2}
\end{equation}

or

\begin{equation}
\sin(\theta_1)=\sin(\theta_2)
\end{equation}

which finally requires
\begin{equation*}
\begin{split}\theta_1=\theta_2\end{split}
\end{equation*}
and is our law of reflection. Thus, refraction (of course) satisfies Fermat’s principle.

While this has been a special example of how to apply Fermat’s principle we can define a more general version of it correponding to the following situation also involving an inhomogeneous refractive index \(n(\vec{r})\).


\begin{savenotes}\sphinxattablestart
\centering
\begin{tabulary}{\linewidth}[t]{|T|}
\hline
\sphinxstyletheadfamily 
\sphinxincludegraphics[width=0.400\linewidth]{{fermat_general}.png}
\\
\hline
\sphinxstylestrong{Fig.:} Sketch for a general description of Fermat’s principle.
\\
\hline
\end{tabulary}
\par
\sphinxattableend\end{savenotes}

For this general scenario of light traveling along a path, Fermat’s principle is given by

\begin{equation}
\delta \int\limits_{A}^{C} n(\vec{r}) \mathrm ds=0,
\end{equation}

where the \(\delta\) denotes a variation of the path indicated.

The following section was created from \sphinxcode{\sphinxupquote{source/notebooks/L1/Refraction.ipynb}}.


\section{Refraction}
\label{\detokenize{notebooks/L1/Refraction:Refraction}}\label{\detokenize{notebooks/L1/Refraction::doc}}
The law of refraction is the second important law of geometrical optics. It connects the refractive index \(n_1\) at the incident side and the angle of incidence \(\theta_1\) to the refractive index \(n_2\) and the angle of refraction \(\theta_2\) on the transmission side. It will later turn out, that both laws, the law of reflection and the law of refraction actually correspond to a conservation of momentum.


\subsection{Refractive Index}
\label{\detokenize{notebooks/L1/Refraction:Refractive-Index}}
The refractive index \(n\) is a material constant representing the factor by which the speed of light is slowed down in the medium. As the maximum speed is the speed of light in vacuum, the refractive index is typically a number \(n \ge 1\). Yet, it will turn out later, that the refractive index can be smaller than 1 or even negative. This, however, requires first to understand the origin of the refractive index.


\subsection{Snells Law}
\label{\detokenize{notebooks/L1/Refraction:Snells-Law}}

\begin{savenotes}\sphinxattablestart
\centering
\begin{tabulary}{\linewidth}[t]{|T|}
\hline
\sphinxstyletheadfamily 
\sphinxincludegraphics[width=0.590\linewidth]{{snell}.png} \sphinxincludegraphics[width=0.400\linewidth]{{refraction_law}.png}
\\
\hline
\sphinxstylestrong{Fig.:} Snell’s law.
\\
\hline
\end{tabulary}
\par
\sphinxattableend\end{savenotes}

\begin{sphinxadmonition}{note}{}\unskip
\sphinxstylestrong{Law of Reflection (Snells Law)}

The law of refraction (Snell’s law) is given for the above sketch by

\begin{equation}
n_1 \sin(\theta_1)=n_2 \sin(\theta_2)
\end{equation}
\end{sphinxadmonition}


\bigskip\hrule\bigskip


In the previous section in refraction we have derived the law of reflection from Fermat’s principle. \sphinxstylestrong{Try yourself to obtain Snell’s law with the help of Fermat’s principle}. Check out the \sphinxhref{https://mybinder.org/v2/gh/fcichos/exp3/master?urlpath=tree/source/snippets/Refraction\%20Explorer.ipynb}{Refraction Explorer Snippet} if you want to play around with Snells law. You may tray there even negative refraction.


\bigskip\hrule\bigskip


According to Snell’s law, there is a general behavior of the corresponding angles, which you might want to remember (see also Fig. above). Consider the following cases:

\(n_1<n_2\): \sphinxhyphen{} refraction is towards the optical axis \sphinxhyphen{} \(\theta_2<\theta_1\)

\(n_2<n_1\): \sphinxhyphen{} refraction is away from the optical axis \sphinxhyphen{} \(\theta_1<\theta_2\)

The plot below shows this result in three plots with varying incident angle and two different refractive index combinations (glass/air, ait/glass).


\begin{savenotes}\sphinxattablestart
\centering
\begin{tabulary}{\linewidth}[t]{|T|}
\hline
\sphinxstyletheadfamily 
\sphinxincludegraphics[width=0.600\linewidth]{{angles}.png}
\\
\hline
\sphinxstylestrong{Fig.:} Snells law for different combinations of refractive indices.
\\
\hline
\end{tabulary}
\par
\sphinxattableend\end{savenotes}


\subsection{Total Internal Reflection}
\label{\detokenize{notebooks/L1/Refraction:Total-Internal-Reflection}}
The above diagram reveals a special case occurring when \(n_1>n_2\). Under these condition, one may change the incident angle \(\theta_1\) such that the outgoing angle becomes \(\theta_2=\frac{\pi}{2}\). For any incident angle \(\theta_1\) larger than this critical angle, the is no refracted ray anymore, but just a reflected ray. This is despite the fact that the material with \(n_2\) is completely transparent. This phenomenon is called \sphinxstylestrong{total internal reflection} and it
has several important applications.

Let’s first formalize this. Using the Snell’s law. For \(\theta_2=\frac{\pi}{2}\) we obtain
\begin{equation*}
\begin{split}\theta_1=\theta_c=\sin^{-1}\left (\frac{n_2}{n_1}\right )\end{split}
\end{equation*}
for the critical angle \(\theta_c\). As the \(\sin^{-1}()\) requires an argument \(\le1\), this works only if we have \(n_2 < n_1\)


\begin{savenotes}\sphinxattablestart
\centering
\begin{tabulary}{\linewidth}[t]{|T|}
\hline
\sphinxstyletheadfamily 
\sphinxincludegraphics[width=0.500\linewidth]{{tir}.png} \sphinxincludegraphics[width=0.490\linewidth]{{tir_disk}.png}
\\
\hline
\sphinxstylestrong{Fig.:} Total internal reflection
\\
\hline
\end{tabulary}
\par
\sphinxattableend\end{savenotes}

\begin{sphinxadmonition}{note}{}\unskip
\sphinxstylestrong{Total Internal Reflection}

Total internal reflection occurs when light is passing from higher refractive index to lower refractive index materials for incidence angle larger than a critical angle

\begin{equation}
\theta_c=\sin^{-1}\left (\frac{n_2}{n_1}\right )
\end{equation}
\end{sphinxadmonition}

We can demonstrate total internal reflection very easily with a water basin, for example, where we couple in light from a laser from the side.


\begin{savenotes}\sphinxattablestart
\centering
\begin{tabulary}{\linewidth}[t]{|T|}
\hline
\sphinxstyletheadfamily 
\sphinxincludegraphics[width=0.590\linewidth]{{basin_tir}.png} \sphinxincludegraphics[width=0.400\linewidth]{{tir_basin}.png}
\\
\hline
\sphinxstylestrong{Fig.:} Total internal reflection at a water/air interface.
\\
\hline
\end{tabulary}
\par
\sphinxattableend\end{savenotes}

But you could try that yourself also in the bath tub diving below the water surface.

\sphinxstylestrong{Optical Fiber} Total internal reflection is very important for guiding light in telecommunications, for example. There, glass wires with a diameter from a few to several 100 µm are used to transport light. The glass wire with a central core of refractive index \(n_1\) is surrounded by a cladding layer of slightly lower refractive index \(n_2\). Light is then coupled into the fiber from one side. To obtain total internal reflection in this setting, the incident rays have to hit the
front of the fiber at a maximum angle \(\theta_{a}\)


\begin{savenotes}\sphinxattablestart
\centering
\begin{tabulary}{\linewidth}[t]{|T|}
\hline
\sphinxstyletheadfamily 
\sphinxincludegraphics[width=0.590\linewidth]{{fiber}.png} \sphinxincludegraphics[width=0.400\linewidth]{{tir_rod}.png}
\\
\hline
\sphinxstylestrong{Fig.:} Total internal reflection in an optical fiber and a glass rod.
\\
\hline
\end{tabulary}
\par
\sphinxattableend\end{savenotes}

The angle \(\theta_{a}\) can be easily calculated from Snells law. To characterize this opening angle one typically defines a new quantity called numerical aperature \(NA\), which is the sine of the opening angle \(\theta_a\)

\begin{equation}
NA=\sin(\theta_a)=\sqrt{n_1^2-n_2^{2}}
\end{equation}

Using typical values of the refractive indices \(n_1=1.475\) and \(n_2=1.46\) one obtains a numerical apeture of \(NA\approx 0.2\).

The following section was created from \sphinxcode{\sphinxupquote{source/snippets/Refraction Explorer.ipynb}}.


\section{EXP3 Snippets \textendash{} Refraction Explorer}
\label{\detokenize{snippets/Refraction Explorer:EXP3-Snippets-_-Refraction-Explorer}}\label{\detokenize{snippets/Refraction Explorer::doc}}
This is a small python code snippet, which you can explore on the \sphinxstyleemphasis{myBinder} service with the button on the top of this webpage. It shows the refraction of a light ray (red) incident to an interface (horizontal line), which is then refracted. The interface is seperating two areas with different refractive index \(n_1, n_2\), which you can modify with the sliders in the same way as the incident angle. The refractive index \(n_2\) may even go negative and you may want to explore what
happens then.
\begin{enumerate}
\sphinxsetlistlabels{\alph}{enumi}{enumii}{(}{)}%
\setcounter{enumi}{2}
\item {} \begin{enumerate}
\sphinxsetlistlabels{\Alph}{enumii}{enumiii}{}{.}%
\setcounter{enumii}{5}
\item {} 
Cichos 2020

\end{enumerate}

\end{enumerate}

{
\sphinxsetup{VerbatimColor={named}{nbsphinx-code-bg}}
\sphinxsetup{VerbatimBorderColor={named}{nbsphinx-code-border}}
\begin{sphinxVerbatim}[commandchars=\\\{\}]
\llap{\color{nbsphinxin}[1]:\,\hspace{\fboxrule}\hspace{\fboxsep}}\PYG{o}{\PYGZpc{}}\PYG{k}{matplotlib} widget
\PYG{k+kn}{import} \PYG{n+nn}{ipywidgets} \PYG{k}{as} \PYG{n+nn}{widgets}
\PYG{k+kn}{import} \PYG{n+nn}{matplotlib}\PYG{n+nn}{.}\PYG{n+nn}{pyplot} \PYG{k}{as} \PYG{n+nn}{plt}
\PYG{k+kn}{import} \PYG{n+nn}{numpy} \PYG{k}{as} \PYG{n+nn}{np}
\end{sphinxVerbatim}
}

{
\sphinxsetup{VerbatimColor={named}{nbsphinx-code-bg}}
\sphinxsetup{VerbatimBorderColor={named}{nbsphinx-code-border}}
\begin{sphinxVerbatim}[commandchars=\\\{\}]
\llap{\color{nbsphinxin}[2]:\,\hspace{\fboxrule}\hspace{\fboxsep}}\PYG{k}{def} \PYG{n+nf}{magnitude}\PYG{p}{(}\PYG{n}{vector}\PYG{p}{)}\PYG{p}{:}
   \PYG{k}{return} \PYG{n}{np}\PYG{o}{.}\PYG{n}{sqrt}\PYG{p}{(}\PYG{n}{np}\PYG{o}{.}\PYG{n}{dot}\PYG{p}{(}\PYG{n}{np}\PYG{o}{.}\PYG{n}{array}\PYG{p}{(}\PYG{n}{vector}\PYG{p}{)}\PYG{p}{,}\PYG{n}{np}\PYG{o}{.}\PYG{n}{array}\PYG{p}{(}\PYG{n}{vector}\PYG{p}{)}\PYG{p}{)}\PYG{p}{)}

\PYG{k}{def} \PYG{n+nf}{norm}\PYG{p}{(}\PYG{n}{vector}\PYG{p}{)}\PYG{p}{:}
   \PYG{k}{return} \PYG{n}{np}\PYG{o}{.}\PYG{n}{array}\PYG{p}{(}\PYG{n}{vector}\PYG{p}{)}\PYG{o}{/}\PYG{n}{magnitude}\PYG{p}{(}\PYG{n}{np}\PYG{o}{.}\PYG{n}{array}\PYG{p}{(}\PYG{n}{vector}\PYG{p}{)}\PYG{p}{)}

\PYG{k}{def} \PYG{n+nf}{lineRayIntersectionPoint}\PYG{p}{(}\PYG{n}{rayOrigin}\PYG{p}{,} \PYG{n}{rayDirection}\PYG{p}{,} \PYG{n}{point1}\PYG{p}{,} \PYG{n}{point2}\PYG{p}{)}\PYG{p}{:}
        \PYG{c+c1}{\PYGZsh{} Convert to numpy arrays}
    \PYG{n}{rayOrigin} \PYG{o}{=} \PYG{n}{np}\PYG{o}{.}\PYG{n}{array}\PYG{p}{(}\PYG{n}{rayOrigin}\PYG{p}{,} \PYG{n}{dtype}\PYG{o}{=}\PYG{n}{np}\PYG{o}{.}\PYG{n}{float}\PYG{p}{)}
    \PYG{n}{rayDirection} \PYG{o}{=} \PYG{n}{np}\PYG{o}{.}\PYG{n}{array}\PYG{p}{(}\PYG{n}{norm}\PYG{p}{(}\PYG{n}{rayDirection}\PYG{p}{)}\PYG{p}{,} \PYG{n}{dtype}\PYG{o}{=}\PYG{n}{np}\PYG{o}{.}\PYG{n}{float}\PYG{p}{)}
    \PYG{n}{point1} \PYG{o}{=} \PYG{n}{np}\PYG{o}{.}\PYG{n}{array}\PYG{p}{(}\PYG{n}{point1}\PYG{p}{,} \PYG{n}{dtype}\PYG{o}{=}\PYG{n}{np}\PYG{o}{.}\PYG{n}{float}\PYG{p}{)}
    \PYG{n}{point2} \PYG{o}{=} \PYG{n}{np}\PYG{o}{.}\PYG{n}{array}\PYG{p}{(}\PYG{n}{point2}\PYG{p}{,} \PYG{n}{dtype}\PYG{o}{=}\PYG{n}{np}\PYG{o}{.}\PYG{n}{float}\PYG{p}{)}

    \PYG{c+c1}{\PYGZsh{} Ray\PYGZhy{}Line Segment Intersection Test in 2D}
    \PYG{c+c1}{\PYGZsh{} http://bit.ly/1CoxdrG}
    \PYG{n}{v1} \PYG{o}{=} \PYG{n}{rayOrigin} \PYG{o}{\PYGZhy{}} \PYG{n}{point1}
    \PYG{n}{v2} \PYG{o}{=} \PYG{n}{point2} \PYG{o}{\PYGZhy{}} \PYG{n}{point1}
    \PYG{n}{v3} \PYG{o}{=} \PYG{n}{np}\PYG{o}{.}\PYG{n}{array}\PYG{p}{(}\PYG{p}{[}\PYG{o}{\PYGZhy{}}\PYG{n}{rayDirection}\PYG{p}{[}\PYG{l+m+mi}{1}\PYG{p}{]}\PYG{p}{,} \PYG{n}{rayDirection}\PYG{p}{[}\PYG{l+m+mi}{0}\PYG{p}{]}\PYG{p}{]}\PYG{p}{)}
    \PYG{n}{t1} \PYG{o}{=} \PYG{n}{np}\PYG{o}{.}\PYG{n}{cross}\PYG{p}{(}\PYG{n}{v2}\PYG{p}{,} \PYG{n}{v1}\PYG{p}{)} \PYG{o}{/} \PYG{n}{np}\PYG{o}{.}\PYG{n}{dot}\PYG{p}{(}\PYG{n}{v2}\PYG{p}{,} \PYG{n}{v3}\PYG{p}{)}
    \PYG{n}{t2} \PYG{o}{=} \PYG{n}{np}\PYG{o}{.}\PYG{n}{dot}\PYG{p}{(}\PYG{n}{v1}\PYG{p}{,} \PYG{n}{v3}\PYG{p}{)} \PYG{o}{/} \PYG{n}{np}\PYG{o}{.}\PYG{n}{dot}\PYG{p}{(}\PYG{n}{v2}\PYG{p}{,} \PYG{n}{v3}\PYG{p}{)}
    \PYG{k}{if} \PYG{n}{t1} \PYG{o}{\PYGZgt{}}\PYG{o}{=} \PYG{l+m+mf}{0.0} \PYG{o+ow}{and} \PYG{n}{t2} \PYG{o}{\PYGZgt{}}\PYG{o}{=} \PYG{l+m+mf}{0.0} \PYG{o+ow}{and} \PYG{n}{t2} \PYG{o}{\PYGZlt{}}\PYG{o}{=} \PYG{l+m+mf}{1.0}\PYG{p}{:}
        \PYG{k}{return} \PYG{p}{[}\PYG{n}{rayOrigin} \PYG{o}{+} \PYG{n}{t1} \PYG{o}{*} \PYG{n}{rayDirection}\PYG{p}{]}
    \PYG{k}{return} \PYG{p}{[}\PYG{p}{]}


\end{sphinxVerbatim}
}

{
\sphinxsetup{VerbatimColor={named}{nbsphinx-code-bg}}
\sphinxsetup{VerbatimBorderColor={named}{nbsphinx-code-border}}
\begin{sphinxVerbatim}[commandchars=\\\{\}]
\llap{\color{nbsphinxin}[3]:\,\hspace{\fboxrule}\hspace{\fboxsep}}\PYG{n}{fig}\PYG{p}{,} \PYG{n}{ax} \PYG{o}{=} \PYG{n}{plt}\PYG{o}{.}\PYG{n}{subplots}\PYG{p}{(}\PYG{n}{figsize}\PYG{o}{=}\PYG{p}{(}\PYG{l+m+mi}{4}\PYG{p}{,} \PYG{l+m+mi}{4}\PYG{p}{)}\PYG{p}{)}
\PYG{n}{fig}\PYG{o}{.}\PYG{n}{canvas}\PYG{o}{.}\PYG{n}{header\PYGZus{}visible} \PYG{o}{=} \PYG{k+kc}{False}

\PYG{n+nd}{@widgets}\PYG{o}{.}\PYG{n}{interact}\PYG{p}{(}\PYG{n}{n1}\PYG{o}{=}\PYG{p}{(}\PYG{l+m+mi}{1}\PYG{p}{,}\PYG{l+m+mi}{2}\PYG{p}{,}\PYG{l+m+mf}{0.01}\PYG{p}{)}\PYG{p}{,}\PYG{n}{n2}\PYG{o}{=}\PYG{p}{(}\PYG{o}{\PYGZhy{}}\PYG{l+m+mi}{2}\PYG{p}{,}\PYG{l+m+mi}{3}\PYG{p}{,}\PYG{l+m+mf}{0.01}\PYG{p}{)}\PYG{p}{,} \PYG{n}{phi}\PYG{o}{=}\PYG{p}{(}\PYG{l+m+mi}{0}\PYG{p}{,} \PYG{l+m+mi}{90}\PYG{p}{,} \PYG{l+m+mf}{0.1}\PYG{p}{)}\PYG{p}{)}
\PYG{k}{def} \PYG{n+nf}{update}\PYG{p}{(}\PYG{n}{n1}\PYG{o}{=}\PYG{l+m+mi}{1}\PYG{p}{,}\PYG{n}{n2}\PYG{o}{=}\PYG{l+m+mf}{1.5}\PYG{p}{,}\PYG{n}{phi}\PYG{o}{=}\PYG{l+m+mi}{45}\PYG{p}{)}\PYG{p}{:}
    \PYG{l+s+sd}{\PYGZdq{}\PYGZdq{}\PYGZdq{}Remove old lines from plot and plot new one\PYGZdq{}\PYGZdq{}\PYGZdq{}}
    \PYG{n}{ax}\PYG{o}{.}\PYG{n}{cla}\PYG{p}{(}\PYG{p}{)}
    \PYG{n}{theta1}\PYG{o}{=}\PYG{n}{phi}\PYG{o}{*}\PYG{n}{np}\PYG{o}{.}\PYG{n}{pi}\PYG{o}{/}\PYG{l+m+mi}{180}
    \PYG{k}{if} \PYG{n}{n1}\PYG{o}{*}\PYG{n}{np}\PYG{o}{.}\PYG{n}{sin}\PYG{p}{(}\PYG{n}{theta1}\PYG{p}{)}\PYG{o}{/}\PYG{n}{n2}\PYG{o}{\PYGZlt{}}\PYG{o}{=}\PYG{l+m+mi}{1}\PYG{p}{:}
        \PYG{n}{theta2}\PYG{o}{=}\PYG{n}{np}\PYG{o}{.}\PYG{n}{arcsin}\PYG{p}{(}\PYG{n}{n1}\PYG{o}{*}\PYG{n}{np}\PYG{o}{.}\PYG{n}{sin}\PYG{p}{(}\PYG{n}{theta1}\PYG{p}{)}\PYG{o}{/}\PYG{n}{n2}\PYG{p}{)}
    \PYG{k}{else}\PYG{p}{:}
        \PYG{n}{theta2}\PYG{o}{=}\PYG{o}{\PYGZhy{}}\PYG{n}{theta1}\PYG{o}{+}\PYG{n}{np}\PYG{o}{.}\PYG{n}{pi}
    \PYG{n}{ax}\PYG{o}{.}\PYG{n}{set\PYGZus{}title}\PYG{p}{(}\PYG{l+s+s2}{\PYGZdq{}}\PYG{l+s+s2}{Refraction Explorer}\PYG{l+s+s2}{\PYGZdq{}}\PYG{p}{)}
    \PYG{n}{ax}\PYG{o}{.}\PYG{n}{axvline}\PYG{p}{(}\PYG{n}{x}\PYG{o}{=}\PYG{l+m+mi}{0}\PYG{p}{,}\PYG{n}{ls}\PYG{o}{=}\PYG{l+s+s1}{\PYGZsq{}}\PYG{l+s+s1}{\PYGZhy{}\PYGZhy{}}\PYG{l+s+s1}{\PYGZsq{}}\PYG{p}{)}
    \PYG{n}{ax}\PYG{o}{.}\PYG{n}{text}\PYG{p}{(}\PYG{o}{\PYGZhy{}}\PYG{l+m+mf}{0.04}\PYG{p}{,}\PYG{l+m+mf}{0.04}\PYG{p}{,}\PYG{l+s+sa}{r}\PYG{l+s+s1}{\PYGZsq{}}\PYG{l+s+s1}{\PYGZdl{}n\PYGZus{}2\PYGZdl{}=}\PYG{l+s+si}{\PYGZob{}\PYGZcb{}}\PYG{l+s+s1}{\PYGZsq{}}\PYG{o}{.}\PYG{n}{format}\PYG{p}{(}\PYG{n}{n2}\PYG{p}{)}\PYG{p}{)}
    \PYG{n}{ax}\PYG{o}{.}\PYG{n}{text}\PYG{p}{(}\PYG{o}{\PYGZhy{}}\PYG{l+m+mf}{0.04}\PYG{p}{,}\PYG{o}{\PYGZhy{}}\PYG{l+m+mf}{0.04}\PYG{p}{,}\PYG{l+s+sa}{r}\PYG{l+s+s1}{\PYGZsq{}}\PYG{l+s+s1}{\PYGZdl{}n\PYGZus{}1\PYGZdl{}=}\PYG{l+s+si}{\PYGZob{}\PYGZcb{}}\PYG{l+s+s1}{\PYGZsq{}}\PYG{o}{.}\PYG{n}{format}\PYG{p}{(}\PYG{n}{n1}\PYG{p}{)}\PYG{p}{)}
    \PYG{n}{ax}\PYG{o}{.}\PYG{n}{text}\PYG{p}{(}\PYG{l+m+mf}{0.03}\PYG{p}{,}\PYG{l+m+mf}{0.04}\PYG{p}{,}\PYG{l+s+sa}{r}\PYG{l+s+s1}{\PYGZsq{}}\PYG{l+s+s1}{\PYGZdl{}}\PYG{l+s+s1}{\PYGZbs{}}\PYG{l+s+s1}{theta\PYGZus{}2\PYGZdl{}=}\PYG{l+s+si}{\PYGZob{}\PYGZcb{}}\PYG{l+s+s1}{\PYGZsq{}}\PYG{o}{.}\PYG{n}{format}\PYG{p}{(}\PYG{n+nb}{round}\PYG{p}{(}\PYG{n}{theta2}\PYG{o}{*}\PYG{l+m+mi}{180}\PYG{o}{/}\PYG{n}{np}\PYG{o}{.}\PYG{n}{pi}\PYG{p}{)}\PYG{p}{,}\PYG{l+m+mi}{1}\PYG{p}{)}\PYG{p}{)}
    \PYG{n}{ax}\PYG{o}{.}\PYG{n}{text}\PYG{p}{(}\PYG{l+m+mf}{0.03}\PYG{p}{,}\PYG{o}{\PYGZhy{}}\PYG{l+m+mf}{0.04}\PYG{p}{,}\PYG{l+s+sa}{r}\PYG{l+s+s1}{\PYGZsq{}}\PYG{l+s+s1}{\PYGZdl{}}\PYG{l+s+s1}{\PYGZbs{}}\PYG{l+s+s1}{theta\PYGZus{}1\PYGZdl{}=}\PYG{l+s+si}{\PYGZob{}\PYGZcb{}}\PYG{l+s+s1}{\PYGZsq{}}\PYG{o}{.}\PYG{n}{format}\PYG{p}{(}\PYG{n+nb}{round}\PYG{p}{(}\PYG{n}{theta1}\PYG{o}{*}\PYG{l+m+mi}{180}\PYG{o}{/}\PYG{n}{np}\PYG{o}{.}\PYG{n}{pi}\PYG{p}{)}\PYG{p}{,}\PYG{l+m+mi}{1}\PYG{p}{)}\PYG{p}{)}


    \PYG{n}{ax}\PYG{o}{.}\PYG{n}{axhline}\PYG{p}{(}\PYG{n}{y}\PYG{o}{=}\PYG{l+m+mi}{0}\PYG{p}{,}\PYG{n}{color}\PYG{o}{=}\PYG{l+s+s1}{\PYGZsq{}}\PYG{l+s+s1}{k}\PYG{l+s+s1}{\PYGZsq{}}\PYG{p}{,}\PYG{n}{lw}\PYG{o}{=}\PYG{l+m+mf}{0.5}\PYG{p}{)}
    \PYG{n}{ax}\PYG{o}{.}\PYG{n}{quiver}\PYG{p}{(}\PYG{l+m+mi}{0}\PYG{p}{,}\PYG{l+m+mi}{0}\PYG{p}{,}\PYG{n}{np}\PYG{o}{.}\PYG{n}{sin}\PYG{p}{(}\PYG{n}{theta1}\PYG{p}{)}\PYG{p}{,}\PYG{n}{np}\PYG{o}{.}\PYG{n}{cos}\PYG{p}{(}\PYG{n}{theta1}\PYG{p}{)}\PYG{p}{,}\PYG{n}{scale}\PYG{o}{=}\PYG{l+m+mi}{3}\PYG{p}{,}\PYG{n}{pivot}\PYG{o}{=}\PYG{l+s+s1}{\PYGZsq{}}\PYG{l+s+s1}{tip}\PYG{l+s+s1}{\PYGZsq{}} \PYG{p}{,}\PYG{n}{color}\PYG{o}{=}\PYG{l+s+s1}{\PYGZsq{}}\PYG{l+s+s1}{red}\PYG{l+s+s1}{\PYGZsq{}}\PYG{p}{)}
    \PYG{n}{ax}\PYG{o}{.}\PYG{n}{quiver}\PYG{p}{(}\PYG{l+m+mi}{0}\PYG{p}{,}\PYG{l+m+mi}{0}\PYG{p}{,}\PYG{n}{np}\PYG{o}{.}\PYG{n}{sin}\PYG{p}{(}\PYG{n}{theta2}\PYG{p}{)}\PYG{p}{,}\PYG{n}{np}\PYG{o}{.}\PYG{n}{cos}\PYG{p}{(}\PYG{n}{theta2}\PYG{p}{)}\PYG{p}{,}\PYG{n}{scale}\PYG{o}{=}\PYG{l+m+mi}{3}\PYG{p}{,}\PYG{n}{color}\PYG{o}{=}\PYG{l+s+s1}{\PYGZsq{}}\PYG{l+s+s1}{blue}\PYG{l+s+s1}{\PYGZsq{}}\PYG{p}{)}

\end{sphinxVerbatim}
}

{

\kern-\sphinxverbatimsmallskipamount\kern-\baselineskip
\kern+\FrameHeightAdjust\kern-\fboxrule
\vspace{\nbsphinxcodecellspacing}

\sphinxsetup{VerbatimColor={named}{white}}
\sphinxsetup{VerbatimBorderColor={named}{nbsphinx-code-border}}
\begin{sphinxVerbatim}[commandchars=\\\{\}]
Canvas(toolbar=Toolbar(toolitems=[('Home', 'Reset original view', 'home', 'home'), ('Back', 'Back to previous …
\end{sphinxVerbatim}
}

{

\kern-\sphinxverbatimsmallskipamount\kern-\baselineskip
\kern+\FrameHeightAdjust\kern-\fboxrule
\vspace{\nbsphinxcodecellspacing}

\sphinxsetup{VerbatimColor={named}{white}}
\sphinxsetup{VerbatimBorderColor={named}{nbsphinx-code-border}}
\begin{sphinxVerbatim}[commandchars=\\\{\}]
interactive(children=(FloatSlider(value=1.0, description='n1', max=2.0, min=1.0, step=0.01), FloatSlider(value…
\end{sphinxVerbatim}
}

\sphinxincludegraphics[width=400\sphinxpxdimen,height=400\sphinxpxdimen]{{refraction_explorer}.png}

{
\sphinxsetup{VerbatimColor={named}{nbsphinx-code-bg}}
\sphinxsetup{VerbatimBorderColor={named}{nbsphinx-code-border}}
\begin{sphinxVerbatim}[commandchars=\\\{\}]
\llap{\color{nbsphinxin}[ ]:\,\hspace{\fboxrule}\hspace{\fboxsep}}
\end{sphinxVerbatim}
}


\section{Lecture Contents}
\label{\detokenize{lectures/L2/overview_2:lecture-contents}}\label{\detokenize{lectures/L2/overview_2::doc}}
In Lecture 2 we will discuss several optical elements, which help us to construct optical instruments like spectrometers, microscopes or telescopes.
Such optical elements are prisms, lenses and curved mirrors.

\noindent\sphinxincludegraphics[width=600\sphinxpxdimen]{{slides5}.png}

Lecture 2 slides for download \sphinxcode{\sphinxupquote{pdf}}

The following section was created from \sphinxcode{\sphinxupquote{source/notebooks/L2/Optical Elements.ipynb}}.


\section{Optical Elements Part I}
\label{\detokenize{notebooks/L2/Optical Elements:Optical-Elements-Part-I}}\label{\detokenize{notebooks/L2/Optical Elements::doc}}
{
\sphinxsetup{VerbatimColor={named}{nbsphinx-code-bg}}
\sphinxsetup{VerbatimBorderColor={named}{nbsphinx-code-border}}
\begin{sphinxVerbatim}[commandchars=\\\{\}]
\llap{\color{nbsphinxin}[29]:\,\hspace{\fboxrule}\hspace{\fboxsep}}\PYG{c+c1}{\PYGZsh{}\PYGZsh{} just for plotting later}

\PYG{k+kn}{import} \PYG{n+nn}{pandas} \PYG{k}{as} \PYG{n+nn}{pd}
\PYG{k+kn}{import} \PYG{n+nn}{matplotlib}\PYG{n+nn}{.}\PYG{n+nn}{pyplot} \PYG{k}{as} \PYG{n+nn}{plt}
\PYG{o}{\PYGZpc{}}\PYG{k}{matplotlib} inline
\PYG{o}{\PYGZpc{}}\PYG{k}{config} InlineBackend.figure\PYGZus{}format = \PYGZsq{}retina\PYGZsq{}

\PYG{n}{plt}\PYG{o}{.}\PYG{n}{rcParams}\PYG{o}{.}\PYG{n}{update}\PYG{p}{(}\PYG{p}{\PYGZob{}}\PYG{l+s+s1}{\PYGZsq{}}\PYG{l+s+s1}{font.size}\PYG{l+s+s1}{\PYGZsq{}}\PYG{p}{:} \PYG{l+m+mi}{12}\PYG{p}{,}
                     \PYG{l+s+s1}{\PYGZsq{}}\PYG{l+s+s1}{axes.titlesize}\PYG{l+s+s1}{\PYGZsq{}}\PYG{p}{:} \PYG{l+m+mi}{18}\PYG{p}{,}
                     \PYG{l+s+s1}{\PYGZsq{}}\PYG{l+s+s1}{axes.labelsize}\PYG{l+s+s1}{\PYGZsq{}}\PYG{p}{:} \PYG{l+m+mi}{16}\PYG{p}{,}
                     \PYG{l+s+s1}{\PYGZsq{}}\PYG{l+s+s1}{axes.labelpad}\PYG{l+s+s1}{\PYGZsq{}}\PYG{p}{:} \PYG{l+m+mi}{14}\PYG{p}{,}
                     \PYG{l+s+s1}{\PYGZsq{}}\PYG{l+s+s1}{lines.linewidth}\PYG{l+s+s1}{\PYGZsq{}}\PYG{p}{:} \PYG{l+m+mi}{1}\PYG{p}{,}
                     \PYG{l+s+s1}{\PYGZsq{}}\PYG{l+s+s1}{lines.markersize}\PYG{l+s+s1}{\PYGZsq{}}\PYG{p}{:} \PYG{l+m+mi}{10}\PYG{p}{,}
                     \PYG{l+s+s1}{\PYGZsq{}}\PYG{l+s+s1}{xtick.labelsize}\PYG{l+s+s1}{\PYGZsq{}} \PYG{p}{:} \PYG{l+m+mi}{16}\PYG{p}{,}
                     \PYG{l+s+s1}{\PYGZsq{}}\PYG{l+s+s1}{ytick.labelsize}\PYG{l+s+s1}{\PYGZsq{}} \PYG{p}{:} \PYG{l+m+mi}{16}\PYG{p}{,}
                     \PYG{l+s+s1}{\PYGZsq{}}\PYG{l+s+s1}{xtick.top}\PYG{l+s+s1}{\PYGZsq{}} \PYG{p}{:} \PYG{k+kc}{True}\PYG{p}{,}
                     \PYG{l+s+s1}{\PYGZsq{}}\PYG{l+s+s1}{xtick.direction}\PYG{l+s+s1}{\PYGZsq{}} \PYG{p}{:} \PYG{l+s+s1}{\PYGZsq{}}\PYG{l+s+s1}{in}\PYG{l+s+s1}{\PYGZsq{}}\PYG{p}{,}
                     \PYG{l+s+s1}{\PYGZsq{}}\PYG{l+s+s1}{ytick.right}\PYG{l+s+s1}{\PYGZsq{}} \PYG{p}{:} \PYG{k+kc}{True}\PYG{p}{,}
                     \PYG{l+s+s1}{\PYGZsq{}}\PYG{l+s+s1}{ytick.direction}\PYG{l+s+s1}{\PYGZsq{}} \PYG{p}{:} \PYG{l+s+s1}{\PYGZsq{}}\PYG{l+s+s1}{in}\PYG{l+s+s1}{\PYGZsq{}}\PYG{p}{,}\PYG{p}{\PYGZcb{}}\PYG{p}{)}
\end{sphinxVerbatim}
}


\subsection{Mirrors}
\label{\detokenize{notebooks/L2/Optical Elements:Mirrors}}

\subsubsection{Plane Mirrors}
\label{\detokenize{notebooks/L2/Optical Elements:Plane-Mirrors}}
If a point \(P\) send out light that is reflected on a mirror as depicted in the image below, the rays reflected from the mirror diverge, but appear to originate from a point \(P^\prime\) behind the mirror. The distance of this image point from the mirror is the same as the one of the object point from the mirror due to the reflection law. An observer, who would thus collect all the reflected rays and image them e.g. onto the retina of his eye will see the point behind the mirror.


\begin{savenotes}\sphinxattablestart
\centering
\begin{tabulary}{\linewidth}[t]{|T|}
\hline
\sphinxstyletheadfamily 
\sphinxincludegraphics[width=0.490\linewidth]{{plane_mirror}.png} \sphinxincludegraphics[width=0.500\linewidth]{{reflection_plane}.png}
\\
\hline
\sphinxstylestrong{Fig.:} Image formation on a plane mirror.
\\
\hline
\end{tabulary}
\par
\sphinxattableend\end{savenotes}

If now many point of an object emit light towards the mirror then this is also true for all of the points and the whole object appears as an image behind the mirror. Consequently, the image has the same size than the object. Therefore the magnification defined as

\begin{equation}
M=\frac{h_{\rm image}}{h_{\rm object}}=1
\end{equation}


\begin{savenotes}\sphinxattablestart
\centering
\begin{tabulary}{\linewidth}[t]{|T|}
\hline
\sphinxstyletheadfamily 
\sphinxincludegraphics[width=0.400\linewidth]{{image_plane_mirror}.png}
\\
\hline
\sphinxstylestrong{Fig.:} Image formation on a plane mirror.
\\
\hline
\end{tabulary}
\par
\sphinxattableend\end{savenotes}


\subsubsection{Concave Mirrors}
\label{\detokenize{notebooks/L2/Optical Elements:Concave-Mirrors}}
When we apply the law of reflection for a concave mirror (i.e. the reflecting side of the mirror is on the inside of the spherical surface), the light parallel to the optical axis at a distance \(h\) is reflected towards the optical axis and crosses the axis at a specifc point \(F\). For symmetry reasons, a ray on the other side of the mirror needs to cross the axis in the same point.


\begin{savenotes}\sphinxattablestart
\centering
\begin{tabulary}{\linewidth}[t]{|T|}
\hline
\sphinxstyletheadfamily 
\sphinxincludegraphics[width=0.400\linewidth]{{concave_mirror}.png}
\\
\hline
\sphinxstylestrong{Fig.:} Ray propagation on a concave mirror.
\\
\hline
\end{tabulary}
\par
\sphinxattableend\end{savenotes}

We may calculate the position of the point \(F\), e.g. the distance from the mirror surface point \(O\), by applying the law of reflection. If the spherical mirror surface has a radius \(R\), then the distance between the center of the sphere \(M\) and the point \(F\) is given by
\begin{equation*}
\begin{split}FM=\frac{R}{2\cos(\alpha)}\end{split}
\end{equation*}
Therefore, we can also calculate the distance of the mirror surafce from the point \(F\), which results in

\begin{equation}
OF=R\left (1-\frac{1}{2\cos(\alpha)}\right)=f
\end{equation}

This distance is the so\sphinxhyphen{}called focal length of the concave mirror \(f\). For small angle \(\alpha\), the above equation yields the so called paraxial limit (all angles are small and the rays are close to the optical axis). In this limit we find \(\cos(\alpha)\approx 1\) and the focal length becomes \(f=R/2\). If we replace the cosine function by \(\cos(\alpha)=\sqrt{1-sin^2(\alpha)}\) with \(\sin(\alpha)=h/R\), we find

\begin{equation}
f=R\left [ 1-\frac{R}{2\sqrt{R^2-h^2}}\right ]
\end{equation}

This equation is telling us, that the focal distance is not a single value for a concave mirror. The focal distance rather changes with the distance \(h\) from the optical axis. If \(h\) approaches \(R\) the focal length become shorter.

To obtain now an equation which predicts the point at which the reflected ray intersects the optical axis if it emerged at a point \(A\), we just consider the following sketch.


\begin{savenotes}\sphinxattablestart
\centering
\begin{tabulary}{\linewidth}[t]{|T|}
\hline
\sphinxstyletheadfamily 
\sphinxincludegraphics[width=0.400\linewidth]{{image_concave_mirror}.png}
\\
\hline
\sphinxstylestrong{Fig.:} Image formation on a concave mirror.
\\
\hline
\end{tabulary}
\par
\sphinxattableend\end{savenotes}

For this situation, we can write down immediately the following relations
\begin{equation*}
\begin{split}\delta=\alpha+\gamma\end{split}
\end{equation*}\begin{equation*}
\begin{split}\gamma+\beta=2\delta\end{split}
\end{equation*}
Further
\begin{equation*}
\begin{split}\tan(\gamma)=\frac{h}{g}\approx \gamma\end{split}
\end{equation*}
and
\begin{equation*}
\begin{split}\tan(\beta)=\frac{h}{g}\approx \beta\end{split}
\end{equation*}
and finally
\begin{equation*}
\begin{split}\sin(\delta)=\frac{h}{R}\approx \delta\end{split}
\end{equation*}
from which we obtain the imaging equation
\begin{equation*}
\begin{split}\frac{1}{g}+\frac{1}{b}\approx \frac{2}{R}\approx \frac{1}{f}\end{split}
\end{equation*}
This equation has some surprising property. It if completely independent of \(h\) and \(\gamma\). That means all points in a plane at a distance \(g\) are images into a plane at a distance \(b\). Both planes are therefore called conjugated planes.

\begin{sphinxadmonition}{note}{}\unskip
\sphinxstylestrong{Imaging Equation Concave Mirror}

The sum of the inverse object and image distances equals the inverse focal length of the cocave mirror.
\begin{equation*}
\begin{split}\frac{1}{g}+\frac{1}{b}\approx\frac{1}{f}\end{split}
\end{equation*}\end{sphinxadmonition}

This equation now helps to construct the image of an object in front of a concave mirror and we may define 3 different rays to identify the size of an image \(h_{\rm image}\) from the size of an object \(h_{\rm object}\).


\begin{savenotes}\sphinxattablestart
\centering
\begin{tabulary}{\linewidth}[t]{|T|}
\hline
\sphinxstyletheadfamily 
\sphinxincludegraphics[width=0.700\linewidth]{{image_size}.png}
\\
\hline
\sphinxstylestrong{Fig.:} Image formation on a concave mirror.
\\
\hline
\end{tabulary}
\par
\sphinxattableend\end{savenotes}

In the above sketch, there are three different rays, which help us to construct the image.
\begin{enumerate}
\sphinxsetlistlabels{\arabic}{enumi}{enumii}{}{.}%
\item {} 
\sphinxstylestrong{red ray:} parallel ray \(\rightarrow\) focal ray

\item {} 
\sphinxstylestrong{green ray:} focal ray \(\rightarrow\) parallel ray

\item {} 
\sphinxstylestrong{central ray:} central ray \(\rightarrow\) central ray

\end{enumerate}

All three reflected rays will either intersect on the same side of the mirror as the object. In this case, the image is called \sphinxstylestrong{real image} and it will be inverted as in the sketch. If the reflected rays on the other hand diverge, the appear to intersect behind the mirror. In this case, the image is called \sphinxstylestrong{virtual image}. It is located behind the mirror. There is no real intersection of rays in that case.

The intersection point further gives the size of the image. Using a ray from the tip of the object to the intersection between mirror and optical axis (O), we may easily determine the image height \(h_{\rm image}\). Try yourself, if you see that the magnification of a concave mirror is described by
\begin{equation*}
\begin{split}\frac{h_{\rm image}}{h_{\rm image}}=-\frac{b}{g}=M\end{split}
\end{equation*}
As mentioned earlier, this resembles to be the magnification \(M\). The minus sign in the ratio of image and object distance appears due to the fact that the image is reversed in its orientation in a real image.

With the help of the imaging equation and the magnification we may in general differentiate between the following general situations:
\begin{enumerate}
\sphinxsetlistlabels{\arabic}{enumi}{enumii}{}{.}%
\item {} 
\(g>2f\): image is real, inverted and smaller than the object

\item {} 
\(g=2f\): image is real, inverted and of the same size than the object

\item {} 
\(f<g<2f\): image is real, inverted and larger than the object

\item {} 
\(g<f\): image is virtual, upright and larger than the object

\end{enumerate}


\section{Lecture Contents}
\label{\detokenize{lectures/L3/overview_3:lecture-contents}}\label{\detokenize{lectures/L3/overview_3::doc}}
In Lecture 3 we will cntinue to discuss several optical elements, which help us to construct optical instruments like spectrometers, microscopes or telescopes. The optical elements which are left, are prism and lenses.

\noindent\sphinxincludegraphics[width=600\sphinxpxdimen]{{slides6}.png}

Lecture 3 slides for download \sphinxcode{\sphinxupquote{pdf}}

The following section was created from \sphinxcode{\sphinxupquote{source/notebooks/L3/Optical Elements.ipynb}}.


\section{Optical Elements Part II}
\label{\detokenize{notebooks/L3/Optical Elements:Optical-Elements-Part-II}}\label{\detokenize{notebooks/L3/Optical Elements::doc}}
{
\sphinxsetup{VerbatimColor={named}{nbsphinx-code-bg}}
\sphinxsetup{VerbatimBorderColor={named}{nbsphinx-code-border}}
\begin{sphinxVerbatim}[commandchars=\\\{\}]
\llap{\color{nbsphinxin}[5]:\,\hspace{\fboxrule}\hspace{\fboxsep}}\PYG{c+c1}{\PYGZsh{}\PYGZsh{} just for plotting later}

\PYG{k+kn}{import} \PYG{n+nn}{pandas} \PYG{k}{as} \PYG{n+nn}{pd}
\PYG{k+kn}{import} \PYG{n+nn}{numpy} \PYG{k}{as} \PYG{n+nn}{np}
\PYG{k+kn}{import} \PYG{n+nn}{matplotlib}\PYG{n+nn}{.}\PYG{n+nn}{pyplot} \PYG{k}{as} \PYG{n+nn}{plt}
\PYG{k+kn}{from} \PYG{n+nn}{spectrumRGB} \PYG{k+kn}{import} \PYG{n}{wavelength\PYGZus{}to\PYGZus{}rgb}
\PYG{o}{\PYGZpc{}}\PYG{k}{matplotlib} inline
\PYG{o}{\PYGZpc{}}\PYG{k}{config} InlineBackend.figure\PYGZus{}format = \PYGZsq{}retina\PYGZsq{}

\PYG{n}{plt}\PYG{o}{.}\PYG{n}{rcParams}\PYG{o}{.}\PYG{n}{update}\PYG{p}{(}\PYG{p}{\PYGZob{}}\PYG{l+s+s1}{\PYGZsq{}}\PYG{l+s+s1}{font.size}\PYG{l+s+s1}{\PYGZsq{}}\PYG{p}{:} \PYG{l+m+mi}{12}\PYG{p}{,}
                     \PYG{l+s+s1}{\PYGZsq{}}\PYG{l+s+s1}{axes.titlesize}\PYG{l+s+s1}{\PYGZsq{}}\PYG{p}{:} \PYG{l+m+mi}{18}\PYG{p}{,}
                     \PYG{l+s+s1}{\PYGZsq{}}\PYG{l+s+s1}{axes.labelsize}\PYG{l+s+s1}{\PYGZsq{}}\PYG{p}{:} \PYG{l+m+mi}{16}\PYG{p}{,}
                     \PYG{l+s+s1}{\PYGZsq{}}\PYG{l+s+s1}{axes.labelpad}\PYG{l+s+s1}{\PYGZsq{}}\PYG{p}{:} \PYG{l+m+mi}{14}\PYG{p}{,}
                     \PYG{l+s+s1}{\PYGZsq{}}\PYG{l+s+s1}{lines.linewidth}\PYG{l+s+s1}{\PYGZsq{}}\PYG{p}{:} \PYG{l+m+mi}{1}\PYG{p}{,}
                     \PYG{l+s+s1}{\PYGZsq{}}\PYG{l+s+s1}{lines.markersize}\PYG{l+s+s1}{\PYGZsq{}}\PYG{p}{:} \PYG{l+m+mi}{10}\PYG{p}{,}
                     \PYG{l+s+s1}{\PYGZsq{}}\PYG{l+s+s1}{xtick.labelsize}\PYG{l+s+s1}{\PYGZsq{}} \PYG{p}{:} \PYG{l+m+mi}{16}\PYG{p}{,}
                     \PYG{l+s+s1}{\PYGZsq{}}\PYG{l+s+s1}{ytick.labelsize}\PYG{l+s+s1}{\PYGZsq{}} \PYG{p}{:} \PYG{l+m+mi}{16}\PYG{p}{,}
                     \PYG{l+s+s1}{\PYGZsq{}}\PYG{l+s+s1}{xtick.top}\PYG{l+s+s1}{\PYGZsq{}} \PYG{p}{:} \PYG{k+kc}{True}\PYG{p}{,}
                     \PYG{l+s+s1}{\PYGZsq{}}\PYG{l+s+s1}{xtick.direction}\PYG{l+s+s1}{\PYGZsq{}} \PYG{p}{:} \PYG{l+s+s1}{\PYGZsq{}}\PYG{l+s+s1}{in}\PYG{l+s+s1}{\PYGZsq{}}\PYG{p}{,}
                     \PYG{l+s+s1}{\PYGZsq{}}\PYG{l+s+s1}{ytick.right}\PYG{l+s+s1}{\PYGZsq{}} \PYG{p}{:} \PYG{k+kc}{True}\PYG{p}{,}
                     \PYG{l+s+s1}{\PYGZsq{}}\PYG{l+s+s1}{ytick.direction}\PYG{l+s+s1}{\PYGZsq{}} \PYG{p}{:} \PYG{l+s+s1}{\PYGZsq{}}\PYG{l+s+s1}{in}\PYG{l+s+s1}{\PYGZsq{}}\PYG{p}{,}\PYG{p}{\PYGZcb{}}\PYG{p}{)}
\end{sphinxVerbatim}
}


\subsection{Prisms}
\label{\detokenize{notebooks/L3/Optical Elements:Prisms}}
Prisms are wedge shaped optical elements made of a transparent material as, for example, of glass. A special form of such a prism is an isosceles prism with two sides of equal length. The two equal sides enclose an angle \(\gamma\). When light is coupled into that prism light is refracted twice. First the incendent angle \(\alpha_1\) is changed into a refracted angle \(\beta_1\). This refracted ray then hits the second interface under an angle \(\beta_2\) leading again to a
refraction into the outgoing angle \(\alpha_2\). What is now interesting is the total deflection of the incident ray, which is measured by the angle \(\delta\).




\begin{savenotes}\sphinxattablestart
\centering
\begin{tabulary}{\linewidth}[t]{|T|}
\hline
\sphinxstyletheadfamily 
\sphinxincludegraphics[width=0.400\linewidth]{{prism}.png}
\\
\hline
\sphinxstylestrong{Fig.:} Refraction of rays on a prism.
\\
\hline
\end{tabulary}
\par
\sphinxattableend\end{savenotes}




\subsubsection{Deflection angle}
\label{\detokenize{notebooks/L3/Optical Elements:Deflection-angle}}
We can calculate the deflection angle \(\delta\) from a number of considerations. First consider the following relations between the angles in the prism and Snell’s law
\begin{equation*}
\begin{split}\beta_1=\sin^{-1}\left (\frac{n_0}{n_1}\sin(\alpha_1) \right)\end{split}
\end{equation*}\begin{equation*}
\begin{split}\beta_2=\gamma-\beta_1\end{split}
\end{equation*}\begin{equation*}
\begin{split}\alpha_2=\sin^{-1}\left (\frac{n_1}{n_0}\sin(\beta_2)\right )\end{split}
\end{equation*}\begin{equation*}
\begin{split}\theta_2=\alpha_2-\gamma\end{split}
\end{equation*}
where \(\theta_2\) is the angle between the incident surface normal and the outgoing ray. The total deflection angle \(\delta\) is then
\begin{equation*}
\begin{split}\delta =\alpha_1-\beta_1+\alpha_2-\beta_2\end{split}
\end{equation*}
or
\begin{equation*}
\begin{split}\delta =\alpha_1+\alpha_2-\gamma\end{split}
\end{equation*}
from which we obtain
\begin{equation*}
\begin{split}\delta=\alpha_1+\sin^{-1}\left (\frac{n_1}{n_0}\sin\left [\gamma-\sin^{-1}\left (\frac{n_0}{n_1}\sin(\alpha_1) \right)\right]\right )\end{split}
\end{equation*}
as the deflection angle.

{
\sphinxsetup{VerbatimColor={named}{nbsphinx-code-bg}}
\sphinxsetup{VerbatimBorderColor={named}{nbsphinx-code-border}}
\begin{sphinxVerbatim}[commandchars=\\\{\}]
\llap{\color{nbsphinxin}[2]:\,\hspace{\fboxrule}\hspace{\fboxsep}}\PYG{k}{def} \PYG{n+nf}{deflection}\PYG{p}{(}\PYG{n}{alpha\PYGZus{}1}\PYG{p}{,}\PYG{n}{gamma}\PYG{p}{,}\PYG{n}{n0}\PYG{p}{,}\PYG{n}{n1}\PYG{p}{)}\PYG{p}{:}
    \PYG{n}{g}\PYG{o}{=}\PYG{n}{gamma}\PYG{o}{*}\PYG{n}{np}\PYG{o}{.}\PYG{n}{pi}\PYG{o}{/}\PYG{l+m+mi}{180}
    \PYG{k}{return}\PYG{p}{(}\PYG{n}{alpha\PYGZus{}1}\PYG{o}{+}\PYG{n}{np}\PYG{o}{.}\PYG{n}{arcsin}\PYG{p}{(}\PYG{n}{n1}\PYG{o}{*}\PYG{n}{np}\PYG{o}{.}\PYG{n}{sin}\PYG{p}{(}\PYG{n}{g}\PYG{o}{\PYGZhy{}}\PYG{n}{np}\PYG{o}{.}\PYG{n}{arcsin}\PYG{p}{(}\PYG{n}{n0}\PYG{o}{*}\PYG{n}{np}\PYG{o}{.}\PYG{n}{sin}\PYG{p}{(}\PYG{n}{alpha\PYGZus{}1}\PYG{p}{)}\PYG{o}{/}\PYG{n}{n1}\PYG{p}{)}\PYG{p}{)}\PYG{p}{)}\PYG{p}{)}\PYG{o}{\PYGZhy{}}\PYG{n}{g}
\end{sphinxVerbatim}
}

{
\sphinxsetup{VerbatimColor={named}{nbsphinx-code-bg}}
\sphinxsetup{VerbatimBorderColor={named}{nbsphinx-code-border}}
\begin{sphinxVerbatim}[commandchars=\\\{\}]
\llap{\color{nbsphinxin}[3]:\,\hspace{\fboxrule}\hspace{\fboxsep}}\PYG{n}{a\PYGZus{}1}\PYG{o}{=}\PYG{n}{np}\PYG{o}{.}\PYG{n}{linspace}\PYG{p}{(}\PYG{l+m+mf}{0.1}\PYG{p}{,}\PYG{n}{np}\PYG{o}{.}\PYG{n}{pi}\PYG{o}{/}\PYG{l+m+mi}{2}\PYG{p}{,}\PYG{l+m+mi}{100}\PYG{p}{)}
\PYG{n}{plt}\PYG{o}{.}\PYG{n}{figure}\PYG{p}{(}\PYG{n}{figsize}\PYG{o}{=}\PYG{p}{(}\PYG{l+m+mi}{4}\PYG{p}{,}\PYG{l+m+mi}{5}\PYG{p}{)}\PYG{p}{)}
\PYG{n}{plt}\PYG{o}{.}\PYG{n}{plot}\PYG{p}{(}\PYG{n}{a\PYGZus{}1}\PYG{o}{*}\PYG{l+m+mi}{180}\PYG{o}{/}\PYG{n}{np}\PYG{o}{.}\PYG{n}{pi}\PYG{p}{,}\PYG{n}{deflection}\PYG{p}{(}\PYG{n}{a\PYGZus{}1}\PYG{p}{,}\PYG{l+m+mi}{45}\PYG{p}{,}\PYG{l+m+mi}{1}\PYG{p}{,}\PYG{l+m+mf}{1.5}\PYG{p}{)}\PYG{o}{*}\PYG{l+m+mi}{180}\PYG{o}{/}\PYG{n}{np}\PYG{o}{.}\PYG{n}{pi}\PYG{p}{,}\PYG{n}{label}\PYG{o}{=}\PYG{l+s+sa}{r}\PYG{l+s+s2}{\PYGZdq{}}\PYG{l+s+s2}{\PYGZdl{}}\PYG{l+s+s2}{\PYGZbs{}}\PYG{l+s+s2}{gamma=45\PYGZdl{} °}\PYG{l+s+s2}{\PYGZdq{}}\PYG{p}{)}
\PYG{n}{plt}\PYG{o}{.}\PYG{n}{plot}\PYG{p}{(}\PYG{n}{a\PYGZus{}1}\PYG{o}{*}\PYG{l+m+mi}{180}\PYG{o}{/}\PYG{n}{np}\PYG{o}{.}\PYG{n}{pi}\PYG{p}{,}\PYG{n}{deflection}\PYG{p}{(}\PYG{n}{a\PYGZus{}1}\PYG{p}{,}\PYG{l+m+mi}{30}\PYG{p}{,}\PYG{l+m+mi}{1}\PYG{p}{,}\PYG{l+m+mf}{1.5}\PYG{p}{)}\PYG{o}{*}\PYG{l+m+mi}{180}\PYG{o}{/}\PYG{n}{np}\PYG{o}{.}\PYG{n}{pi}\PYG{p}{,}\PYG{n}{label}\PYG{o}{=}\PYG{l+s+sa}{r}\PYG{l+s+s2}{\PYGZdq{}}\PYG{l+s+s2}{\PYGZdl{}}\PYG{l+s+s2}{\PYGZbs{}}\PYG{l+s+s2}{gamma=30\PYGZdl{} °}\PYG{l+s+s2}{\PYGZdq{}}\PYG{p}{)}
\PYG{n}{plt}\PYG{o}{.}\PYG{n}{plot}\PYG{p}{(}\PYG{n}{a\PYGZus{}1}\PYG{o}{*}\PYG{l+m+mi}{180}\PYG{o}{/}\PYG{n}{np}\PYG{o}{.}\PYG{n}{pi}\PYG{p}{,}\PYG{n}{deflection}\PYG{p}{(}\PYG{n}{a\PYGZus{}1}\PYG{p}{,}\PYG{l+m+mi}{10}\PYG{p}{,}\PYG{l+m+mi}{1}\PYG{p}{,}\PYG{l+m+mf}{1.5}\PYG{p}{)}\PYG{o}{*}\PYG{l+m+mi}{180}\PYG{o}{/}\PYG{n}{np}\PYG{o}{.}\PYG{n}{pi}\PYG{p}{,}\PYG{n}{label}\PYG{o}{=}\PYG{l+s+sa}{r}\PYG{l+s+s2}{\PYGZdq{}}\PYG{l+s+s2}{\PYGZdl{}}\PYG{l+s+s2}{\PYGZbs{}}\PYG{l+s+s2}{gamma=10\PYGZdl{} °}\PYG{l+s+s2}{\PYGZdq{}}\PYG{p}{)}
\PYG{n}{plt}\PYG{o}{.}\PYG{n}{xlabel}\PYG{p}{(}\PYG{l+s+sa}{r}\PYG{l+s+s2}{\PYGZdq{}}\PYG{l+s+s2}{incindence angle \PYGZdl{}}\PYG{l+s+s2}{\PYGZbs{}}\PYG{l+s+s2}{alpha\PYGZus{}1\PYGZdl{} [°]}\PYG{l+s+s2}{\PYGZdq{}}\PYG{p}{)}
\PYG{n}{plt}\PYG{o}{.}\PYG{n}{ylabel}\PYG{p}{(}\PYG{l+s+sa}{r}\PYG{l+s+s2}{\PYGZdq{}}\PYG{l+s+s2}{deflection angle \PYGZdl{}}\PYG{l+s+s2}{\PYGZbs{}}\PYG{l+s+s2}{delta\PYGZdl{} [°]}\PYG{l+s+s2}{\PYGZdq{}}\PYG{p}{)}
\PYG{n}{plt}\PYG{o}{.}\PYG{n}{legend}\PYG{p}{(}\PYG{p}{)}
\PYG{n}{plt}\PYG{o}{.}\PYG{n}{show}\PYG{p}{(}\PYG{p}{)}
\end{sphinxVerbatim}
}

\hrule height -\fboxrule\relax
\vspace{\nbsphinxcodecellspacing}

\makeatletter\setbox\nbsphinxpromptbox\box\voidb@x\makeatother

\begin{nbsphinxfancyoutput}

\noindent\sphinxincludegraphics[width=292\sphinxpxdimen,height=336\sphinxpxdimen]{{notebooks_L3_Optical_Elements_8_0}.png}

\end{nbsphinxfancyoutput}


\subsubsection{Minimum deflection angle}
\label{\detokenize{notebooks/L3/Optical Elements:Minimum-deflection-angle}}
If we now would like to know how the deflection angle changes with the incident angle \(\alpha_1\), we calculate the derivative of the deflection angle \(\delta\) with respect to \(\alpha_1\), i.e.,
\begin{equation*}
\begin{split}\frac{\mathrm d\delta}{\mathrm d\alpha_1}=1+\frac{\mathrm d\alpha_2}{\mathrm d \alpha_1}.\end{split}
\end{equation*}
We are here especially interested in the case, where this change in deflection is reaching a minimum, i.e., \(\mathrm d\delta/\mathrm d\alpha_1 =0\). This readily yields
\begin{equation*}
\begin{split}\mathrm d \alpha_2=-\mathrm d\alpha_1.\end{split}
\end{equation*}
This means a change in the incidence angle \(\mathrm d\alpha_1\) yields an opposite change in the outgoing angle \(-\mathrm d\alpha_2\). We may later observe that in the experiment.

As both, the incident and the outgoing angle are related to each other by Snells’s law, we may introduce the derivatives of Snell’s law for both interfaces, e.g.,
\begin{itemize}
\item {} 
\(\cos(\alpha_1)\mathrm d\alpha_1=n\cos(\beta_1)\mathrm d\beta_1\)

\item {} 
\(\cos(\alpha_2)\mathrm d\alpha_2=n\cos(\beta_2)\mathrm d\beta_2\)

\end{itemize}

where \(n\) is the refractive index of the prism material and the material outside is air (\(n_{\rm air}=1\)). Replacing \(\cos(\alpha)=\sqrt{1-\sin^2(\alpha)}\) and dividing the two previous equations by each other readily yields
\begin{equation*}
\begin{split}\frac{1-\sin^2(\alpha_1)}{1-\sin^2(\alpha_2)}=\frac{n^2-\sin^2(\alpha_1)}{n^2-\sin^2(\alpha_2)}.\end{split}
\end{equation*}
The latter equation is for \(n\neq 1\) only satisfied if \(\alpha_1=\alpha_2=\alpha\). In this case, the light path through the prism must be symmetric and we may write down the minimum deflection angle \(\delta_{\rm min}\):

\begin{sphinxadmonition}{note}{}\unskip
\sphinxstylestrong{Minimum prism deflection}

The minimum deflection angle of an isosceles prism with a prism angle \(\gamma\) is given by
\begin{equation*}
\begin{split}\delta_{\rm min}=2\alpha-\gamma.\end{split}
\end{equation*}\end{sphinxadmonition}

In this case, we may also write down Snell’s law using \(\sin(\alpha)=n\sin(\beta)\), which results in
\begin{equation*}
\begin{split}\sin \left ( \frac{\delta_{\rm min}+\gamma}{2}\right )=n\sin\left (\frac{\gamma}{2}\right).\end{split}
\end{equation*}

\subsubsection{Dispersion}
\label{\detokenize{notebooks/L3/Optical Elements:Dispersion}}
Very important applications now arise from the fact, that the refractive index is a material property, which depends on the color (frequency or wavelength) of light. We do not yet understand the origin of this dependence. The plots below show the wavelength dependence of three different glasses. You may find much more data on the refractive index of different materials in an \sphinxhref{https://refractiveindex.info/}{online database}.

{
\sphinxsetup{VerbatimColor={named}{nbsphinx-code-bg}}
\sphinxsetup{VerbatimBorderColor={named}{nbsphinx-code-border}}
\begin{sphinxVerbatim}[commandchars=\\\{\}]
\llap{\color{nbsphinxin}[4]:\,\hspace{\fboxrule}\hspace{\fboxsep}}\PYG{n}{bk7}\PYG{o}{=}\PYG{n}{pd}\PYG{o}{.}\PYG{n}{read\PYGZus{}csv}\PYG{p}{(}\PYG{l+s+s2}{\PYGZdq{}}\PYG{l+s+s2}{data/BK7.csv}\PYG{l+s+s2}{\PYGZdq{}}\PYG{p}{,}\PYG{n}{delimiter}\PYG{o}{=}\PYG{l+s+s2}{\PYGZdq{}}\PYG{l+s+s2}{,}\PYG{l+s+s2}{\PYGZdq{}}\PYG{p}{)}
\PYG{n}{sf10}\PYG{o}{=}\PYG{n}{pd}\PYG{o}{.}\PYG{n}{read\PYGZus{}csv}\PYG{p}{(}\PYG{l+s+s2}{\PYGZdq{}}\PYG{l+s+s2}{data/SF10.csv}\PYG{l+s+s2}{\PYGZdq{}}\PYG{p}{,}\PYG{n}{delimiter}\PYG{o}{=}\PYG{l+s+s2}{\PYGZdq{}}\PYG{l+s+s2}{,}\PYG{l+s+s2}{\PYGZdq{}}\PYG{p}{)}
\PYG{n}{fk51a}\PYG{o}{=}\PYG{n}{pd}\PYG{o}{.}\PYG{n}{read\PYGZus{}csv}\PYG{p}{(}\PYG{l+s+s2}{\PYGZdq{}}\PYG{l+s+s2}{data/FK51A.csv}\PYG{l+s+s2}{\PYGZdq{}}\PYG{p}{,}\PYG{n}{delimiter}\PYG{o}{=}\PYG{l+s+s2}{\PYGZdq{}}\PYG{l+s+s2}{,}\PYG{l+s+s2}{\PYGZdq{}}\PYG{p}{)}
\end{sphinxVerbatim}
}

{
\sphinxsetup{VerbatimColor={named}{nbsphinx-code-bg}}
\sphinxsetup{VerbatimBorderColor={named}{nbsphinx-code-border}}
\begin{sphinxVerbatim}[commandchars=\\\{\}]
\llap{\color{nbsphinxin}[5]:\,\hspace{\fboxrule}\hspace{\fboxsep}}\PYG{n}{plt}\PYG{o}{.}\PYG{n}{plot}\PYG{p}{(}\PYG{n}{bk7}\PYG{o}{.}\PYG{n}{wl}\PYG{o}{*}\PYG{l+m+mi}{1000}\PYG{p}{,}\PYG{n}{bk7}\PYG{o}{.}\PYG{n}{n}\PYG{p}{,}\PYG{n}{label}\PYG{o}{=}\PYG{l+s+s2}{\PYGZdq{}}\PYG{l+s+s2}{BK7}\PYG{l+s+s2}{\PYGZdq{}}\PYG{p}{)}
\PYG{n}{plt}\PYG{o}{.}\PYG{n}{plot}\PYG{p}{(}\PYG{n}{sf10}\PYG{o}{.}\PYG{n}{wl}\PYG{o}{*}\PYG{l+m+mi}{1000}\PYG{p}{,}\PYG{n}{sf10}\PYG{o}{.}\PYG{n}{n}\PYG{p}{,}\PYG{n}{label}\PYG{o}{=}\PYG{l+s+s2}{\PYGZdq{}}\PYG{l+s+s2}{SF10}\PYG{l+s+s2}{\PYGZdq{}}\PYG{p}{)}
\PYG{n}{plt}\PYG{o}{.}\PYG{n}{plot}\PYG{p}{(}\PYG{n}{fk51a}\PYG{o}{.}\PYG{n}{wl}\PYG{o}{*}\PYG{l+m+mi}{1000}\PYG{p}{,}\PYG{n}{fk51a}\PYG{o}{.}\PYG{n}{n}\PYG{p}{,}\PYG{n}{label}\PYG{o}{=}\PYG{l+s+s2}{\PYGZdq{}}\PYG{l+s+s2}{FK51A}\PYG{l+s+s2}{\PYGZdq{}}\PYG{p}{)}
\PYG{n}{plt}\PYG{o}{.}\PYG{n}{xlim}\PYG{p}{(}\PYG{l+m+mi}{300}\PYG{p}{,}\PYG{l+m+mi}{900}\PYG{p}{)}
\PYG{n}{plt}\PYG{o}{.}\PYG{n}{xlabel}\PYG{p}{(}\PYG{l+s+s2}{\PYGZdq{}}\PYG{l+s+s2}{wavelength [nm]}\PYG{l+s+s2}{\PYGZdq{}}\PYG{p}{)}
\PYG{n}{plt}\PYG{o}{.}\PYG{n}{ylabel}\PYG{p}{(}\PYG{l+s+s2}{\PYGZdq{}}\PYG{l+s+s2}{refractive index n}\PYG{l+s+s2}{\PYGZdq{}}\PYG{p}{)}
\PYG{n}{plt}\PYG{o}{.}\PYG{n}{legend}\PYG{p}{(}\PYG{p}{)}
\PYG{n}{plt}\PYG{o}{.}\PYG{n}{show}\PYG{p}{(}\PYG{p}{)}
\end{sphinxVerbatim}
}

\hrule height -\fboxrule\relax
\vspace{\nbsphinxcodecellspacing}

\makeatletter\setbox\nbsphinxpromptbox\box\voidb@x\makeatother

\begin{nbsphinxfancyoutput}

\noindent\sphinxincludegraphics[width=422\sphinxpxdimen,height=282\sphinxpxdimen]{{notebooks_L3_Optical_Elements_15_0}.png}

\end{nbsphinxfancyoutput}

{
\sphinxsetup{VerbatimColor={named}{nbsphinx-code-bg}}
\sphinxsetup{VerbatimBorderColor={named}{nbsphinx-code-border}}
\begin{sphinxVerbatim}[commandchars=\\\{\}]
\llap{\color{nbsphinxin}[6]:\,\hspace{\fboxrule}\hspace{\fboxsep}}\PYG{n}{bk7}\PYG{o}{=}\PYG{n}{pd}\PYG{o}{.}\PYG{n}{read\PYGZus{}csv}\PYG{p}{(}\PYG{l+s+s2}{\PYGZdq{}}\PYG{l+s+s2}{data/BK7.csv}\PYG{l+s+s2}{\PYGZdq{}}\PYG{p}{,}\PYG{n}{delimiter}\PYG{o}{=}\PYG{l+s+s2}{\PYGZdq{}}\PYG{l+s+s2}{,}\PYG{l+s+s2}{\PYGZdq{}}\PYG{p}{)}
\end{sphinxVerbatim}
}

{
\sphinxsetup{VerbatimColor={named}{nbsphinx-code-bg}}
\sphinxsetup{VerbatimBorderColor={named}{nbsphinx-code-border}}
\begin{sphinxVerbatim}[commandchars=\\\{\}]
\llap{\color{nbsphinxin}[86]:\,\hspace{\fboxrule}\hspace{\fboxsep}}\PYG{n}{a\PYGZus{}1}\PYG{o}{=}\PYG{n}{np}\PYG{o}{.}\PYG{n}{linspace}\PYG{p}{(}\PYG{l+m+mf}{0.15}\PYG{p}{,}\PYG{n}{np}\PYG{o}{.}\PYG{n}{pi}\PYG{o}{/}\PYG{l+m+mi}{2}\PYG{p}{,}\PYG{l+m+mi}{100}\PYG{p}{)}
\PYG{n}{plt}\PYG{o}{.}\PYG{n}{figure}\PYG{p}{(}\PYG{n}{figsize}\PYG{o}{=}\PYG{p}{(}\PYG{l+m+mi}{12}\PYG{p}{,}\PYG{l+m+mi}{4}\PYG{p}{)}\PYG{p}{)}
\PYG{n}{plt}\PYG{o}{.}\PYG{n}{subplot}\PYG{p}{(}\PYG{l+m+mi}{1}\PYG{p}{,}\PYG{l+m+mi}{2}\PYG{p}{,}\PYG{l+m+mi}{1}\PYG{p}{)}
\PYG{k}{for} \PYG{n}{wl} \PYG{o+ow}{in} \PYG{n}{np}\PYG{o}{.}\PYG{n}{linspace}\PYG{p}{(}\PYG{l+m+mf}{0.400}\PYG{p}{,}\PYG{l+m+mf}{0.700}\PYG{p}{,}\PYG{l+m+mi}{100}\PYG{p}{)}\PYG{p}{:}
    \PYG{n}{n1}\PYG{o}{=}\PYG{n}{np}\PYG{o}{.}\PYG{n}{interp}\PYG{p}{(}\PYG{n}{wl}\PYG{p}{,}\PYG{n}{bk7}\PYG{o}{.}\PYG{n}{wl}\PYG{p}{,}\PYG{n}{bk7}\PYG{o}{.}\PYG{n}{n}\PYG{p}{)}
    \PYG{n}{c}\PYG{o}{=}\PYG{n}{wavelength\PYGZus{}to\PYGZus{}rgb}\PYG{p}{(}\PYG{n}{wl}\PYG{o}{*}\PYG{l+m+mi}{1000}\PYG{p}{,} \PYG{n}{gamma}\PYG{o}{=}\PYG{l+m+mf}{0.8}\PYG{p}{)}
    \PYG{n}{plt}\PYG{o}{.}\PYG{n}{plot}\PYG{p}{(}\PYG{n}{a\PYGZus{}1}\PYG{o}{*}\PYG{l+m+mi}{180}\PYG{o}{/}\PYG{n}{np}\PYG{o}{.}\PYG{n}{pi}\PYG{p}{,}\PYG{n}{deflection}\PYG{p}{(}\PYG{n}{a\PYGZus{}1}\PYG{p}{,}\PYG{l+m+mi}{45}\PYG{p}{,}\PYG{l+m+mi}{1}\PYG{p}{,}\PYG{n}{n1}\PYG{p}{)}\PYG{o}{*}\PYG{l+m+mi}{180}\PYG{o}{/}\PYG{n}{np}\PYG{o}{.}\PYG{n}{pi}\PYG{p}{,}\PYG{n}{color}\PYG{o}{=}\PYG{n}{c}\PYG{p}{)}

\PYG{n}{plt}\PYG{o}{.}\PYG{n}{xlabel}\PYG{p}{(}\PYG{l+s+sa}{r}\PYG{l+s+s2}{\PYGZdq{}}\PYG{l+s+s2}{incindence angle \PYGZdl{}}\PYG{l+s+s2}{\PYGZbs{}}\PYG{l+s+s2}{alpha\PYGZus{}1\PYGZdl{} [°]}\PYG{l+s+s2}{\PYGZdq{}}\PYG{p}{)}
\PYG{n}{plt}\PYG{o}{.}\PYG{n}{ylabel}\PYG{p}{(}\PYG{l+s+sa}{r}\PYG{l+s+s2}{\PYGZdq{}}\PYG{l+s+s2}{deflection angle \PYGZdl{}}\PYG{l+s+s2}{\PYGZbs{}}\PYG{l+s+s2}{delta\PYGZdl{} [°]}\PYG{l+s+s2}{\PYGZdq{}}\PYG{p}{)}


\PYG{n}{plt}\PYG{o}{.}\PYG{n}{subplot}\PYG{p}{(}\PYG{l+m+mi}{1}\PYG{p}{,}\PYG{l+m+mi}{2}\PYG{p}{,}\PYG{l+m+mi}{2}\PYG{p}{)}
\PYG{k}{for} \PYG{n}{wl} \PYG{o+ow}{in} \PYG{n}{np}\PYG{o}{.}\PYG{n}{linspace}\PYG{p}{(}\PYG{l+m+mf}{0.400}\PYG{p}{,}\PYG{l+m+mf}{0.700}\PYG{p}{,}\PYG{l+m+mi}{100}\PYG{p}{)}\PYG{p}{:}
    \PYG{n}{n1}\PYG{o}{=}\PYG{n}{np}\PYG{o}{.}\PYG{n}{interp}\PYG{p}{(}\PYG{n}{wl}\PYG{p}{,}\PYG{n}{bk7}\PYG{o}{.}\PYG{n}{wl}\PYG{p}{,}\PYG{n}{bk7}\PYG{o}{.}\PYG{n}{n}\PYG{p}{)}
    \PYG{n}{c}\PYG{o}{=}\PYG{n}{wavelength\PYGZus{}to\PYGZus{}rgb}\PYG{p}{(}\PYG{n}{wl}\PYG{o}{*}\PYG{l+m+mi}{1000}\PYG{p}{,} \PYG{n}{gamma}\PYG{o}{=}\PYG{l+m+mf}{0.8}\PYG{p}{)}
    \PYG{n}{plt}\PYG{o}{.}\PYG{n}{plot}\PYG{p}{(}\PYG{n}{a\PYGZus{}1}\PYG{o}{*}\PYG{l+m+mi}{180}\PYG{o}{/}\PYG{n}{np}\PYG{o}{.}\PYG{n}{pi}\PYG{p}{,}\PYG{n}{deflection}\PYG{p}{(}\PYG{n}{a\PYGZus{}1}\PYG{p}{,}\PYG{l+m+mi}{45}\PYG{p}{,}\PYG{l+m+mi}{1}\PYG{p}{,}\PYG{n}{n1}\PYG{p}{)}\PYG{o}{*}\PYG{l+m+mi}{180}\PYG{o}{/}\PYG{n}{np}\PYG{o}{.}\PYG{n}{pi}\PYG{p}{,}\PYG{n}{color}\PYG{o}{=}\PYG{n}{c}\PYG{p}{)}

\PYG{n}{plt}\PYG{o}{.}\PYG{n}{xlabel}\PYG{p}{(}\PYG{l+s+sa}{r}\PYG{l+s+s2}{\PYGZdq{}}\PYG{l+s+s2}{incindence angle \PYGZdl{}}\PYG{l+s+s2}{\PYGZbs{}}\PYG{l+s+s2}{alpha\PYGZus{}1\PYGZdl{} [°]}\PYG{l+s+s2}{\PYGZdq{}}\PYG{p}{)}
\PYG{n}{plt}\PYG{o}{.}\PYG{n}{ylabel}\PYG{p}{(}\PYG{l+s+sa}{r}\PYG{l+s+s2}{\PYGZdq{}}\PYG{l+s+s2}{deflection angle \PYGZdl{}}\PYG{l+s+s2}{\PYGZbs{}}\PYG{l+s+s2}{delta\PYGZdl{} [°]}\PYG{l+s+s2}{\PYGZdq{}}\PYG{p}{)}
\PYG{n}{plt}\PYG{o}{.}\PYG{n}{xlim}\PYG{p}{(}\PYG{l+m+mi}{30}\PYG{p}{,}\PYG{l+m+mi}{45}\PYG{p}{)}
\PYG{n}{plt}\PYG{o}{.}\PYG{n}{ylim}\PYG{p}{(}\PYG{l+m+mi}{25}\PYG{p}{,}\PYG{l+m+mi}{30}\PYG{p}{)}
\PYG{n}{plt}\PYG{o}{.}\PYG{n}{show}\PYG{p}{(}\PYG{p}{)}
\end{sphinxVerbatim}
}

\hrule height -\fboxrule\relax
\vspace{\nbsphinxcodecellspacing}

\makeatletter\setbox\nbsphinxpromptbox\box\voidb@x\makeatother

\begin{nbsphinxfancyoutput}

\noindent\sphinxincludegraphics[width=757\sphinxpxdimen,height=288\sphinxpxdimen]{{notebooks_L3_Optical_Elements_17_0}.png}

\end{nbsphinxfancyoutput}

The plots have a general feature, which is that the refractive index is largest at small wavelength (blue colors), while it drops continuously with increasing wavelength towards the red (800 nm). If you would characterize the dependence by the slope, i.e., \(\mathrm dn/\mathrm d\lambda\) then all displayed curves show in the visible range
\begin{itemize}
\item {} 
\(\frac{\mathrm dn}{\mathrm d\lambda}<0\), is called normal dispersion

\end{itemize}

while
\begin{itemize}
\item {} 
\(\frac{\mathrm dn}{\mathrm d\lambda}<0\), is called anomalous dispersion

\end{itemize}

This wavelength dependence of the refractive index will yield a dependence of the deflection angle on the color of light as well. The change of the deflection angle with the refractive index can be calculated to be
\begin{equation*}
\begin{split}\frac{\mathrm d\delta}{\mathrm d n}=\frac{2\sin(\gamma/2)}{\sqrt{1-n^2\sin^2(\gamma/2)}}\end{split}
\end{equation*}
together with the relation
\begin{equation*}
\begin{split}\frac{\mathrm d \delta}{\mathrm d \lambda}=\frac{\mathrm d\delta}{\mathrm d n}\frac{\mathrm d n}{\mathrm d\lambda}\end{split}
\end{equation*}
we obtain
\begin{equation*}
\begin{split}\frac{\mathrm d\delta}{\mathrm d\lambda}=\frac{2\sin(\gamma/2)}{\sqrt{1-n^2\sin^2(\gamma/2)}}\frac{\mathrm d n}{\mathrm d \lambda}.\end{split}
\end{equation*}
The refraction of white light on a prism, therefore, splits the different colors composing white light spatially into a colored spectrum. Thereby the light with the longest wavelength is deflected the least, while the one with the highest refractive index is deflected most.




\begin{savenotes}\sphinxattablestart
\centering
\begin{tabulary}{\linewidth}[t]{|T|}
\hline
\sphinxstyletheadfamily 
\sphinxincludegraphics[width=0.490\linewidth]{{spectrum}.png}
\\
\hline
\sphinxstylestrong{Fig.:} Spectrum as created by a prism in the lecture.
\\
\hline
\end{tabulary}
\par
\sphinxattableend\end{savenotes}




\begin{savenotes}\sphinxattablestart
\centering
\begin{tabulary}{\linewidth}[t]{|T|}
\hline
\sphinxstyletheadfamily 
\sphinxincludegraphics[width=0.490\linewidth]{{spectrum}.jpeg} \sphinxincludegraphics[width=0.490\linewidth]{{prism1}.png}
\\
\hline
\sphinxstylestrong{Fig.:} Deflection of different wavelength of light in a prism with normal dispersion.
\\
\hline
\end{tabulary}
\par
\sphinxattableend\end{savenotes}

This dependence is very important as it enables spectroscopy, i.e., the recording of the intensity of light as a function of the wavelength.


\begin{savenotes}\sphinxattablestart
\centering
\begin{tabulary}{\linewidth}[t]{|T|}
\hline
\sphinxstyletheadfamily 
\sphinxincludegraphics[width=0.400\linewidth]{{prism_spec_principle}.jpeg} \sphinxincludegraphics[width=0.400\linewidth]{{prism_spectrometer}.jpeg}
\\
\hline
\sphinxstylestrong{Fig.:} Principle and technical realization of a prism spectrometer.
\\
\hline
\end{tabulary}
\par
\sphinxattableend\end{savenotes}

\sphinxstylestrong{DIY prism}

If you do not have a prism at home, which of course most people do, you can try to create your own, by a mirror, which you dip into a water basin. Shine some white light with the flash lamp on it and observe the reflected and refracted light especially at the edges. If you want it a bit better, integrate a small aperature into a piece of black paper and put it in fornt of the flash light.


\begin{savenotes}\sphinxattablestart
\centering
\begin{tabulary}{\linewidth}[t]{|T|}
\hline
\sphinxstyletheadfamily 
\sphinxincludegraphics[width=0.600\linewidth]{{diy_prism}.png}
\\
\hline
\sphinxstylestrong{Fig.:} Home made water prism.
\\
\hline
\end{tabulary}
\par
\sphinxattableend\end{savenotes}

The dependence of the refractive index of water on the wavelength is, however, weak. Yet, it is enough to also show the well known colors of the rainbow. We will later refer to the colors of the rainbow.

{
\sphinxsetup{VerbatimColor={named}{nbsphinx-code-bg}}
\sphinxsetup{VerbatimBorderColor={named}{nbsphinx-code-border}}
\begin{sphinxVerbatim}[commandchars=\\\{\}]
\llap{\color{nbsphinxin}[8]:\,\hspace{\fboxrule}\hspace{\fboxsep}}\PYG{n}{h2o}\PYG{o}{=}\PYG{n}{pd}\PYG{o}{.}\PYG{n}{read\PYGZus{}csv}\PYG{p}{(}\PYG{l+s+s2}{\PYGZdq{}}\PYG{l+s+s2}{data/H2O.csv}\PYG{l+s+s2}{\PYGZdq{}}\PYG{p}{,}\PYG{n}{delimiter}\PYG{o}{=}\PYG{l+s+s2}{\PYGZdq{}}\PYG{l+s+s2}{,}\PYG{l+s+s2}{\PYGZdq{}}\PYG{p}{)}
\end{sphinxVerbatim}
}

{
\sphinxsetup{VerbatimColor={named}{nbsphinx-code-bg}}
\sphinxsetup{VerbatimBorderColor={named}{nbsphinx-code-border}}
\begin{sphinxVerbatim}[commandchars=\\\{\}]
\llap{\color{nbsphinxin}[9]:\,\hspace{\fboxrule}\hspace{\fboxsep}}\PYG{n}{plt}\PYG{o}{.}\PYG{n}{plot}\PYG{p}{(}\PYG{n}{h2o}\PYG{o}{.}\PYG{n}{wl}\PYG{o}{*}\PYG{l+m+mi}{1000}\PYG{p}{,}\PYG{n}{h2o}\PYG{o}{.}\PYG{n}{n}\PYG{p}{,}\PYG{n}{label}\PYG{o}{=}\PYG{l+s+s2}{\PYGZdq{}}\PYG{l+s+s2}{H2O}\PYG{l+s+s2}{\PYGZdq{}}\PYG{p}{)}
\PYG{n}{plt}\PYG{o}{.}\PYG{n}{xlim}\PYG{p}{(}\PYG{l+m+mi}{300}\PYG{p}{,}\PYG{l+m+mi}{900}\PYG{p}{)}
\PYG{n}{plt}\PYG{o}{.}\PYG{n}{ylim}\PYG{p}{(}\PYG{l+m+mf}{1.3}\PYG{p}{,}\PYG{l+m+mf}{1.36}\PYG{p}{)}
\PYG{n}{plt}\PYG{o}{.}\PYG{n}{xlabel}\PYG{p}{(}\PYG{l+s+s2}{\PYGZdq{}}\PYG{l+s+s2}{wavelength [nm]}\PYG{l+s+s2}{\PYGZdq{}}\PYG{p}{)}
\PYG{n}{plt}\PYG{o}{.}\PYG{n}{ylabel}\PYG{p}{(}\PYG{l+s+s2}{\PYGZdq{}}\PYG{l+s+s2}{refractive index n}\PYG{l+s+s2}{\PYGZdq{}}\PYG{p}{)}
\PYG{n}{plt}\PYG{o}{.}\PYG{n}{legend}\PYG{p}{(}\PYG{p}{)}
\PYG{n}{plt}\PYG{o}{.}\PYG{n}{show}\PYG{p}{(}\PYG{p}{)}
\end{sphinxVerbatim}
}

\hrule height -\fboxrule\relax
\vspace{\nbsphinxcodecellspacing}

\makeatletter\setbox\nbsphinxpromptbox\box\voidb@x\makeatother

\begin{nbsphinxfancyoutput}

\noindent\sphinxincludegraphics[width=433\sphinxpxdimen,height=286\sphinxpxdimen]{{notebooks_L3_Optical_Elements_28_0}.png}

\end{nbsphinxfancyoutput}


\subsection{Lenses}
\label{\detokenize{notebooks/L3/Optical Elements:Lenses}}
The most important optical elements are lenses, which come in many different flavors. They consist of curved surfaces, which most commonly have the shape of a part of a spherical cap. It is, therefore, useful to have a look at the refraction at spherical surfaces.


\subsubsection{Refraction at spherical surfaces}
\label{\detokenize{notebooks/L3/Optical Elements:Refraction-at-spherical-surfaces}}
For our calculations of the refraction at spherical surfaces, we consider the sketch below.


\begin{savenotes}\sphinxattablestart
\centering
\begin{tabulary}{\linewidth}[t]{|T|}
\hline
\sphinxstyletheadfamily 
\sphinxincludegraphics[width=0.600\linewidth]{{curved_surface}.png}
\\
\hline
\sphinxstylestrong{Fig.:} Refraction at a curved surface.
\\
\hline
\end{tabulary}
\par
\sphinxattableend\end{savenotes}

What we would like to calculate is the distance \(b\) and the angle \(\theta_2\) at which a ray crosses the optical axis if it originated at a distance \(a\) under an angle \(\theta_1\) on the left side. This will help us to obtain an imaging equation for a lens. For the above geometry we may write down Snell’s law as
\begin{equation*}
\begin{split}n_{1}\sin(\alpha+\theta_1)=n_{2}\sin(\alpha+\theta_2).\end{split}
\end{equation*}
In addition, we may write down a number of definitions for the angular functions, which will turn out to be useful.
\begin{itemize}
\item {} 
\(\sin(\alpha)=\frac{y}{R}\)

\item {} 
\(\tan(\theta_1)=\frac{y}{a}\)

\item {} 
\(\tan(\theta_2)=\frac{y}{b}\)

\end{itemize}

To find now an imaging equation, we do an approximation, which is very common in optics. This is the so called \sphinxstylestrong{paraxial} approximation. It assumes that all angles involved in the calculation are small, such that we can resort to the first order approaximation of the angular functions. At the end, this will yield a linearization of these function. For small angles, we obtain
\begin{itemize}
\item {} 
\(\sin(\theta)\approx \theta\)

\item {} 
\(\cos(\theta)\approx 1\)

\item {} 
\(\tan(\theta)=\frac{\sin(\theta)}{\cos(\theta)}\approx \theta\)

\end{itemize}

With the help of these approximations we can write Snell’s law for the curved surface as
\begin{equation*}
\begin{split}n_1(\alpha+\theta_1)=n_2(\alpha-\theta_2).\end{split}
\end{equation*}
With some slight transformation which you will find in the video of the online lecture we obtain, therefore,
\begin{equation*}
\begin{split}\theta_2=\frac{n_2-n_1}{n_2 R}y -\frac{n_1}{n_2}\theta_1,\end{split}
\end{equation*}
which is a purely linear equation in \(y\) and \(\theta_1\).

Let us now assume light is coming form a point a distance \(y\) from the optical axis and there is a ray traveling parallel to the optical axis hitting the spherical surface at \(y\), while a second ray is incident for \(y=0\).


\begin{savenotes}\sphinxattablestart
\centering
\begin{tabulary}{\linewidth}[t]{|T|}
\hline
\sphinxstyletheadfamily 
\sphinxincludegraphics[width=0.600\linewidth]{{image_curved}.png}
\\
\hline
\sphinxstylestrong{Fig.:} Image formation at a curved surface.
\\
\hline
\end{tabulary}
\par
\sphinxattableend\end{savenotes}

We may apply our obatined formula for the two cases.

\(\theta_1=0\):
\begin{equation*}
\begin{split}\theta_2=\frac{n_2-n_1}{n_2}\frac{y}{R}\end{split}
\end{equation*}\begin{equation*}
\begin{split}\theta_2=\frac{y+\Delta y}{b}\end{split}
\end{equation*}
and thus
\begin{equation*}
\begin{split}\frac{y+\Delta y}{b}=\frac{n_2-n_1}{n_2}\frac{y}{R}\end{split}
\end{equation*}
\(y=0\):
\begin{equation*}
\begin{split}n_2\frac{\Delta y}{b}=n_1\frac{y}{a}\end{split}
\end{equation*}
Combining both equations yields
\begin{equation*}
\begin{split}\frac{n_1}{a}+\frac{n_2}{b}=\frac{n_2-n_1}{ R},\end{split}
\end{equation*}
where we define the new quantity \sphinxstylestrong{focal length} which only depends on the properties of the curved surface
\begin{equation*}
\begin{split}f=\frac{n_2}{n_2-n_1}R.\end{split}
\end{equation*}
\begin{sphinxadmonition}{note}{}\unskip
\sphinxstylestrong{Imaging Equation Spherical Refracting Surface}

The sum of the inverse object and image distances equals the inverse focal length of the spherical refracting surface
\begin{equation*}
\begin{split}\frac{n_1}{g}+\frac{n_2}{b}\approx\frac{n_2}{f},\end{split}
\end{equation*}
where the focal length of the refracting surface is given by
\begin{equation*}
\begin{split}f=\frac{n_2}{n_2-n_1}R\end{split}
\end{equation*}
in the paraxial approximation.
\end{sphinxadmonition}


\subsubsection{Refraction with two spherical surfaces}
\label{\detokenize{notebooks/L3/Optical Elements:Refraction-with-two-spherical-surfaces}}
In our previous calculation we have found a linear relation between the incident angle \(\theta_1\) with the optical axis, the incident height of the ray \(y\) and the outgoing angle \(\theta_2\):
\begin{equation*}
\begin{split}\theta_2=\frac{n_2-n_1}{n_2 R_1}y -\frac{n_1}{n_2}\theta_1.\end{split}
\end{equation*}
Remember that the refractive index \(n_1\) is the one of the medium from where we enter the lens with a refractive index \(n_2\). The radius of the first spherical surface is \(R_1\). For a lens as displayed below, we have to consider now a second refraction, where we take the outgoing angle of the first refraction as the incident angle and we have to reverse the use of the refractive indices. The radius of the second spherical surface is now \(R_2\) and its value will turn out
to be negative as the curvature with respect to the optical axis has reversed.


\begin{savenotes}\sphinxattablestart
\centering
\begin{tabulary}{\linewidth}[t]{|T|}
\hline
\sphinxstyletheadfamily 
\sphinxincludegraphics[width=0.400\linewidth]{{thin_lens}.png}
\\
\hline
\sphinxstylestrong{Fig.:} Refraction on two spherical surfaces.
\\
\hline
\end{tabulary}
\par
\sphinxattableend\end{savenotes}

When applying our equation a second time, we also need the second height \(y^{\prime}\) for the calculation. For a \sphinxstylestrong{thin lens}, we now have a thickness \(d<<R_1, R_2\) much smaller than the radii of the spherical surfaces. From that we may assume that \(y\approx y^{\prime}\), which means that the displacement in height of paraxial rays inside the lens can be neglected. If this is the case, we may just assume, that all refracting actions happening on a plane in the center of a lens,
which is drawn above as the dashed line. This plane is the so\sphinxhyphen{}called \sphinxstylestrong{principle plane} of the thin lens and all image construction can be reduced to the plane.

The result of the above calculation is leading to the imaging equation for the thin lens.

\begin{sphinxadmonition}{note}{}\unskip
\sphinxstylestrong{Imaging Equation Thin Lens}

The sum of the inverse object and image distances equals the inverse focal length of the thin lens
\begin{equation*}
\begin{split}\frac{1}{a}+\frac{1}{b}\approx\frac{n_2-n_1}{n_1}\left (\frac{1}{R_1}-\frac{1}{R_2}\right )=\frac{1}{f},\end{split}
\end{equation*}
where the focal length of the thin lens is given by
\begin{equation*}
\begin{split}f=\frac{n_1}{n_2-n_1}\left ( \frac{R_1 R_2}{R_2 -R_1}\right)\end{split}
\end{equation*}
in the paraxial approximation.
\end{sphinxadmonition}

The equation for the focal length has some important consequence. It says that if the difference of the refractive indices inside (\(n_2\)) and outside \(n_1\) get smaller, the focal length becomes larger and finally infinity. This can be nicely observed by placing a lens outside and inside a water filled basin as shown below.


\begin{savenotes}\sphinxattablestart
\centering
\begin{tabulary}{\linewidth}[t]{|T|}
\hline
\sphinxstyletheadfamily 
\sphinxincludegraphics[width=0.400\linewidth]{{lens_contrast_out}.png} \sphinxincludegraphics[width=0.400\linewidth]{{lens_contrast_in}.png}
\\
\hline
\sphinxstylestrong{Fig.:} Focusing of parallel rays by a lens in air (\(n_1=1\), left) and in water (\(n_1=1.36\), right). The images clearly show the change in focal length between the two situations.
\\
\hline
\end{tabulary}
\par
\sphinxattableend\end{savenotes}


\subsubsection{Image Construction}
\label{\detokenize{notebooks/L3/Optical Elements:Image-Construction}}
Images of objects can be now constructed if we refer to rays which do not emerge from a position on the optical axis only. In this case, we consider three different rays (two are actually enough). If we use as in the case of a concave mirror a central and a parallel ray, we will find a position where all rays cross on the other side. The conversion of the rays is exactly the same as in the case of a spherical mirror. The relation between the position of the object and the image along the optical
axis is described by the imaging equation.


\begin{savenotes}\sphinxattablestart
\centering
\begin{tabulary}{\linewidth}[t]{|T|}
\hline
\sphinxstyletheadfamily 
\sphinxincludegraphics[width=0.400\linewidth]{{thin_lens_imaging}.png}
\\
\hline
\sphinxstylestrong{Fig.:} Image construction on a thin lens.
\\
\hline
\end{tabulary}
\par
\sphinxattableend\end{savenotes}

Similar to the concave mirror, we may now also find out the image size or the magnification of the lens.

\begin{sphinxadmonition}{note}{}\unskip
\sphinxstylestrong{Magnification of a Lens}

The magnification is given by
\begin{equation*}
\begin{split}M=\frac{h_{\rm image}}{h_{\rm object}}=-\frac{b}{a}=\frac{f}{f-a}\end{split}
\end{equation*}
where the negative sign is the result of the reverse orientation of the real images created by a lens.
\end{sphinxadmonition}

According to our previous consideration \(M<0\) corresponds to a reversed image, while it is upright as the object for \(M>0\). We, therefore, easily see the following:
\begin{itemize}
\item {} 
for \(a<f\) the image is upright and magnified (image is virtual however)

\item {} 
for \(a>f\) the image is always reversed (image is real)

\item {} 
for \(2f>a>f\) the image is magnified

\item {} 
for \(a>2f\) the image is shrinked

\item {} 
for \(a=f\) the image appears at infinity with \(M=\infty\)

\end{itemize}

The image below shows the construction of images in 4 of the above cases for a bi\sphinxhyphen{}convex lens including the gerenation of a virtual image.


\begin{savenotes}\sphinxattablestart
\centering
\begin{tabulary}{\linewidth}[t]{|T|}
\hline
\sphinxstyletheadfamily 
\sphinxincludegraphics[width=0.800\linewidth]{{image_construction}.png}
\\
\hline
\sphinxstylestrong{Fig.:} Image construction on a biconvex lens with a parallel and a central ray for different object distances.
\\
\hline
\end{tabulary}
\par
\sphinxattableend\end{savenotes}


\subsubsection{Lens types}
\label{\detokenize{notebooks/L3/Optical Elements:Lens-types}}
Depending on the radii of curvature and their sign, one can now construct different types of lenses, that are used in many applications. Modern microscopy lenses, for example, contain sometimes up to 20 different lenses with


\begin{savenotes}\sphinxattablestart
\centering
\begin{tabulary}{\linewidth}[t]{|T|}
\hline
\sphinxstyletheadfamily 
\sphinxincludegraphics[width=0.400\linewidth]{{lens_types}.png}
\\
\hline
\sphinxstylestrong{Fig.:} Different lens types.
\\
\hline
\end{tabulary}
\par
\sphinxattableend\end{savenotes}


\begin{savenotes}\sphinxattablestart
\centering
\begin{tabulary}{\linewidth}[t]{|T|}
\hline
\sphinxstyletheadfamily 
\sphinxincludegraphics[width=0.320\linewidth]{{convex_plane_thick}.png} \sphinxincludegraphics[width=0.320\linewidth]{{convex_plane_thin}.png} \sphinxincludegraphics[width=0.320\linewidth]{{bi-concave_lens}.png}
\\
\hline
\sphinxstylestrong{Fig.:} Focusing behavior of a few different lens types.
\\
\hline
\end{tabulary}
\par
\sphinxattableend\end{savenotes}


\section{Lecture Contents}
\label{\detokenize{lectures/L4/overview_4:lecture-contents}}\label{\detokenize{lectures/L4/overview_4::doc}}
In Lecture 4 we will continue to discuss several optical elements, which help us to create optical instruments like spectrometers, microscopes or telescopes. The optical elements which are left, are thick lenses. Additionally we will talk about a rainbow and a dry version of it for you at home. In addition we will start a new section of optical instruments.

\noindent\sphinxincludegraphics[width=600\sphinxpxdimen]{{slides7}.png}

Lecture 4 slides for download \sphinxcode{\sphinxupquote{pdf}}

The following section was created from \sphinxcode{\sphinxupquote{source/notebooks/L4/Optical Elements.ipynb}}.


\section{Optical Elements Part III}
\label{\detokenize{notebooks/L4/Optical Elements:Optical-Elements-Part-III}}\label{\detokenize{notebooks/L4/Optical Elements::doc}}

\subsection{Thick lens}
\label{\detokenize{notebooks/L4/Optical Elements:Thick-lens}}
For a thin lens, the displacement of the beam in height (\(y,y^{\prime}\)) due to the thickness has been neglected. That means that we can reduce all refracting action of the lens to a single plane, which we call a principle plane. This approximation is (independent of the paraxial approximation) not anymore true for lenses if the displacement \(\Delta\) of the ray as in the image below cannot be neglected. Such lenses are called \sphinxstylestrong{thick lenses} and they do not have a single principle
plane anymore. In fact, the principle plane splits up into two principle planes at a distance \(h\).


\begin{savenotes}\sphinxattablestart
\centering
\begin{tabulary}{\linewidth}[t]{|T|}
\hline
\sphinxstyletheadfamily 
\sphinxincludegraphics[width=0.400\linewidth]{{thick_lens}.png}
\\
\hline
\sphinxstylestrong{Fig.:} Thick lens principle planes.
\\
\hline
\end{tabulary}
\par
\sphinxattableend\end{savenotes}

As indicated in the sketch above, an incident ray which is not deflected can be extended to its intersection with the optical axis at a point, which is a distance \(h_1\) behind the lens surface. This is the location for the first principle plane. The position of the second principle plane at a distance \(h_2\) before the back surface is found for by reversing the ray path. According to that, both principle planes have a distance \(h=d-h_1-h_2\). Using some mathematical effort, one
can show that the same imaging equation as for a thins lens can be used with a new definition of the focal length and taking into account that object and image distances refer to their principle planes.

\begin{sphinxadmonition}{note}{}\unskip
\sphinxstylestrong{Imaging Equation Thick Lens}

The sum of the inverse object and image distances to the principle planes (\(H_1,H_2\)) equals the inverse focal length of the thin lens
\begin{equation*}
\begin{split}\frac{1}{a}+\frac{1}{b}\approx\frac{1}{f},\end{split}
\end{equation*}
where the inverse focal length of the thick lens is given by
\begin{equation*}
\begin{split}\frac{1}{f}=\frac{n_2-n_1}{n_1}\left (\frac{1}{R_1}-\frac{1}{R_2}+\frac{(n_2-n_1)d}{n_2 R_1 R_2}\right )\end{split}
\end{equation*}
in the paraxial approximation.
\end{sphinxadmonition}

The location of the two principle planes are found to be
\begin{equation*}
\begin{split}h_{1}=-\frac{(n-1)f d}{n R_2}\end{split}
\end{equation*}\begin{equation*}
\begin{split}h_{2}=-\frac{(n-1)f d}{n R_1}\end{split}
\end{equation*}
As compared to the construction of an image on a thin lens, we now have to consider some pecularities for the thick lens. An incident parallel ray, which turns into a focal ray is now refracted at the second principle plane. The reverse must, therefore, be true for an incident focal ray. This ray is refracted on the first principle plane. The central ray is deflected on both principle planes. It is incident under a certain angle at the first principle plane and outgoing with the same principle
angle to the second principle plane. The sketch below summarizes these issues for a thick lens.


\begin{savenotes}\sphinxattablestart
\centering
\begin{tabulary}{\linewidth}[t]{|T|}
\hline
\sphinxstyletheadfamily 
\sphinxincludegraphics[width=0.500\linewidth]{{thick_lens_construction}.png}
\\
\hline
\sphinxstylestrong{Fig.:} Thick lens image construction.
\\
\hline
\end{tabulary}
\par
\sphinxattableend\end{savenotes}

The following section was created from \sphinxcode{\sphinxupquote{source/notebooks/L4/Rainbow.ipynb}}.


\section{Rainbow}
\label{\detokenize{notebooks/L4/Rainbow:Rainbow}}\label{\detokenize{notebooks/L4/Rainbow::doc}}
{
\sphinxsetup{VerbatimColor={named}{nbsphinx-code-bg}}
\sphinxsetup{VerbatimBorderColor={named}{nbsphinx-code-border}}
\begin{sphinxVerbatim}[commandchars=\\\{\}]
\llap{\color{nbsphinxin}[2]:\,\hspace{\fboxrule}\hspace{\fboxsep}}\PYG{c+c1}{\PYGZsh{}\PYGZsh{} just for plotting later}

\PYG{k+kn}{import} \PYG{n+nn}{pandas} \PYG{k}{as} \PYG{n+nn}{pd}
\PYG{k+kn}{import} \PYG{n+nn}{numpy} \PYG{k}{as} \PYG{n+nn}{np}
\PYG{k+kn}{import} \PYG{n+nn}{matplotlib}\PYG{n+nn}{.}\PYG{n+nn}{pyplot} \PYG{k}{as} \PYG{n+nn}{plt}
\PYG{k+kn}{from} \PYG{n+nn}{spectrumRGB} \PYG{k+kn}{import} \PYG{n}{wavelength\PYGZus{}to\PYGZus{}rgb}
\PYG{o}{\PYGZpc{}}\PYG{k}{matplotlib} inline
\PYG{o}{\PYGZpc{}}\PYG{k}{config} InlineBackend.figure\PYGZus{}format = \PYGZsq{}retina\PYGZsq{}

\PYG{n}{plt}\PYG{o}{.}\PYG{n}{rcParams}\PYG{o}{.}\PYG{n}{update}\PYG{p}{(}\PYG{p}{\PYGZob{}}\PYG{l+s+s1}{\PYGZsq{}}\PYG{l+s+s1}{font.size}\PYG{l+s+s1}{\PYGZsq{}}\PYG{p}{:} \PYG{l+m+mi}{12}\PYG{p}{,}
                     \PYG{l+s+s1}{\PYGZsq{}}\PYG{l+s+s1}{axes.titlesize}\PYG{l+s+s1}{\PYGZsq{}}\PYG{p}{:} \PYG{l+m+mi}{18}\PYG{p}{,}
                     \PYG{l+s+s1}{\PYGZsq{}}\PYG{l+s+s1}{axes.labelsize}\PYG{l+s+s1}{\PYGZsq{}}\PYG{p}{:} \PYG{l+m+mi}{16}\PYG{p}{,}
                     \PYG{l+s+s1}{\PYGZsq{}}\PYG{l+s+s1}{axes.labelpad}\PYG{l+s+s1}{\PYGZsq{}}\PYG{p}{:} \PYG{l+m+mi}{14}\PYG{p}{,}
                     \PYG{l+s+s1}{\PYGZsq{}}\PYG{l+s+s1}{lines.linewidth}\PYG{l+s+s1}{\PYGZsq{}}\PYG{p}{:} \PYG{l+m+mi}{1}\PYG{p}{,}
                     \PYG{l+s+s1}{\PYGZsq{}}\PYG{l+s+s1}{lines.markersize}\PYG{l+s+s1}{\PYGZsq{}}\PYG{p}{:} \PYG{l+m+mi}{10}\PYG{p}{,}
                     \PYG{l+s+s1}{\PYGZsq{}}\PYG{l+s+s1}{xtick.labelsize}\PYG{l+s+s1}{\PYGZsq{}} \PYG{p}{:} \PYG{l+m+mi}{16}\PYG{p}{,}
                     \PYG{l+s+s1}{\PYGZsq{}}\PYG{l+s+s1}{ytick.labelsize}\PYG{l+s+s1}{\PYGZsq{}} \PYG{p}{:} \PYG{l+m+mi}{16}\PYG{p}{,}
                     \PYG{l+s+s1}{\PYGZsq{}}\PYG{l+s+s1}{xtick.top}\PYG{l+s+s1}{\PYGZsq{}} \PYG{p}{:} \PYG{k+kc}{True}\PYG{p}{,}
                     \PYG{l+s+s1}{\PYGZsq{}}\PYG{l+s+s1}{xtick.direction}\PYG{l+s+s1}{\PYGZsq{}} \PYG{p}{:} \PYG{l+s+s1}{\PYGZsq{}}\PYG{l+s+s1}{in}\PYG{l+s+s1}{\PYGZsq{}}\PYG{p}{,}
                     \PYG{l+s+s1}{\PYGZsq{}}\PYG{l+s+s1}{ytick.right}\PYG{l+s+s1}{\PYGZsq{}} \PYG{p}{:} \PYG{k+kc}{True}\PYG{p}{,}
                     \PYG{l+s+s1}{\PYGZsq{}}\PYG{l+s+s1}{ytick.direction}\PYG{l+s+s1}{\PYGZsq{}} \PYG{p}{:} \PYG{l+s+s1}{\PYGZsq{}}\PYG{l+s+s1}{in}\PYG{l+s+s1}{\PYGZsq{}}\PYG{p}{,}\PYG{p}{\PYGZcb{}}\PYG{p}{)}
\end{sphinxVerbatim}
}


\subsection{Deflection with one reflection}
\label{\detokenize{notebooks/L4/Rainbow:Deflection-with-one-reflection}}
As the last topic of the optical elements we would like to have a look at a phenomenon, which has nothing to do with optical elements but is fun and just fits to the topic of dispersion. We will explore the rainbow and in addition a DIY version, the glassbow. To understand the rainbow we will have first a look at the reflection of rays from a single droplet.


\begin{savenotes}\sphinxattablestart
\centering
\begin{tabulary}{\linewidth}[t]{|T|}
\hline
\sphinxstyletheadfamily 
\sphinxincludegraphics[width=0.400\linewidth]{{rainbow}.png}
\\
\hline
\sphinxstylestrong{Fig.:} Reflection of rays in a single drop.
\\
\hline
\end{tabulary}
\par
\sphinxattableend\end{savenotes}

In the sketch above a light ray of white light is entering the droplet under an angle \(\alpha\) to the surface normal on the top. The ray is refracted and enters the droplet under an angle \(\beta\) to the surface normal. The angle can be calculated from Snell’s law
\begin{equation*}
\begin{split}n_{\rm air}\sin(\alpha)=n_{\rm water}\sin(\beta).\end{split}
\end{equation*}
Inside the droplet, the ray is now hitting the water/air surface at the backside from which it gets reflected. There, the incident angle is also \(\beta\) and the ray is reflected under an angle \(\beta\) as well. At that point, most of the light will, however, exit the drop on the backside, so that only a small fraction is reflected and traveling further to hit a second time the water/air surface at the angle \(\beta\). The light refracted out at that point leaves the droplet under
an angle \(\alpha\) with the surface normal due to the reversiblity of the light path. We are, however, interested in the angle \(\phi\) that the ray makes with the incident direction.

This angle \(\phi\) can be calculated from the above sketch to be
\begin{equation*}
\begin{split}\phi=4\beta-2\alpha.\end{split}
\end{equation*}
Since
\begin{equation*}
\begin{split}\beta=\sin^{-1}\left (\frac{n_{\rm air}}{n_{\rm water}}\sin(\alpha) \right)\end{split}
\end{equation*}
such that finally
\begin{equation*}
\begin{split}\phi=4\sin^{-1}\left (\frac{n_{\rm air}}{n_{\rm water}}\sin(\alpha) \right)-2\alpha\end{split}
\end{equation*}
So let us have a look at this dependence of the deflection angle as a function of the incidence angle.

{
\sphinxsetup{VerbatimColor={named}{nbsphinx-code-bg}}
\sphinxsetup{VerbatimBorderColor={named}{nbsphinx-code-border}}
\begin{sphinxVerbatim}[commandchars=\\\{\}]
\llap{\color{nbsphinxin}[3]:\,\hspace{\fboxrule}\hspace{\fboxsep}}\PYG{n}{alpha}\PYG{o}{=}\PYG{n}{np}\PYG{o}{.}\PYG{n}{linspace}\PYG{p}{(}\PYG{l+m+mi}{0}\PYG{p}{,}\PYG{n}{np}\PYG{o}{.}\PYG{n}{pi}\PYG{o}{/}\PYG{l+m+mi}{2}\PYG{p}{,}\PYG{l+m+mi}{10000}\PYG{p}{)}
\PYG{n}{h2o}\PYG{o}{=}\PYG{n}{pd}\PYG{o}{.}\PYG{n}{read\PYGZus{}csv}\PYG{p}{(}\PYG{l+s+s2}{\PYGZdq{}}\PYG{l+s+s2}{data/H2O.csv}\PYG{l+s+s2}{\PYGZdq{}}\PYG{p}{,}\PYG{n}{delimiter}\PYG{o}{=}\PYG{l+s+s2}{\PYGZdq{}}\PYG{l+s+s2}{,}\PYG{l+s+s2}{\PYGZdq{}}\PYG{p}{)}
\PYG{n}{n}\PYG{o}{=}\PYG{n}{np}\PYG{o}{.}\PYG{n}{interp}\PYG{p}{(}\PYG{l+m+mf}{0.500}\PYG{p}{,}\PYG{n}{h2o}\PYG{o}{.}\PYG{n}{wl}\PYG{p}{,}\PYG{n}{h2o}\PYG{o}{.}\PYG{n}{n}\PYG{p}{)}
\end{sphinxVerbatim}
}

{
\sphinxsetup{VerbatimColor={named}{nbsphinx-code-bg}}
\sphinxsetup{VerbatimBorderColor={named}{nbsphinx-code-border}}
\begin{sphinxVerbatim}[commandchars=\\\{\}]
\llap{\color{nbsphinxin}[4]:\,\hspace{\fboxrule}\hspace{\fboxsep}}\PYG{k}{def} \PYG{n+nf}{rainbow}\PYG{p}{(}\PYG{n}{alpha}\PYG{p}{,}\PYG{n}{n}\PYG{p}{)}\PYG{p}{:}
    \PYG{k}{return}\PYG{p}{(}\PYG{l+m+mi}{4}\PYG{o}{*}\PYG{n}{np}\PYG{o}{.}\PYG{n}{arcsin}\PYG{p}{(}\PYG{n}{np}\PYG{o}{.}\PYG{n}{sin}\PYG{p}{(}\PYG{n}{alpha}\PYG{p}{)}\PYG{o}{/}\PYG{n}{n}\PYG{p}{)}\PYG{o}{\PYGZhy{}}\PYG{l+m+mi}{2}\PYG{o}{*}\PYG{n}{alpha}\PYG{p}{)}
\end{sphinxVerbatim}
}

{
\sphinxsetup{VerbatimColor={named}{nbsphinx-code-bg}}
\sphinxsetup{VerbatimBorderColor={named}{nbsphinx-code-border}}
\begin{sphinxVerbatim}[commandchars=\\\{\}]
\llap{\color{nbsphinxin}[5]:\,\hspace{\fboxrule}\hspace{\fboxsep}}\PYG{n}{plt}\PYG{o}{.}\PYG{n}{plot}\PYG{p}{(}\PYG{n}{alpha}\PYG{o}{*}\PYG{l+m+mi}{180}\PYG{o}{/}\PYG{n}{np}\PYG{o}{.}\PYG{n}{pi}\PYG{p}{,}\PYG{n}{rainbow}\PYG{p}{(}\PYG{n}{alpha}\PYG{p}{,}\PYG{n}{n}\PYG{p}{)}\PYG{o}{*}\PYG{l+m+mi}{180}\PYG{o}{/}\PYG{n}{np}\PYG{o}{.}\PYG{n}{pi}\PYG{p}{)}
\PYG{n}{plt}\PYG{o}{.}\PYG{n}{xlabel}\PYG{p}{(}\PYG{l+s+sa}{r}\PYG{l+s+s2}{\PYGZdq{}}\PYG{l+s+s2}{incident angle \PYGZdl{}}\PYG{l+s+s2}{\PYGZbs{}}\PYG{l+s+s2}{alpha\PYGZdl{} [°]}\PYG{l+s+s2}{\PYGZdq{}}\PYG{p}{)}
\PYG{n}{plt}\PYG{o}{.}\PYG{n}{ylabel}\PYG{p}{(}\PYG{l+s+sa}{r}\PYG{l+s+s2}{\PYGZdq{}}\PYG{l+s+s2}{deflection angle \PYGZdl{}}\PYG{l+s+s2}{\PYGZbs{}}\PYG{l+s+s2}{phi\PYGZdl{} [°]}\PYG{l+s+s2}{\PYGZdq{}}\PYG{p}{)}
\PYG{n}{plt}\PYG{o}{.}\PYG{n}{show}\PYG{p}{(}\PYG{p}{)}
\end{sphinxVerbatim}
}

\hrule height -\fboxrule\relax
\vspace{\nbsphinxcodecellspacing}

\makeatletter\setbox\nbsphinxpromptbox\box\voidb@x\makeatother

\begin{nbsphinxfancyoutput}

\noindent\sphinxincludegraphics[width=404\sphinxpxdimen,height=282\sphinxpxdimen]{{notebooks_L4_Rainbow_10_0}.png}

\end{nbsphinxfancyoutput}

The dependence seems to show a maximum deflection angle at an incidence angle of around \(\alpha=60^{\circ}\). This is an important finding, as the whole appearance of the rainbow depends on that.

{
\sphinxsetup{VerbatimColor={named}{nbsphinx-code-bg}}
\sphinxsetup{VerbatimBorderColor={named}{nbsphinx-code-border}}
\begin{sphinxVerbatim}[commandchars=\\\{\}]
\llap{\color{nbsphinxin}[6]:\,\hspace{\fboxrule}\hspace{\fboxsep}}\PYG{n+nb}{print}\PYG{p}{(}\PYG{l+s+s1}{\PYGZsq{}}\PYG{l+s+s1}{Maximum deflection angle }\PYG{l+s+s1}{\PYGZsq{}}\PYG{p}{,} \PYG{n}{np}\PYG{o}{.}\PYG{n}{max}\PYG{p}{(}\PYG{n}{rainbow}\PYG{p}{(}\PYG{n}{alpha}\PYG{p}{,}\PYG{n}{n}\PYG{p}{)}\PYG{o}{*}\PYG{l+m+mi}{180}\PYG{o}{/}\PYG{n}{np}\PYG{o}{.}\PYG{n}{pi}\PYG{p}{)}\PYG{p}{)}
\end{sphinxVerbatim}
}

{

\kern-\sphinxverbatimsmallskipamount\kern-\baselineskip
\kern+\FrameHeightAdjust\kern-\fboxrule
\vspace{\nbsphinxcodecellspacing}

\sphinxsetup{VerbatimColor={named}{white}}
\sphinxsetup{VerbatimBorderColor={named}{nbsphinx-code-border}}
\begin{sphinxVerbatim}[commandchars=\\\{\}]
Maximum deflection angle  41.78815648670841
\end{sphinxVerbatim}
}


\subsection{Color of the rainbow}
\label{\detokenize{notebooks/L4/Rainbow:Color-of-the-rainbow}}
The color of the rainbow is now the result of the fact that the maximum deflection angle depends on the color of the light due to the dispersion. Since we have a refraction, reflection and another refraction, the largest maximum deflection angle is observed for red light, while the smallest one appears for blue light. The diagrams below show this result, which is in general true for materials with normal dispersion.

{
\sphinxsetup{VerbatimColor={named}{nbsphinx-code-bg}}
\sphinxsetup{VerbatimBorderColor={named}{nbsphinx-code-border}}
\begin{sphinxVerbatim}[commandchars=\\\{\}]
\llap{\color{nbsphinxin}[7]:\,\hspace{\fboxrule}\hspace{\fboxsep}}\PYG{n}{plt}\PYG{o}{.}\PYG{n}{figure}\PYG{p}{(}\PYG{n}{figsize}\PYG{o}{=}\PYG{p}{(}\PYG{l+m+mi}{12}\PYG{p}{,}\PYG{l+m+mi}{4}\PYG{p}{)}\PYG{p}{)}
\PYG{n}{plt}\PYG{o}{.}\PYG{n}{subplot}\PYG{p}{(}\PYG{l+m+mi}{1}\PYG{p}{,}\PYG{l+m+mi}{2}\PYG{p}{,}\PYG{l+m+mi}{1}\PYG{p}{)}
\PYG{k}{for} \PYG{n}{wl} \PYG{o+ow}{in} \PYG{n}{np}\PYG{o}{.}\PYG{n}{linspace}\PYG{p}{(}\PYG{l+m+mf}{0.400}\PYG{p}{,}\PYG{l+m+mf}{0.700}\PYG{p}{,}\PYG{l+m+mi}{100}\PYG{p}{)}\PYG{p}{:}
    \PYG{n}{c}\PYG{o}{=}\PYG{n}{wavelength\PYGZus{}to\PYGZus{}rgb}\PYG{p}{(}\PYG{n}{wl}\PYG{o}{*}\PYG{l+m+mi}{1000}\PYG{p}{,} \PYG{n}{gamma}\PYG{o}{=}\PYG{l+m+mf}{0.8}\PYG{p}{)}
    \PYG{n}{n}\PYG{o}{=}\PYG{n}{np}\PYG{o}{.}\PYG{n}{interp}\PYG{p}{(}\PYG{n}{wl}\PYG{p}{,}\PYG{n}{h2o}\PYG{o}{.}\PYG{n}{wl}\PYG{p}{,}\PYG{n}{h2o}\PYG{o}{.}\PYG{n}{n}\PYG{p}{)}
    \PYG{n}{plt}\PYG{o}{.}\PYG{n}{plot}\PYG{p}{(}\PYG{n}{alpha}\PYG{o}{*}\PYG{l+m+mi}{180}\PYG{o}{/}\PYG{n}{np}\PYG{o}{.}\PYG{n}{pi}\PYG{p}{,}\PYG{n}{rainbow}\PYG{p}{(}\PYG{n}{alpha}\PYG{p}{,}\PYG{n}{n}\PYG{p}{)}\PYG{o}{*}\PYG{l+m+mi}{180}\PYG{o}{/}\PYG{n}{np}\PYG{o}{.}\PYG{n}{pi}\PYG{p}{,}\PYG{n}{c}\PYG{o}{=}\PYG{n}{c}\PYG{p}{,}\PYG{n}{alpha}\PYG{o}{=}\PYG{l+m+mi}{1}\PYG{p}{,}\PYG{n}{lw}\PYG{o}{=}\PYG{l+m+mi}{1}\PYG{p}{)}
\PYG{n}{plt}\PYG{o}{.}\PYG{n}{xlabel}\PYG{p}{(}\PYG{l+s+sa}{r}\PYG{l+s+s2}{\PYGZdq{}}\PYG{l+s+s2}{incident angle \PYGZdl{}}\PYG{l+s+s2}{\PYGZbs{}}\PYG{l+s+s2}{alpha\PYGZdl{} [°]}\PYG{l+s+s2}{\PYGZdq{}}\PYG{p}{)}
\PYG{n}{plt}\PYG{o}{.}\PYG{n}{ylabel}\PYG{p}{(}\PYG{l+s+sa}{r}\PYG{l+s+s2}{\PYGZdq{}}\PYG{l+s+s2}{deflection angle \PYGZdl{}}\PYG{l+s+s2}{\PYGZbs{}}\PYG{l+s+s2}{phi\PYGZdl{} [°]}\PYG{l+s+s2}{\PYGZdq{}}\PYG{p}{)}



\PYG{n}{plt}\PYG{o}{.}\PYG{n}{subplot}\PYG{p}{(}\PYG{l+m+mi}{1}\PYG{p}{,}\PYG{l+m+mi}{2}\PYG{p}{,}\PYG{l+m+mi}{2}\PYG{p}{)}
\PYG{k}{for} \PYG{n}{wl} \PYG{o+ow}{in} \PYG{n}{np}\PYG{o}{.}\PYG{n}{linspace}\PYG{p}{(}\PYG{l+m+mf}{0.400}\PYG{p}{,}\PYG{l+m+mf}{0.700}\PYG{p}{,}\PYG{l+m+mi}{100}\PYG{p}{)}\PYG{p}{:}
    \PYG{n}{c}\PYG{o}{=}\PYG{n}{wavelength\PYGZus{}to\PYGZus{}rgb}\PYG{p}{(}\PYG{n}{wl}\PYG{o}{*}\PYG{l+m+mi}{1000}\PYG{p}{,} \PYG{n}{gamma}\PYG{o}{=}\PYG{l+m+mf}{0.8}\PYG{p}{)}
    \PYG{n}{n}\PYG{o}{=}\PYG{n}{np}\PYG{o}{.}\PYG{n}{interp}\PYG{p}{(}\PYG{n}{wl}\PYG{p}{,}\PYG{n}{h2o}\PYG{o}{.}\PYG{n}{wl}\PYG{p}{,}\PYG{n}{h2o}\PYG{o}{.}\PYG{n}{n}\PYG{p}{)}
    \PYG{n}{plt}\PYG{o}{.}\PYG{n}{plot}\PYG{p}{(}\PYG{n}{alpha}\PYG{o}{*}\PYG{l+m+mi}{180}\PYG{o}{/}\PYG{n}{np}\PYG{o}{.}\PYG{n}{pi}\PYG{p}{,}\PYG{n}{rainbow}\PYG{p}{(}\PYG{n}{alpha}\PYG{p}{,}\PYG{n}{n}\PYG{p}{)}\PYG{o}{*}\PYG{l+m+mi}{180}\PYG{o}{/}\PYG{n}{np}\PYG{o}{.}\PYG{n}{pi}\PYG{p}{,}\PYG{n}{c}\PYG{o}{=}\PYG{n}{c}\PYG{p}{,}\PYG{n}{alpha}\PYG{o}{=}\PYG{l+m+mi}{1}\PYG{p}{,}\PYG{n}{lw}\PYG{o}{=}\PYG{l+m+mi}{1}\PYG{p}{)}
\PYG{n}{plt}\PYG{o}{.}\PYG{n}{xlabel}\PYG{p}{(}\PYG{l+s+sa}{r}\PYG{l+s+s2}{\PYGZdq{}}\PYG{l+s+s2}{incident angle \PYGZdl{}}\PYG{l+s+s2}{\PYGZbs{}}\PYG{l+s+s2}{alpha\PYGZdl{} [°]}\PYG{l+s+s2}{\PYGZdq{}}\PYG{p}{)}
\PYG{n}{plt}\PYG{o}{.}\PYG{n}{ylabel}\PYG{p}{(}\PYG{l+s+sa}{r}\PYG{l+s+s2}{\PYGZdq{}}\PYG{l+s+s2}{deflection angle \PYGZdl{}}\PYG{l+s+s2}{\PYGZbs{}}\PYG{l+s+s2}{phi\PYGZdl{} [°]}\PYG{l+s+s2}{\PYGZdq{}}\PYG{p}{)}
\PYG{n}{plt}\PYG{o}{.}\PYG{n}{xlim}\PYG{p}{(}\PYG{l+m+mi}{45}\PYG{p}{,}\PYG{l+m+mi}{70}\PYG{p}{)}
\PYG{n}{plt}\PYG{o}{.}\PYG{n}{ylim}\PYG{p}{(}\PYG{l+m+mi}{40}\PYG{p}{,}\PYG{l+m+mi}{43}\PYG{p}{)}
\PYG{n}{plt}\PYG{o}{.}\PYG{n}{tight\PYGZus{}layout}\PYG{p}{(}\PYG{p}{)}
\PYG{n}{plt}\PYG{o}{.}\PYG{n}{show}\PYG{p}{(}\PYG{p}{)}
\end{sphinxVerbatim}
}

\hrule height -\fboxrule\relax
\vspace{\nbsphinxcodecellspacing}

\makeatletter\setbox\nbsphinxpromptbox\box\voidb@x\makeatother

\begin{nbsphinxfancyoutput}

\noindent\sphinxincludegraphics[width=850\sphinxpxdimen,height=274\sphinxpxdimen]{{notebooks_L4_Rainbow_15_0}.png}

\end{nbsphinxfancyoutput}

This order of the colors is actually true for all incident angles, which raises the question, why the rainbow is actually colored. The blue color of a certain incidence angle would actually overlap with the green color of a different incidence angle and the red color of an even different incidence angle. If you select a specific outgoing angle under which you observe the rainbow, let us say 41°, then you find under this observation angle all color and, therefore, should observe always white
light.

This is actually true if you look at the inside of a rainbow. You clearly recognize that inside the rainbow it is much brighter than outside. Yet when you reach the maximum angle of each color, you have a region, where even for larger angles for the incidence angles, the deflection angle does not change. Thus, if you assume you send in rays at constantly spaced incidence angles, you will have more rays with a deflection angle close to the maximum. The diagram below just counts the number of
deflection angles in the different for each color and you clearly see that around the maximum deflection angles for each color you have a strong peak.

{
\sphinxsetup{VerbatimColor={named}{nbsphinx-code-bg}}
\sphinxsetup{VerbatimBorderColor={named}{nbsphinx-code-border}}
\begin{sphinxVerbatim}[commandchars=\\\{\}]
\llap{\color{nbsphinxin}[8]:\,\hspace{\fboxrule}\hspace{\fboxsep}}\PYG{n}{plt}\PYG{o}{.}\PYG{n}{figure}\PYG{p}{(}\PYG{n}{figsize}\PYG{o}{=}\PYG{p}{(}\PYG{l+m+mi}{5}\PYG{p}{,}\PYG{l+m+mi}{6}\PYG{p}{)}\PYG{p}{)}
\PYG{k}{for} \PYG{n}{wl} \PYG{o+ow}{in} \PYG{n}{np}\PYG{o}{.}\PYG{n}{linspace}\PYG{p}{(}\PYG{l+m+mf}{0.400}\PYG{p}{,}\PYG{l+m+mf}{0.700}\PYG{p}{,}\PYG{l+m+mi}{4}\PYG{p}{)}\PYG{p}{:}
    \PYG{n}{c}\PYG{o}{=}\PYG{n}{wavelength\PYGZus{}to\PYGZus{}rgb}\PYG{p}{(}\PYG{n}{wl}\PYG{o}{*}\PYG{l+m+mi}{1000}\PYG{p}{,} \PYG{n}{gamma}\PYG{o}{=}\PYG{l+m+mf}{0.8}\PYG{p}{)}
    \PYG{n}{n}\PYG{o}{=}\PYG{n}{np}\PYG{o}{.}\PYG{n}{interp}\PYG{p}{(}\PYG{n}{wl}\PYG{p}{,}\PYG{n}{h2o}\PYG{o}{.}\PYG{n}{wl}\PYG{p}{,}\PYG{n}{h2o}\PYG{o}{.}\PYG{n}{n}\PYG{p}{)}
    \PYG{n}{plt}\PYG{o}{.}\PYG{n}{hist}\PYG{p}{(}\PYG{n}{rainbow}\PYG{p}{(}\PYG{n}{alpha}\PYG{p}{,}\PYG{n}{n}\PYG{p}{)}\PYG{o}{*}\PYG{l+m+mi}{180}\PYG{o}{/}\PYG{n}{np}\PYG{o}{.}\PYG{n}{pi}\PYG{p}{,}\PYG{n}{bins}\PYG{o}{=}\PYG{l+m+mi}{200}\PYG{p}{,}\PYG{n}{color}\PYG{o}{=}\PYG{n}{c}\PYG{p}{,}\PYG{n}{alpha}\PYG{o}{=}\PYG{l+m+mf}{0.6}\PYG{p}{,}\PYG{n}{lw}\PYG{o}{=}\PYG{l+m+mi}{1}\PYG{p}{)}\PYG{p}{;}
\PYG{n}{plt}\PYG{o}{.}\PYG{n}{xlabel}\PYG{p}{(}\PYG{l+s+sa}{r}\PYG{l+s+s2}{\PYGZdq{}}\PYG{l+s+s2}{deflection angle \PYGZdl{}}\PYG{l+s+s2}{\PYGZbs{}}\PYG{l+s+s2}{alpha\PYGZdl{} [°]}\PYG{l+s+s2}{\PYGZdq{}}\PYG{p}{)}
\PYG{n}{plt}\PYG{o}{.}\PYG{n}{ylabel}\PYG{p}{(}\PYG{l+s+sa}{r}\PYG{l+s+s2}{\PYGZdq{}}\PYG{l+s+s2}{intensity [a.u.]}\PYG{l+s+s2}{\PYGZdq{}}\PYG{p}{)}

\PYG{n}{plt}\PYG{o}{.}\PYG{n}{xlim}\PYG{p}{(}\PYG{l+m+mi}{30}\PYG{p}{,}\PYG{l+m+mi}{45}\PYG{p}{)}
\PYG{n}{plt}\PYG{o}{.}\PYG{n}{show}\PYG{p}{(}\PYG{p}{)}
\end{sphinxVerbatim}
}

\hrule height -\fboxrule\relax
\vspace{\nbsphinxcodecellspacing}

\makeatletter\setbox\nbsphinxpromptbox\box\voidb@x\makeatother

\begin{nbsphinxfancyoutput}

\noindent\sphinxincludegraphics[width=367\sphinxpxdimen,height=391\sphinxpxdimen]{{notebooks_L4_Rainbow_18_0}.png}

\end{nbsphinxfancyoutput}

Thus, around each droplet on the sky, there is a cone of deflected light reflected back from the sun. On the outside of that cone is red light under an angle of almost \(42^{\circ}\) while on the inside edge we find blue light and finally white light (see left image below). We just have to connect that to the observer now. This is shown in the right sketch. The rainbow, therefore, results from the fact that we look at different height at different edges of the cone.


\begin{savenotes}\sphinxattablestart
\centering
\begin{tabulary}{\linewidth}[t]{|T|}
\hline
\sphinxstyletheadfamily 
\sphinxincludegraphics[width=0.400\linewidth]{{cone}.png} \sphinxincludegraphics[width=0.400\linewidth]{{observation}.png}
\\
\hline
\sphinxstylestrong{Fig.:} Deflection cones of different color on a single drop in a rainbow and rainbow as a result of the observation of the cones.
\\
\hline
\end{tabulary}
\par
\sphinxattableend\end{savenotes}


\subsubsection{Rainbow}
\label{\detokenize{notebooks/L4/Rainbow:id1}}
The photos below show a rainbow, where you should see the enhanced white color in the center of the rainbow, the color order of the bow itself and the darker region outside. But there is also a second rainbow appearing on the outside.


\begin{savenotes}\sphinxattablestart
\centering
\begin{tabulary}{\linewidth}[t]{|T|}
\hline
\sphinxstyletheadfamily 
\sphinxincludegraphics[width=1.000\linewidth]{{rainbow_full}.JPG}
\\
\hline
\sphinxstylestrong{Fig.:} Double rainbow over the Grand Canyon. (c) Picture by Frank Cichos
\\
\hline
\end{tabulary}
\par
\sphinxattableend\end{savenotes}

This second outside rainbow has a reversed color order and is weaker than the inside rainbow. This secondary rainbow comes from a process involving two refractions and two reflections inside each drop.


\begin{savenotes}\sphinxattablestart
\centering
\begin{tabulary}{\linewidth}[t]{|T|}
\hline
\sphinxstyletheadfamily 
\sphinxincludegraphics[width=1.000\linewidth]{{rainbow_gczoom}.JPG}
\\
\hline
\sphinxstylestrong{Fig.:} Zoom into the rainbow over the Grand Canyon. (c) Picture by Frank Cichos
\\
\hline
\end{tabulary}
\par
\sphinxattableend\end{savenotes}


\subsubsection{Glassbow}
\label{\detokenize{notebooks/L4/Rainbow:Glassbow}}
You can observe a beautiful rainbow at home if you just have a few glass beads. You can get these glass beeds from us. They have a diameter of about 200 µm. If you place them on a black background and use a flash lamp, you will observe a nice rainbow as Mr. Märcker did in the photo below. The main difference is the observation angle. Try to calculate the new observation angle for the refractive indices of glass.


\begin{savenotes}\sphinxattablestart
\centering
\begin{tabulary}{\linewidth}[t]{|T|}
\hline
\sphinxstyletheadfamily 
\sphinxincludegraphics[width=1.000\linewidth]{{glassbow}.JPG}
\\
\hline
\sphinxstylestrong{Fig.:} Reflection of rays in a single drop. (c) Picture by Axel Märcker.
\\
\hline
\end{tabulary}
\par
\sphinxattableend\end{savenotes}


\section{Lecture Contents}
\label{\detokenize{lectures/L5/overview_5:lecture-contents}}\label{\detokenize{lectures/L5/overview_5::doc}}
In Lecture 5 we will discuss all the experiments we have missed in Lecture 4. In addition we will start a new section of optical instruments, which discusses systems of lenses and the imaging with the eye.

\noindent\sphinxincludegraphics[width=600\sphinxpxdimen]{{slides8}.png}

Lecture 5 slides for download \sphinxcode{\sphinxupquote{pdf}}

The following section was created from \sphinxcode{\sphinxupquote{source/notebooks/L5/Lens Systems and Optical Instruments.ipynb}}.


\section{Lens Systems and Optical Instruments}
\label{\detokenize{notebooks/L5/Lens Systems and Optical Instruments:Lens-Systems-and-Optical-Instruments}}\label{\detokenize{notebooks/L5/Lens Systems and Optical Instruments::doc}}

\subsection{Lens Systems}
\label{\detokenize{notebooks/L5/Lens Systems and Optical Instruments:Lens-Systems}}
Most of the optical instruments consist of multiple lenses that are used to image objects or to magnify them. They are combined at distances, which are either larger or smaller than the sum of their focal distances. The image below shows, for example, an artifical image of a lens system contained in a single microscope objctive lens, where you see multiple elements, e.g., doublets and triplets of lenses. These elements as joined objects may considerably improve the performance of the optics,
e.g., correct for imaging errors.


\begin{savenotes}\sphinxattablestart
\centering
\begin{tabulary}{\linewidth}[t]{|T|}
\hline
\sphinxstyletheadfamily 
\sphinxincludegraphics[width=0.600\linewidth]{{objective_lens1}.jpg}
\\
\hline
\sphinxstylestrong{Fig.:} System of of lenses inside a microscope objective lens.
\\
\hline
\end{tabulary}
\par
\sphinxattableend\end{savenotes}

We will consider first a pair of two biconvex lenses that are at a distance \(D>f_1+f_2\), where \(f_1\) and \(f_2\) are their focal distances.


\begin{savenotes}\sphinxattablestart
\centering
\begin{tabulary}{\linewidth}[t]{|T|}
\hline
\sphinxstyletheadfamily 
\sphinxincludegraphics[width=0.600\linewidth]{{lens_system1}.png}
\\
\hline
\sphinxstylestrong{Fig.:} System of two biconvex lenses at a distance larger than the sum of their focal distances.
\\
\hline
\end{tabulary}
\par
\sphinxattableend\end{savenotes}

Here the first lens creates an image at a position
\begin{equation*}
\begin{split}b_1=\frac{a_1 f_1}{(a_1-f_1)}.\end{split}
\end{equation*}
Accordingly the object distance for lens 2 is then \(a_2=D-b_1\).

The second lens then images this intermediate object into
\begin{equation*}
\begin{split}b_2=\frac{a_2 f_2}{a_2-f_2}=\frac{(D-b_1) f_2}{D-b_1-f_2},\end{split}
\end{equation*}
from which we can finally calculate the image location and also the size with the help of \(b_1\).

Instead of doing a lengthy transformation I would like to draw your attention to a simple method which arises from the linearization of Snell’s law. This is called \sphinxstylestrong{matrix optics}. Matrix optics relates the outgoing angle \(\theta_2\) and height \(y_2\) of an optical element to its incident angle \(\theta_1\) and \(y_1\) via a matrix operation. Let us have a look at an example of a biconvex lens with a focal length \(f\). An incident ray is then expressed by a vector
\begin{equation*}
\begin{split}\begin{bmatrix}
y_1\\
\theta_1
\end{bmatrix},\end{split}
\end{equation*}
which is converted by the lens into an outgoing ray vector
\begin{equation*}
\begin{split}\begin{bmatrix}
y_2\\
\theta_2
\end{bmatrix}.\end{split}
\end{equation*}
If you have two 2D vectors which you want to transform by a linear operation into each other, then the transformation can be described by a \(2 \times 2\) matrix, which is given for a lens with
\begin{equation*}
\begin{split}\begin{bmatrix}
A & B\\
C & D
\end{bmatrix}
=
\begin{bmatrix}
1 & 0\\
-\frac{1}{f} & 1
\end{bmatrix}.\end{split}
\end{equation*}
Similarly we can also define a free space propagation matrix for a propagation by a distance \(D\), which is
\begin{equation*}
\begin{split}\begin{bmatrix}
A & B\\
C & D
\end{bmatrix}
=
\begin{bmatrix}
1 & D\\
0 & 1
\end{bmatrix}.\end{split}
\end{equation*}
The propagation of light through two lenses, which are seprarated by a distance \(D\) is then the product of two matrices for the lenses and one for the free space. The transformation then in addition requires to multiply from the right side the incident vector and we obtain on the left side the outgoing ray vector.
\begin{equation*}
\begin{split}\begin{bmatrix}
y_2\\
\theta_2\\
\end{bmatrix}
=
\begin{bmatrix}
1 & 0\\
-\frac{1}{f_2} & 1 \\
\end{bmatrix}
\begin{bmatrix}
1 & D\\
0 & 1
\end{bmatrix}
\begin{bmatrix}
1 & 0\\
-\frac{1}{f_1} & 1
\end{bmatrix}
\begin{bmatrix}
y_1\\
\theta_1
\end{bmatrix}.\end{split}
\end{equation*}
If you multiply the \(2 \times 2\) matrices with each other, you will obtain a new \(2 \times 2\) matrix with a matrix element \(C\), which should correspond to the effective focal length of that system. This is the case since the element \(C=-\frac{1}{f}\).

The result of that calculation is
\begin{equation*}
\begin{split}\frac{1}{f}=\frac{1}{f_1}+\frac{1}{f_2}-\frac{D}{f_1 f_2},\end{split}
\end{equation*}
which gives the effective focal length of two biconvex lenses at a distance \(D\).

Following this equation the inverse of the total focal length of the combined lenses is just the sum of its inverse focal distances minus a term, which depends on the distance of the two lenses. If this distance is small as compared to the focal length, i.e., the lenses are close to each other, the total inverse focal is just given by the first two terms. The inverse focal distances characterizes the refractive power of a lens. The larger the inverse value, the smaller is the focal distance.
This refractive power is commonly measured in the unit \sphinxstylestrong{diopter}. One diopter corresponds to \sphinxstylestrong{1 dpt=1 :math:\textasciigrave{}m\textasciicircum{}\{\sphinxhyphen{}1\}\textasciigrave{}}.

The image above shows also that in the case of the combined lenses and a real inverted intermediate image, the final image will be upright again. Thus, if the first lens has a magnification \(M_1=-b_1/a_1\) and the second lens a magnification \(M_2=-b_2/a_2\), the total magnification is the product, which is
\begin{equation*}
\begin{split}M=\frac{b_1}{a_1}\frac{b_2}{a_2},\end{split}
\end{equation*}
from which we may finally obtain with \(M=\frac{f}{f-a}\)
\begin{equation*}
\begin{split}M=\frac{1}{1-\frac{a_1}{f_1}-\frac{a_1+D}{f_2}+\frac{a_1D}{f_1 f_2}}.\end{split}
\end{equation*}
For more than two lenses, there is a versatile framework called \sphinxstylestrong{Matrix Optics}, which treats each optical element as a \(2 \times 2\) matrix. This becomes possible as we derived earlier equations which were linear in \(y\) and \(\theta\). A whole system of different lenses, plates and other optical elements can, thus, be treated as a matrix multiplication, which is quite useful.

The following section was created from \sphinxcode{\sphinxupquote{source/notebooks/L5/Optical Instruments.ipynb}}.


\section{Optical Instruments}
\label{\detokenize{notebooks/L5/Optical Instruments:Optical-Instruments}}\label{\detokenize{notebooks/L5/Optical Instruments::doc}}
Optical instruments now combine a number of optical elements or even consist only out of a single one as in the case of the magnifying glass or the eye.


\subsection{The Eye}
\label{\detokenize{notebooks/L5/Optical Instruments:The-Eye}}
The human eye is certainly one of the most amazing biological sensorial systems, which consists of biologically grown optical elements including an aperture, an adaptive lens and a photodetector, which is connected to an intelligent neural network, that is amazingly accurate and fast in pattern recognition. Think of that for a moment, that this performance is given with a few 10 W power consumption and then turn around your laptop and check out its power consumption.

The eye allows you to image objects at variable distance from you. The sketch below indicates some of its essential ingredients.


\begin{savenotes}\sphinxattablestart
\centering
\begin{tabulary}{\linewidth}[t]{|T|}
\hline
\sphinxstyletheadfamily 
\sphinxincludegraphics[width=0.200\linewidth]{{eye}.jpeg} \sphinxincludegraphics[width=0.200\linewidth]{{retina}.png}
\\
\hline
\sphinxstylestrong{Fig.:} Sketch of the ingredients of the human eye. Most important parts the lens, the glass body and the retina containing the bightness (rods) and color sensors (cones). A more detailed sketch is depicted on the right side, which shows the arrangement of rods and cones in the retina including their parallel connection. Their combined signal is ending the optical nerves guiding the signals to the brain.
\\
\hline
\end{tabulary}
\par
\sphinxattableend\end{savenotes}

Most important is the pupil resembling an adjustable aperture for adjusting the total light falling to the retina and the lens, connected to muscles to change the curvature of the lens and, thus, also the focal distance.

The lens is connected to air on one side as the surrounding medium and to a vitreous humor (glass body) on the other side. So there is some asymmetry. The eye, therefore, has a front focal length of \(f_1=17\,{\rm mm}\) and a back focal length of \(f_2=22\,{\rm mm}\). These values may vary by the individual but also by the distance of the object. For close objects the eye can accomodate its focal distance to \(f_1=14\,{\rm mm}\) and \(f_2=19\,{\rm mm}\).


\begin{savenotes}\sphinxattablestart
\centering
\begin{tabulary}{\linewidth}[t]{|T|}
\hline
\sphinxstyletheadfamily 
\sphinxincludegraphics[width=0.400\linewidth]{{eye_focal_distance}.png}
\\
\hline
\sphinxstylestrong{Fig.:} Focal distances for the eye lens.
\\
\hline
\end{tabulary}
\par
\sphinxattableend\end{savenotes}

Cones and rods are differently distributed around the fovea, which is the spot of the highest color sensitivity. The cone density peaks around this center, while the rod density is larger outside. Cones are the color sensitive objects, which gain their light sensitive from a dye molecules called \sphinxstyleemphasis{retinal} undergoing a conformational transition upon excitation with light. This triggers a cascasde of different chemical processes. Cones with three narrow excitation ranges allow our color vision
(see Fig. below).


\begin{savenotes}\sphinxattablestart
\centering
\begin{tabulary}{\linewidth}[t]{|T|}
\hline
\sphinxstyletheadfamily 
\sphinxincludegraphics[width=0.600\linewidth]{{cone_and_rods}.png}
\\
\hline
\sphinxstylestrong{Fig.:} Angular distribution of cones and rods around the fovea, electron micrograph of cones (large) and rods (small) as well as their color sensitivity.
\\
\hline
\end{tabulary}
\par
\sphinxattableend\end{savenotes}

Under normal conditions, objects at infinite distance are imaged to the retina at the back focal distance \(f_2=22\, {\rm mm}\). However, the lens of the eye may have defects changing its refractive power. Often this leads to short\sphinxhyphen{}sightedness or far\sphinxhyphen{}sightedness. If you are short\sphinxhyphen{}sighted, the light from infinite distance is not focused to the retina but to some shorter distance inside the glass body. This may be corrected by a concave lens as displayed in the image below on the left side. If
you are far\sphinxhyphen{}sighted, then the light is focused behind the retina if objects are ininitely distant away. To correct for that error, one needs a convex lens as displayed on the right side.


\begin{savenotes}\sphinxattablestart
\centering
\begin{tabulary}{\linewidth}[t]{|T|}
\hline
\sphinxstyletheadfamily 
\sphinxincludegraphics[width=0.250\linewidth]{{short_sighted}.png} \sphinxincludegraphics[width=0.300\linewidth]{{far_sighted}.png}
\\
\hline
\sphinxstylestrong{Fig.:} Short\sphinxhyphen{}sightedness (left) and far\sphinxhyphen{}sightedness (right) and their correction by additional lenses.
\\
\hline
\end{tabulary}
\par
\sphinxattableend\end{savenotes}

In the normal relaxed state of the eye muscle, one can, therefore, observe objects up to a distance of \(s_0=25\, {\rm cm}\) without additional accomodation of the eye lens. This distance, which varies from person to person, is standardized and called the \sphinxstylestrong{range of clear visual sight}. This range of clear visual sight allows to observe objects under an angle \(\epsilon\).


\begin{savenotes}\sphinxattablestart
\centering
\begin{tabulary}{\linewidth}[t]{|T|}
\hline
\sphinxstyletheadfamily 
\sphinxincludegraphics[width=0.400\linewidth]{{relaxed_eye}.png}
\\
\hline
\sphinxstylestrong{Fig.:} Short\sphinxhyphen{}sightedness (left) and far\sphinxhyphen{}sightedness (right) and their correction by additional lenses.
\\
\hline
\end{tabulary}
\par
\sphinxattableend\end{savenotes}

Instead of calculating the magnification of the following optical instruments from the object and image distances, we refer now to a different measure, which is the \sphinxstylestrong{angular magnification}. The angular magnification is given by
\begin{equation*}
\begin{split}V=\frac{\tan(\epsilon)}{\tan(\epsilon_0)}\approx \frac{\epsilon}{\epsilon_0}.\end{split}
\end{equation*}
This is a useful measure as the image size is not really accessible ;\sphinxhyphen{}).


\section{Lecture Contents}
\label{\detokenize{lectures/L6/overview_6:lecture-contents}}\label{\detokenize{lectures/L6/overview_6::doc}}
In Lecture 6 we address the magnification of optical instruments such as the magnifying glass, the microscope or telescope. If we have time, we will also start to explore  imaging errors and possibilities to correct them.

\noindent\sphinxincludegraphics[width=600\sphinxpxdimen]{{slides9}.png}

Lecture 6 slides for download \sphinxcode{\sphinxupquote{pdf}}

The following section was created from \sphinxcode{\sphinxupquote{source/notebooks/L6/Magnifying Glass.ipynb}}.


\section{Magnifying Glass}
\label{\detokenize{notebooks/L6/Magnifying Glass:Magnifying-Glass}}\label{\detokenize{notebooks/L6/Magnifying Glass::doc}}
A magnifying glass has several applications. First of all, it allows to see objects with details that would otherwise be too small to be observed with the eye even of the eye lens can accomodate to the distances. Such magnifying glasses are also used in microscopes as the so\sphinxhyphen{}called \sphinxstylestrong{eye\sphinxhyphen{}piece} as we will later see in the section on microscopes.

Consider the sketch below. The sketch shows an object of a size \(A\) which is at a distance of \(s_0\) from the eye. The object makes an angle \(\epsilon_0\) with the optical axis. If we insert now a lens into the space between object and eye and the lens is positioned in a way that it is exactly at a distance \(f\) (the focal distance of the lens) from the object then we are able to observe the object under a different angle \(\epsilon\).


\begin{savenotes}\sphinxattablestart
\centering
\begin{tabulary}{\linewidth}[t]{|T|}
\hline
\sphinxstyletheadfamily 
\sphinxincludegraphics[width=0.600\linewidth]{{magnifying_glass_focal}.png}
\\
\hline
\sphinxstylestrong{Fig.:} Magnifying glass at the focal distance.
\\
\hline
\end{tabulary}
\par
\sphinxattableend\end{savenotes}

The magnification of this magnifying glass can the be calculated from the angles \(\epsilon\approx A/f\) and \(\epsilon_0\approx A/s_0\):
\begin{equation*}
\begin{split}V=\frac{\tan(\epsilon)}{\tan(\epsilon_0)}\approx \frac{\epsilon}{\epsilon_0}=\frac{A}{f}\frac{s_0}{A}=\frac{s_0}{f}.\end{split}
\end{equation*}
The angular magnification is, thus, just given by the ratio of the clear visual range to the focal distance of the lens. If the focal distance \(f\) becomes much smaller than \(s_0\), large magnifications are possible.

A second very useful effect is that when the object is placed inside the focal distance from the lens, the eye images a virtual image at infinite distance to the retina (see sketch). This means the eye muscle can stay relaxed when observing the object, while it would otherwise probably have to accommodate to the distance.

Yet, placing the object at exactly the focal distance is rather tedious when holding the magnifying glass by hand. If the object is now placed inside the focal distance of the magnifying glass, we may also calculate a magnification in this case knowing the virtual image size \(B\) created in this case (see sketch below)


\begin{savenotes}\sphinxattablestart
\centering
\begin{tabulary}{\linewidth}[t]{|T|}
\hline
\sphinxstyletheadfamily 
\sphinxincludegraphics[width=0.600\linewidth]{{inside_focus}.png}
\\
\hline
\sphinxstylestrong{Fig.:} Magnifying glass for an object inside the focal range of the lens.
\\
\hline
\end{tabulary}
\par
\sphinxattableend\end{savenotes}

If \(a\) is the distance of the object from the principle plane of the magnifying glass and \(b\) and \(B\) are the distance and the size of the virtual image, respectively, we obtain
\begin{equation*}
\begin{split}V=\frac{\tan(\epsilon)}{\tan(\epsilon_0)}\approx \frac{\epsilon}{\epsilon_0}=\frac{B}{b}\frac{s_0}{A}=\frac{s_0}{a}.\end{split}
\end{equation*}
Using the imaging equation
\begin{equation*}
\begin{split}\frac{1}{f}=\frac{1}{a}+\frac{1}{b}\end{split}
\end{equation*}
we may finally arrive at
\begin{equation*}
\begin{split}V=\frac{s_0(b-f)}{b\,f}\end{split}
\end{equation*}
in this case. If we place the virtual image directly at the clear visual range, i.e., \(b=-s_0\), we find
\begin{equation*}
\begin{split}V=\frac{s_0}{f}+1.\end{split}
\end{equation*}
The following section was created from \sphinxcode{\sphinxupquote{source/notebooks/L6/Microscope.ipynb}}.


\section{Microscope}
\label{\detokenize{notebooks/L6/Microscope:Microscope}}\label{\detokenize{notebooks/L6/Microscope::doc}}
The simplest form of a microscope consists of an objective lens with a focal distance \(f_1\) and a magnifying glass called eye\sphinxhyphen{}piece with a focal length \(f_2\). In this system of two lenses (which are itself systems of lenses in modern microscopes, see below),


\begin{savenotes}\sphinxattablestart
\centering
\begin{tabulary}{\linewidth}[t]{|T|}
\hline
\sphinxstyletheadfamily 
\sphinxincludegraphics[width=0.600\linewidth]{{microscopy_lenses}.png}
\\
\hline
\sphinxstylestrong{Fig.:} Cut through a microscope objective lens (left) and an eye\sphinxhyphen{}piece.
\\
\hline
\end{tabulary}
\par
\sphinxattableend\end{savenotes}

the object is placed at a distance \(f_1< a_1<2f_1\) from the objective lens creating a real and reversed image at a distance \(b_1\) behind the lens. This reversed image is observed by the eye through the eye\sphinxhyphen{}piece. The image of the objective lens is thereby adjusted to appear at the focal distance of the eye\sphinxhyphen{}piece.


\begin{savenotes}\sphinxattablestart
\centering
\begin{tabulary}{\linewidth}[t]{|T|}
\hline
\sphinxstyletheadfamily 
\sphinxincludegraphics[width=0.600\linewidth]{{simple_microscope}.png}
\\
\hline
\sphinxstylestrong{Fig.:} Sketch of a simple microscope. The strange object on the right is an eye.
\\
\hline
\end{tabulary}
\par
\sphinxattableend\end{savenotes}

For this simple microscope system we may calculate first the intermediate image position \(b_1\):
\begin{equation*}
\begin{split}\frac{1}{f_1}=\frac{1}{a_1}+\frac{1}{b_1}\end{split}
\end{equation*}
resulting in
\begin{equation*}
\begin{split}b_1=\frac{a_1 f_1}{a_1-f_1}.\end{split}
\end{equation*}
If we assume a \(\delta\) to be the distance of the object from the focal point of the objective lens, we even find for \(\delta \rightarrow 0\)
\begin{equation*}
\begin{split}b_1=\frac{a_1 f_1}{\delta}.\end{split}
\end{equation*}
The intermediate image of size \(B_1\) is now imaged by a magnifying glass of focal distance \(f_2\). According to what we calculated earlier, we have now the observation angle
\begin{equation*}
\begin{split}\tan(\epsilon)=\frac{B_1}{f_2}=\frac{Ab}{a_1 f_2}.\end{split}
\end{equation*}
If we observe the object of a size \(A\) and the clear visual distance \(s_0\), it would cover an angle of
\begin{equation*}
\begin{split}\tan(\epsilon_0)=\frac{A}{s_0}\end{split}
\end{equation*}
and we may obtain the total angular magnification
\begin{equation*}
\begin{split}V=\frac{A b_1 s_0}{A b_1 f_2}=\frac{b_1 s_0}{g f_2}.\end{split}
\end{equation*}
If we set the distance between the two lenses to \(D=b_1+f_2\) and \(g\approx f_1\) then we obtain
\begin{equation*}
\begin{split}V=\frac{(D-f_2)s_0}{f_1 f_2}\end{split}
\end{equation*}
which says that the magnification is the result of the two focal length \(f_1,f_2\).


\subsection{Modern microscopy}
\label{\detokenize{notebooks/L6/Microscope:Modern-microscopy}}
While the above description is correct for the simplest microscope you can think of, modern microscopes have much more complex light paths and work in general with optics, that is so\sphinxhyphen{}called \sphinxstylestrong{inifinity corrected}. Infinity corrected optics consists of an objective lens, which images the objects in the focal plane to infinity. Such an objctive lens always comes with a second lens, the \sphinxstylestrong{tube lens}, which together are designed to give a magnification specified at the objective lens housing.


\begin{savenotes}\sphinxattablestart
\centering
\begin{tabulary}{\linewidth}[t]{|T|}
\hline
\sphinxstyletheadfamily 
\sphinxincludegraphics[width=0.400\linewidth]{{infinity_optics}.png}
\\
\hline
\sphinxstylestrong{Fig.:} Infinity optics vs. normal microscopy optics.
\\
\hline
\end{tabulary}
\par
\sphinxattableend\end{savenotes}

Infinity optics allows you to have a free length with a parallel optical path where you can insert optical elements. There is no fixed tube length as in the case sketched above, where the distance of the intermediate image has to be considered. Therefore, it has tremendous technical advantages. Common optical microscopes are further today coupled to CCD cameras to record images digitally. Yet, an eye\sphinxhyphen{}piece may still be available in many cases. The sketch below shows the light path for a simple
fluorescence microscope recording fluorescence images with a camera.


\begin{savenotes}\sphinxattablestart
\centering
\begin{tabulary}{\linewidth}[t]{|T|}
\hline
\sphinxstyletheadfamily 
\sphinxincludegraphics[width=0.400\linewidth]{{fluorescence_micr}.png}
\\
\hline
\sphinxstylestrong{Fig.:} Simple fluorescence microscope.
\\
\hline
\end{tabulary}
\par
\sphinxattableend\end{savenotes}

The possibility to digitally record images creates endless possibilities to computationally enhance and combine images. Nowadays the field of optics is one of the fastest developing fields in physics with numerous new techniques appearing every week. In this field of imaging methods of machine learning also play an increasingly important role. While I’m not ablields in physics with numerous new techniques appearing every week. In this field of imaging methods of machine learning also play an
increasingly important role. While I am not able to refer to all possible optical microscopy techniques he to refer to all possible optical microscopy techniques here, I will exemplarily show some data from the Waller group at Berkley using computational methods to enhance the resolution by keeping at the same time a large field of view for imaging. This technique is called \sphinxstylestrong{ptychography} and can be understood if you consider Fourier Optics (a field of optics describing ligh propagation in
terms Fourier transforms).


\begin{savenotes}\sphinxattablestart
\centering
\begin{tabulary}{\linewidth}[t]{|T|}
\hline
\sphinxstyletheadfamily 
\sphinxincludegraphics[width=0.800\linewidth]{{ptychography}.png}
\\
\hline
\sphinxstylestrong{Fig.:} Ptychographic imaging with LED arrays.
\\
\hline
\end{tabulary}
\par
\sphinxattableend\end{savenotes}

There is a massive amount of other techniques with increadible images being generated. Have a look around.

The following section was created from \sphinxcode{\sphinxupquote{source/notebooks/L6/Telescope.ipynb}}.


\section{Telescope}
\label{\detokenize{notebooks/L6/Telescope:Telescope}}\label{\detokenize{notebooks/L6/Telescope::doc}}
Other than the microscope, the telescope is made to observe distant objects, which would appear under a very small observation angle. In the same way as a microscope, the telescope consists of two lenses with the focal distances \(f_1,f_2\).


\begin{savenotes}\sphinxattablestart
\centering
\begin{tabulary}{\linewidth}[t]{|T|}
\hline
\sphinxstyletheadfamily 
\sphinxincludegraphics[width=0.600\linewidth]{{telescope}.png}
\\
\hline
\sphinxstylestrong{Fig.:} Kepler telescope with two biconvex lenses, creating a reversed image of distance objects.
\\
\hline
\end{tabulary}
\par
\sphinxattableend\end{savenotes}

As indicated in the sketch above, the first lens generates an image at the focal length of the first lens. This intermediate image is the magnified by an eye\sphinxhyphen{}piece as well acting as a magnifying glass. We may therefore apply the same kind of techniques as earlier for the calculation of the angular magnification. The angle of observation for the object of size \(D\) is given by
\begin{equation*}
\begin{split}2\epsilon_0=\frac{D}{f_1}\end{split}
\end{equation*}
while the angle of observation through the telescope is given as
\begin{equation*}
\begin{split}2\epsilon=\frac{D}{f_2}\end{split}
\end{equation*}
Correspondingly, the angular magnification is given by
\begin{equation*}
\begin{split}V=\frac{\epsilon}{\epsilon_0}=\frac{D}{f_2}\frac{f_1}{D}=\frac{f_1}{f_2}\end{split}
\end{equation*}
The magnification is therefore given by the ration of the focal length of the entrance lens and the eye\sphinxhyphen{}piece. The above telescope is also termed \sphinxstylestrong{astronomical telescope} or \sphinxstylestrong{Kepler telescope}, since it has been used for astromical observations. It creates an image which is reversed.

A telescope with an upright image may be created with the help of a concave lens. This type of telescope is called Galilei telescope and obeyes the same magnification formula as above. Due to the fact that a concave lens has a negative focal length, the total magnification will be negative as well being indicative for an upright image.


\begin{savenotes}\sphinxattablestart
\centering
\begin{tabulary}{\linewidth}[t]{|T|}
\hline
\sphinxstyletheadfamily 
\sphinxincludegraphics[width=0.400\linewidth]{{galilei}.png}
\\
\hline
\sphinxstylestrong{Fig.:} Galilei telescope for imaging objects into upright images with the help of a concave and a convex lens.
\\
\hline
\end{tabulary}
\par
\sphinxattableend\end{savenotes}

Modern powerful telescopes also use mirrors instead of refracting optical elements, as reflecting elements with nearly 100 percent reflectivity can be built with a much smaller mass than large glass elements. Such telescopes come in different setups. The one below is a Cassegrain telescope, where a secondary convex miror is used for imaging the intermediate image to the eye.


\begin{savenotes}\sphinxattablestart
\centering
\begin{tabulary}{\linewidth}[t]{|T|}
\hline
\sphinxstyletheadfamily 
\sphinxincludegraphics[width=0.200\linewidth]{{real_telescope}.png} \sphinxincludegraphics[width=0.400\linewidth]{{cassegrain}.png}
\\
\hline
\sphinxstylestrong{Fig.:} Reflective optics is commonly used in modern high quality telescopes for the advantage of weight. The image and sketch shows the optics of a so\sphinxhyphen{}called Cassegrain telescope.
\\
\hline
\end{tabulary}
\par
\sphinxattableend\end{savenotes}

The most famous Cassegrain telescope is probably the Hubble space telescope as you might recognize from the sketch below.


\begin{savenotes}\sphinxattablestart
\centering
\begin{tabulary}{\linewidth}[t]{|T|}
\hline
\sphinxstyletheadfamily 
\sphinxincludegraphics[width=0.600\linewidth]{{hubble}.png}
\\
\hline
\sphinxstylestrong{Fig.:} Light path in the Hubble space telescope.
\\
\hline
\end{tabulary}
\par
\sphinxattableend\end{savenotes}


\section{Lecture Contents}
\label{\detokenize{lectures/L7/overview_7:lecture-contents}}\label{\detokenize{lectures/L7/overview_7::doc}}
In Lecture 7 we discuss the different types of imaging errors, which all arise if the paraxial approximation is violated.

\noindent\sphinxincludegraphics[width=600\sphinxpxdimen]{{slides10}.png}

Lecture 7 slides for download \sphinxcode{\sphinxupquote{pdf}}

The following section was created from \sphinxcode{\sphinxupquote{source/notebooks/L7/Imaging Errors.ipynb}}.


\section{Imaging Errors}
\label{\detokenize{notebooks/L7/Imaging Errors:Imaging-Errors}}\label{\detokenize{notebooks/L7/Imaging Errors::doc}}
During our derivation of the imaging equation for lenses and the lens\sphinxhyphen{}maker equation we have been working under the paraxial approximation. This approximation stated, that all rays are close to the optical axis and therefore make only small angles with the surface normals of the curved surfaces of lenses (but also mirrors). If we violate this approximation, i.e. if we use rays, which are incident for from the optical axis or strongly inclined, then we end up with reflections and refraction which
do not obey the imaging equation. In addition we have seen that light propagation for different colors is subject to different refractive indices (remember the prism). Thus we will induce aberrations, related to color.

According to Seidel, aberration are classified the following way
\begin{itemize}
\item {} 
chromatic aberration

\item {} 
spherical aberration

\item {} 
coma

\item {} 
astigmatism

\item {} 
field curvature

\item {} 
field distortion

\end{itemize}


\subsection{Chromatic Aberrations}
\label{\detokenize{notebooks/L7/Imaging Errors:Chromatic-Aberrations}}
Chromatic Aberration are based on the fact that light of different color has a different speed of propgation and thus also a different refractive index. We experienced that also for the prism, where it was useful to create a spectrograph. Here it is causing colored edges in you image, which you do not want.

As the refractive index for shorter wavelength is typically higher, we expect that the blue color has a shorter focal distance than the red color.


\begin{savenotes}\sphinxattablestart
\centering
\begin{tabulary}{\linewidth}[t]{|T|}
\hline
\sphinxstyletheadfamily 
\sphinxincludegraphics[width=0.300\linewidth]{{chromatic}.png} {\color{red}\bfseries{}|4578d16b23514691b1f3f21bcbf6ad74||6c02ec77ba6a4183b1d36a2208a36ace|}
\\
\hline
\sphinxstylestrong{Fig.:} Chromatic aberration. Left: Sketch of the chromatic aberration, focusing red light less strong than blue. Middle: Image from the lecture. Right: Rendered image using the refractive index for BK7 glass.
\\
\hline
\end{tabulary}
\par
\sphinxattableend\end{savenotes}

Such a chromatic aberrations may be corrected by using a system of two lenses as shown below.


\begin{savenotes}\sphinxattablestart
\centering
\begin{tabulary}{\linewidth}[t]{|T|}
\hline
\sphinxstyletheadfamily 
\sphinxincludegraphics[width=0.300\linewidth]{{achromat}.png}
\\
\hline
\sphinxstylestrong{Fig.:} Correction of chromatic aberration.
\\
\hline
\end{tabulary}
\par
\sphinxattableend\end{savenotes}

Such so\sphinxhyphen{}called achromatic lenses are typically consisting of a bixconvex lens that is glues to a bi\sphinxhyphen{}concave lens. Both lenses are made often made of different materials. Each of the lenses \(i\) have a focal length according to the lensmaker equation
\begin{equation*}
\begin{split}\frac{1}{f_i}=(n_i-1)\rho_i\end{split}
\end{equation*}
where the \(\rho\) is given by
\begin{equation*}
\begin{split}\rho_i=\frac{(R_{i2}-R_{i1})}{R_{i1}R_{i2}}\end{split}
\end{equation*}
Here the \(R_{i1},R_{i2}\) denote the different surfaces of the lens from left to right.

As we have treated lens systems already, we may refer to the calculation for short distances of the lenses and write down the refractive power of the lens system as
\begin{equation*}
\begin{split}\frac{1}{f}=(n_1-1)\rho_1+(n_2-1)\rho_2\end{split}
\end{equation*}
If now the regfractive index at two different wavelength is given, i.e. \(n_{1r},n_{1b}\) for the first lens and \(n_{2r},n_{2b}\) for the second lens, we may formulate our color correcrion wish as
\begin{equation*}
\begin{split}(n_{1r}-1)\rho_1+(n_{2r}-1)\rho_1=(n_{1b}-1)\rho_1+(n_{2b}-1)\rho_1\end{split}
\end{equation*}
This is just saying that the refractive powers of the lens system for the red and the blue colors should be the same, as then the two colors at least are focused into the same point. After some slight transformation, this gives us
\begin{equation*}
\begin{split}\frac{\rho_1}{\rho_2}=-\frac{n_{2b}-n_{2r}}{n_{1b}-n_{1r}}\end{split}
\end{equation*}
This is a condition for the radii of the two lenses, which can be further teared down, if we consider the lesn system in the image. There the two inner surfaces of the two lenses have the same radius of curvature, i.e. \(R_{12}=R_{21}\) and we also find \(R_{11}=R_1=-R_{12}=-R_{21}\) and \(R_{22}=R_2\). Using this, we may further simplify the above equation.


\subsection{Spherical Aberration}
\label{\detokenize{notebooks/L7/Imaging Errors:Spherical-Aberration}}
The spherical abberation arises due to the fact that we have always considered a simplification of the angluar functions to their first order Taylor series expansion. If the angles of incidence on the spherical surfaces get to large, we cannot do that anymore and need to consider higher order corrections.

The result is that parallel rays which are far from the optical axis are not imaged into the same focal point as the paraxial rays, but to points closer to the lens. You might have all seen such effect also in the case of your empty coffee cup, when the sunlight enters and causes a so\sphinxhyphen{}called caustics. This pattern, you observe there is also the result of a soherical aberration. The image below shows the spherical aberration of a lens.


\begin{savenotes}\sphinxattablestart
\centering
\begin{tabulary}{\linewidth}[t]{|T|}
\hline
\sphinxstyletheadfamily 
\sphinxincludegraphics[width=0.300\linewidth]{{spherical}.png} \sphinxincludegraphics[width=0.300\linewidth]{{spherical_exp}.png}
\\
\hline
\sphinxstylestrong{Fig.:} Spherical aberration.
\\
\hline
\end{tabulary}
\par
\sphinxattableend\end{savenotes}

To be a bit more qauntitative, we would like to reconsider the refraction at a single spherical surface as depicted in the image below.


\begin{savenotes}\sphinxattablestart
\centering
\begin{tabulary}{\linewidth}[t]{|T|}
\hline
\sphinxstyletheadfamily 
\sphinxincludegraphics[width=0.300\linewidth]{{spherical_theo}.png}
\\
\hline
\sphinxstylestrong{Fig.:} Spherical aberration.
\\
\hline
\end{tabulary}
\par
\sphinxattableend\end{savenotes}

For this case we may again write down the focal length with a more accurate calculation. We use therefore the relations \(\sin(\beta)=\sin(\alpha)/n, \sin(\alpha)=h/R, \alpha=\beta+\gamma\). With the help of those we obtain \(f=R+b\) and \(b=R\sin(\beta)/\sin(\gamma)\), which we may further convert into
\begin{equation*}
\begin{split}f=R\left [ 1+ \frac{1}{n\cos(\beta)-\cos(\alpha)}\right ]\end{split}
\end{equation*}
which we may further transform into
\begin{equation*}
\begin{split}f=R\left [ 1+ \frac{1}{n\sqrt{1-\frac{h^2}{n^2 R^2}})-\sqrt{1-\frac{h^2}{R^2}}}\right ]\end{split}
\end{equation*}
with the help of replacing the consines. Expansion of the square roots will further lead to
\begin{equation*}
\begin{split}f=R\left [ \frac{n}{n-1}- \frac{h^2}{2n(n-1)R^2} \right]\end{split}
\end{equation*}
This result is already enough to understand that the focal length (as in the case of the concave mirror) depends on the height \(h\) under which the ray is incident on the spherical surface. The second term in the square brackets represents exactly this term, which is shortening the focal length when \(h\neq 0\).

From that one may also obatain an imaging equation for s single spherical surface, which reads as
\begin{equation*}
\begin{split}\frac{1}{a}+\frac{n}{b}=\frac{n-1}{R}+h^2\left [ \frac{1}{2a}\left ( \frac{1}{a}+\frac{1}{R}\right )^2 +\frac{n}{2b}\left ( \frac{1}{R}-\frac{1}{b}\right )^2\right]\end{split}
\end{equation*}
This looks already for a single surface complicated but demonstrates that the image plane is now not anymore a plane but the its location depends on \(R\) and \(h\). This dependence for a single surface is then also reflected in other image distortions such as the field curvature.


\subsection{Coma}
\label{\detokenize{notebooks/L7/Imaging Errors:Coma}}
While we considered so far always rays parallel to the optical axis at different distance, rays which eminate from points not on the optical axis (also not on the optical axis at infinite distance), yield also aberrations. One of the is the “coma”. There rays inclined with the optical axis do not meet in a single off\sphinxhyphen{}axis point on the image side. They cause more intense regions with a comet like tail to the outside radial direction.


\begin{savenotes}\sphinxattablestart
\centering
\begin{tabulary}{\linewidth}[t]{|T|}
\hline
\sphinxstyletheadfamily 
\sphinxincludegraphics[width=0.450\linewidth]{{coma}.png} \sphinxincludegraphics[width=0.350\linewidth]{{coma_exp}.png}
\\
\hline
\sphinxstylestrong{Fig.:} Coma.
\\
\hline
\end{tabulary}
\par
\sphinxattableend\end{savenotes}


\subsection{Astigmatism}
\label{\detokenize{notebooks/L7/Imaging Errors:Astigmatism}}
Also the astigmatism arises when rays from point sources away from the optical axis are imaged by a lens. To understand the effect one may split the rays from such a source into rays which are in the vertical plane (meridional plane) and other which are in a plane perpendicular to it (sagittal plane). Considering those rays one finds,that the meridional rays are focused to a point closer to the lens as compared to the sagittal rays, which are focuse to a point at farther distance\sphinxhyphen{} This causes a
that the focus of the lens goes from an horziontal elliptical shape into a vertical elliptical shape when moving the screen through the focal region on the image side.

For an extended image as shown below, this results in the sperate focusing of vertical (left) and horizontal lines (right) in the image.


\begin{savenotes}\sphinxattablestart
\centering
\begin{tabulary}{\linewidth}[t]{|T|}
\hline
\sphinxstyletheadfamily 
\sphinxincludegraphics[width=0.325\linewidth]{{astigmatism}.png} \sphinxincludegraphics[width=0.325\linewidth]{{astigvert_exp}.png} \sphinxincludegraphics[width=0.310\linewidth]{{astighor_exp}.png}
\\
\hline
\sphinxstylestrong{Fig.:} Astigmatism of a lens showing the focusing of vertical (left) and horizontal (right) lines of an object (letter “F”) at different focus positions for a lens tilted in the beam path.
\\
\hline
\end{tabulary}
\par
\sphinxattableend\end{savenotes}

This distortion, i.e. the elliptical shape of the focus has been used advatageously in single molecule microscopy to locate their position alsong the optical axis, which is typically a challenge for optical microscopy.


\subsection{Field Curvature}
\label{\detokenize{notebooks/L7/Imaging Errors:Field-Curvature}}
The field curvature is related to our calculations of the spherical abberation. We have seen there, that the focal distance depends on the height \(h\) of the rays over the optical axis. This means also means that the image plane is actually not anymore a plane but a curved surface as shown below. The rays incident from point \(A_0\) and \(A_1\) do not meet in the same plane. This plane is even different for meridional and saggital rays. This typically results in the fact, that you
may have the center of the image in focus, but not the edges or vice versa.


\begin{savenotes}\sphinxattablestart
\centering
\begin{tabulary}{\linewidth}[t]{|T|}
\hline
\sphinxstyletheadfamily 
\sphinxincludegraphics[width=0.350\linewidth]{{field_curv}.png} \sphinxincludegraphics[width=0.350\linewidth]{{fieldcurvature_exp}.png}
\\
\hline
\sphinxstylestrong{Fig.:} Field curvature.
\\
\hline
\end{tabulary}
\par
\sphinxattableend\end{savenotes}


\subsection{Distortions}
\label{\detokenize{notebooks/L7/Imaging Errors:Distortions}}
Barrel or cushion shaped distortions in the image are found when inserting apertures in the optical path. This results in the removal of certain ray path ending up in field distortions.


\begin{savenotes}\sphinxattablestart
\centering
\begin{tabulary}{\linewidth}[t]{|T|}
\hline
\sphinxstyletheadfamily 
\sphinxincludegraphics[width=0.450\linewidth]{{cushion_exp}.png} \sphinxincludegraphics[width=0.450\linewidth]{{barrel_exp}.png}
\\
\hline
\sphinxstylestrong{Fig.:} Cushion (left) and barrel (right) type of distortions.
\\
\hline
\end{tabulary}
\par
\sphinxattableend\end{savenotes}

These distortions come from the fact that rays, which now must travel through and aperture are further or closer than their actual images without the aperture. This is best understood from the sketches below. The central ray here indicates the center \(M\) of the region into which the point \(A_0\) is imaged. If we insert an aperture behind the lens, the central ray will not be able to pass but other rays, which create an area around \(M_1\), which is at larger distance than
\(M\) from the optical axis. Now, the larger the point \(A_0\) is from the optical axis, the larger is also the distance of \(M_1\) from \(M\) and the optical axis. This turns a regular grid like in th eexperiment above into a cushion shaped object. The situation is reversed in the case when the aperture is inserted infront of the lens as dpiected on the right side of the sketch below. There the distortion is barrel shaped as validated by the experiment.


\begin{savenotes}\sphinxattablestart
\centering
\begin{tabulary}{\linewidth}[t]{|T|}
\hline
\sphinxstyletheadfamily 
\sphinxincludegraphics[width=0.450\linewidth]{{distortions}.png}
\\
\hline
\sphinxstylestrong{Fig.:} Cushion (left) and barrel (right) type of distortions.
\\
\hline
\end{tabulary}
\par
\sphinxattableend\end{savenotes}


\section{Lecture Contents}
\label{\detokenize{lectures/L8/overview_8:lecture-contents}}\label{\detokenize{lectures/L8/overview_8::doc}}
In Lecture 8 we start having a look at the wave properties of light, yet with a simpler description of wave optics. We will look at the solution of the wave equation and understand interfrence and diffraction.

\noindent\sphinxincludegraphics[width=600\sphinxpxdimen]{{slides11}.png}

Lecture 8 slides for download \sphinxcode{\sphinxupquote{pdf}}

The following section was created from \sphinxcode{\sphinxupquote{source/notebooks/L8/Wave Optics.ipynb}}.


\section{Wave Optics}
\label{\detokenize{notebooks/L8/Wave Optics:Wave-Optics}}\label{\detokenize{notebooks/L8/Wave Optics::doc}}
Wave optics extends our insight into optics with the help of a wave description. Light is able to interfere or to be diffracted at edges and it reveals specific colors in the visible range, where are not able to take account of that with the help of geometrical optics. The spectrum of electromagnetic waves extends over a huge range of frequencies and only a tiny fraction of it is related to the visible region.


\begin{savenotes}\sphinxattablestart
\centering
\begin{tabulary}{\linewidth}[t]{|T|}
\hline
\sphinxstyletheadfamily 
\sphinxincludegraphics[width=0.500\linewidth]{{spectrum1}.png}
\\
\hline
\sphinxstylestrong{Fig.:} Electromagnetic Spectrum with its different regions.
\\
\hline
\end{tabulary}
\par
\sphinxattableend\end{savenotes}

In the following, we would like to introduce wave by discarding the fact, that light is related to electric and magnetic fields. This is useful as the vectorial nature of the electric and magnetic field further complicates the calculations, but we do not need those yet. Accordingly we also do not understand how light really interacts with matter and we therefore have to introduce some postulates as well.


\subsection{Postulates of Wave Optics}
\label{\detokenize{notebooks/L8/Wave Optics:Postulates-of-Wave-Optics}}
\begin{sphinxadmonition}{note}{}\unskip
A wave corresponds to a physical quantity which oscillates in space and time. Its energy current density is related to the square magnitude of the amplitude.
\end{sphinxadmonition}


\subsubsection{Wave equation}
\label{\detokenize{notebooks/L8/Wave Optics:Wave-equation}}\begin{equation*}
\begin{split}\nabla^2 u - \frac{1}{c^2}\frac{\partial^2 u}{\partial t^2}=0\end{split}
\end{equation*}
where
\begin{equation*}
\begin{split}\nabla^2 =\frac{\partial^2}{\partial x^2}+\frac{\partial^2}{\partial y^2}+\frac{\partial^2}{\partial z^2}\end{split}
\end{equation*}
is the Laplace operator. The wave equation is a linear differential equation, which means that the superposition principle is valid. This means, that if we have found two solution \(u_1(\vec{r},t)\) and \(u_2(\vec{r},t)\) being solutions of the wave equation, then \(u(\vec{r},t)=a_1 u_1(\vec{r},t)+a_2*u_2(\vec{r},t)\) is a solution as well (\(a_1,a_2\) being constants)


\subsubsection{Intensity of waves}
\label{\detokenize{notebooks/L8/Wave Optics:Intensity-of-waves}}\begin{equation*}
\begin{split}I(\vec{r},t)=2\langle u^2(\vec{r},t)\rangle\end{split}
\end{equation*}
which is given in units \(\left[\frac{W}{m^2}\right]\). The \(\langle \ldots \rangle\) denotes a time average over one cycle of the oscillation of \(u\). This is extremely short. For light of 600 nm wavelength, one cycle just lasts 2 femtoseconds.

The optical power of a wave is obtained when integrating the intensity over an area \(A\).
\begin{equation*}
\begin{split}P=\int_A I(\vec{r},t) dA\end{split}
\end{equation*}

\subsubsection{Monochromatic wave}
\label{\detokenize{notebooks/L8/Wave Optics:Monochromatic-wave}}
A monochromatic wave is obtained whem only one single frequency \(\omega\) is contained in a wave. For this to be true, the wave has to be inifnitely long in time and the oscillation should not have any phase disturbances (such as jumps). A monochromatic wave is given by.
\begin{equation*}
\begin{split}u(\vec{r},t)=a(\vec{r})\cos(\omega t + \phi(\vec{r}))\end{split}
\end{equation*}
where
\begin{itemize}
\item {} 
\(a(\vec{r})\) is the amplitude

\item {} 
\(\phi(\vec{r})\) is the spatial phase

\item {} 
\(\omega\) is the frequency

\end{itemize}


\begin{savenotes}\sphinxattablestart
\centering
\begin{tabulary}{\linewidth}[t]{|T|}
\hline
\sphinxstyletheadfamily 
\sphinxincludegraphics[width=0.500\linewidth]{{wave}.png}
\\
\hline
\sphinxstylestrong{Fig.:} Representation of a wavefunction over time (constant position) denoting the phase \(\phi\) and the period \(T=1/\nu\).
\\
\hline
\end{tabulary}
\par
\sphinxattableend\end{savenotes}


\paragraph{Complex Amplitude}
\label{\detokenize{notebooks/L8/Wave Optics:Complex-Amplitude}}
We may also introduce a complex representation of a wave by
\begin{equation*}
\begin{split}U(\vec{r},t)=a(\vec{r})e^{i\phi(r)}e^{i\omega t}\end{split}
\end{equation*}
which is called the complex wavefunction.


\begin{savenotes}\sphinxattablestart
\centering
\begin{tabulary}{\linewidth}[t]{|T|}
\hline
\sphinxstyletheadfamily 
\sphinxincludegraphics[width=0.500\linewidth]{{complex_rep}.png}
\\
\hline
\sphinxstylestrong{Fig.:} Phasor diagram of the complex amplitude \(U(r)\) (left) and \(U(t)\) (right).
\\
\hline
\end{tabulary}
\par
\sphinxattableend\end{savenotes}

\begin{sphinxadmonition}{note}{}\unskip
A phasor displays the complex amplitude with magnitude and phase as a vector in the complex plane.
\end{sphinxadmonition}

This complex wavefunction is related to the real wavefunction by
\begin{equation*}
\begin{split}u(\vec{r},t)=Re\lbrace U(\vec{r},t)\rbrace=\frac{1}{2}\left [ U(\vec{r},t)+U^*(\vec{r},t)\right ]\end{split}
\end{equation*}
but allows us much easier calculations. In the same way as for real wavefunction we may now also write a wave equation for the complex wavefunction
\begin{equation*}
\begin{split}\nabla^2 U - \frac{1}{c^2}\frac{\partial^2 U}{\partial t^2}=0\end{split}
\end{equation*}
We may now further split up the complex wavefunction into a spatial and temporal dependence
\begin{equation*}
\begin{split}U(\vec{r},t)=U(\vec{r})e^{i\omega t}\end{split}
\end{equation*}
where
\begin{equation*}
\begin{split}U(\vec{r},t)=a(\vec{r})e^{i\phi(\vec{r})}\end{split}
\end{equation*}
The quantity \(\phi\) resembles to be the spatial part of the phase of a wavefunction. The intensity of the wave if further now just given by
\begin{equation*}
\begin{split}I(\vec{r})=|U(r)|^2\end{split}
\end{equation*}

\paragraph{Wavefronts}
\label{\detokenize{notebooks/L8/Wave Optics:Wavefronts}}
Wavefronts are the surfaces in space where the phase \(\phi(\vec{r})={\rm const}\). This constant can be chosen such that we always denote the maximum of the spatial amplitude, i.e. \(\phi(\vec{r})=2\pi q\), where \(q\) is an integer. We will later have a look how those wavefronts propagate in time. We can also define a vector which is perpendicular to the wavefront. This is given by
\begin{equation*}
\begin{split}\vec{n}=\left \{ \frac{\partial \phi}{\partial x},\frac{\partial \phi}{\partial y},\frac{\partial \phi}{\partial z} \right \}\end{split}
\end{equation*}

\subsection{Plane Waves}
\label{\detokenize{notebooks/L8/Wave Optics:Plane-Waves}}
A plane wave is a solution of the homogeneous wave equation and is given in its complex form by

\begin{equation}
U=Ae^{-i\vec{k}\cdot \vec{r}}e^{i\omega t}
\end{equation}

where the two exponentials contain an spatial and a temporal phase. \(A\) denotes the amplitude of the plane wave and can be a complex number as well. The plane is defined by the shape of the wavefront which is given by \(\vec{k}\cdot \vec{r}=2\pi q + {\rm arg}{A}\), which is just the definition of a plane perpendicular to \(\vec{k}\). The planes are seperated by the wavelength \(\lambda=2\pi /k\). The wavenumber \(k\) is therefore the spatial frequency of the oscillation in
position of a wave.

The spatial amplitude of the plane wave is given by

\begin{equation}
U(\vec{r})=Ae^{-i\vec{k}\cdot \vec{r}}
\end{equation}

The propagation direction of the wave is defined by wavevector \(\vec{k}\). In vacuum, the wavevector is just real valued

\begin{equation}
\vec{k}_{0}=
\begin{pmatrix}
k_{0x} \\
k_{0y}\\
k_{0z}\\
\end{pmatrix}
\end{equation}


\begin{savenotes}\sphinxattablestart
\centering
\begin{tabulary}{\linewidth}[t]{|T|}
\hline
\sphinxstyletheadfamily 
\sphinxincludegraphics[width=0.450\linewidth]{{plane_wave}.png}
\\
\hline
\sphinxstylestrong{Fig.:} Plane wave propagating along the horizontal direction.
\\
\hline
\end{tabulary}
\par
\sphinxattableend\end{savenotes}

The wavevector is also proportional to the momentum of the wave, which will be important if we consider the refraction process a bit later. The magnitude of the wavevector is related wo the frequency

\begin{equation}
k_{0}=\frac{\omega}{c_{0}}
\end{equation}

This relation connects the momentum (\(k\)) to the energy \(\omega\) and is in general in physics called \sphinxstylestrong{dispersion relation}. Light in free space obeys a linear dispersion relation.

If we consider a wave propagating in a medium, the frequency of the wave \(\omega\) stays the same as this is defined by the source. Nevertheless the propagation speed of the wave changes, i.e.
\begin{equation*}
\begin{split}c=\frac{c}{n_0}\end{split}
\end{equation*}
and by a factor lower (the refractive index \(n\) of the medium). As a result, the wavelength of the wave changes in a medium and gets typically shorter for \(n>1\).
\begin{equation*}
\begin{split}\lambda=\frac{\lambda_0}{n}\end{split}
\end{equation*}
Correspondingly. The wavenumber gets larger,
\begin{equation*}
\begin{split}k=nk_{0}\end{split}
\end{equation*}
.

The above graph shows a static snapshot of the plane wave at a time \(t=0\). The animation below shows the plane wave propagation of a \(\lambda=532\) nm wave.


\begin{savenotes}\sphinxattablestart
\centering
\begin{tabulary}{\linewidth}[t]{|T|}
\hline
\sphinxstyletheadfamily 
\sphinxincludegraphics[width=0.400\linewidth]{{plane_prop}.mov}
\\
\hline
\sphinxstylestrong{Fig.:} . Propagating spherical waves for positive and negative wavenumber.
\\
\hline
\end{tabulary}
\par
\sphinxattableend\end{savenotes}

A reversal of the wavevector to its negative value, changes the propagation direction to the opposite direction.


\subsection{Spherical Waves}
\label{\detokenize{notebooks/L8/Wave Optics:Spherical-Waves}}
A spherical wave is as well described by two exponentials containing the spatial and temporal dependence of the wave. The only difference is, that the wavefronts shall describe spheres instead of planes. We therefore need \(|\vec{k}||\vec{r}|=k r=const\). The product of the magntitudes of the wavevector and the distance from the source are constant. If we further generalize the position of the source to \(\vec{r}_{0}\) we can write a spherical wave by

\begin{equation}
U=\frac{A}{|\vec{r}-\vec{r}_{0}|}e^{-i k|\vec{r}-\vec{r}_{0}|} e^{i\omega t}
\end{equation}

Note that we have to introduce an additional scaling of the amplitude with the inverse distance of the source. This is due to energy conservation, as we require that all the energy that flows through all spheres around the source is constant.

The line plots below show that the field amplitude rapidly decays and the intensity follows a \(1/r^2\) law as expected. The slight deiviation at small distances is an artifact from our discretization. We used the image above to extract the line plot and therefore never exactly hit \(r=0\).


\begin{savenotes}\sphinxattablestart
\centering
\begin{tabulary}{\linewidth}[t]{|T|}
\hline
\sphinxstyletheadfamily 
\sphinxincludegraphics[width=0.800\linewidth]{{spherical_wave}.png}
\\
\hline
\sphinxstylestrong{Fig.:} Spherical wave amplitude and intensity of the spherical wave as a function of distance from the source.
\\
\hline
\end{tabulary}
\par
\sphinxattableend\end{savenotes}

We can also visualize the animation our spherical wave to check for the direction of the wave propagation.


\begin{savenotes}\sphinxattablestart
\centering
\begin{tabulary}{\linewidth}[t]{|T|}
\hline
\sphinxstyletheadfamily 
\sphinxincludegraphics[width=0.300\linewidth]{{out}.mov} \sphinxincludegraphics[width=0.300\linewidth]{{in}.mov}
\\
\hline
\sphinxstylestrong{Fig.:} . Propagating spherical waves for positive and negative wavenumber.
\\
\hline
\end{tabulary}
\par
\sphinxattableend\end{savenotes}


\subsection{Interference}
\label{\detokenize{notebooks/L8/Wave Optics:Interference}}
Interference is one of the most important phenomena in physics. It is a realization of the superposition principle which is valid, if the differential equation which produce the solutions is linear as we already stated above.

Let’s assume we have two solutions of the wave equation \(U_1(\vec{r})\) and \(U_2(\vec{r})\). The superposition allows to add them to get a new solution \(U(\vec{r})\)
\begin{equation*}
\begin{split}U(\vec{r})=U_{1}(\vec{r})+U_{2}(\vec{r})\end{split}
\end{equation*}
To calculate the intensity of the total wave, we have to calculate therfore the magnitude square of the sum of the two waves.

:nbsphinx\sphinxhyphen{}math:{\color{red}\bfseries{}\textasciigrave{}}begin\{eqnarray\}
I \&= \&|U|\textasciicircum{}2\textbackslash{}
\begin{quote}

\&= \&|U\_1+U\_2|\textbackslash{}
\&= \&|U\_1|\textasciicircum{}2+|U\_2|\textasciicircum{}2+U\textasciicircum{}\{*\}\_1 U\_2 + U\_1 U\textasciicircum{}\{*\}\_1
\end{quote}

end\{eqnarray\}\textasciigrave{}

which results in two terms, which resemble to be the sum of the two waves. Yet, there are two additional cross\sphinxhyphen{}term, which say that the resulting intensity is not just the sum of the two intensities in case the two terms do not cancel out.

If we use
\begin{equation*}
\begin{split}I_{1}=|U_1|^2\end{split}
\end{equation*}\begin{equation*}
\begin{split}I_{2}=|U_2|^2\end{split}
\end{equation*}
we can express the individual waves as
\begin{equation*}
\begin{split}U_{1}=\sqrt{I_1}e^{i\phi_1}\end{split}
\end{equation*}
and
\begin{equation*}
\begin{split}U_{2}=\sqrt{I_2}e^{i\phi_2}\end{split}
\end{equation*}
where the exponential term with \(\phi_1,\phi_2\) denotes the phase of the two spatial amplitudes. Taking the magnitude square of the sum of those two wave we then obtain
\begin{equation*}
\begin{split}I=I_{1}+I_{2}+2\sqrt{I_1 I_2}\cos(\Delta \phi)\end{split}
\end{equation*}
with
\begin{equation*}
\begin{split}\Delta \phi=\phi_{2}-\phi_1\end{split}
\end{equation*}
which now clearly tells us, that the total intensity of two waves in not the sum of the two intensities, but contains an additional term, which depends on the phase difference between the two wave. This term \(2\sqrt{I_1 I_2}\cos(\Delta \phi)\) is called the interference term.

Let us assume for an example two wave, which originally have the same intensity, i.e. \(I_2=I_1=I_0\). If we insert this into the abocve formulas we obtain for the total intensity
\begin{equation*}
\begin{split}I=2I_{0}(1+\cos(\Delta \phi)=4I_{0})\cos^2\left (\frac{\Delta \phi}{2}\right)\end{split}
\end{equation*}
This is quite interesting as we obtain the follwing results for specific phase shifts

\sphinxstylestrong{Contructive Interference}
\begin{itemize}
\item {} 
\(\Delta \phi=0,2\pi,2\pi m\): \(I=4I_{0}\), where m is an integer

\end{itemize}


\begin{savenotes}\sphinxattablestart
\centering
\begin{tabulary}{\linewidth}[t]{|T|}
\hline
\sphinxstyletheadfamily 
\sphinxincludegraphics[width=0.330\linewidth]{{w_0}.png} \sphinxincludegraphics[width=0.330\linewidth]{{w_0}.png} \sphinxincludegraphics[width=0.330\linewidth]{{sum_const}.png}
\\
\hline
\sphinxstylestrong{Fig.:} Constructive interference of two wave (left, middle) and the sumn of the two wave amplitudes.
\\
\hline
\end{tabulary}
\par
\sphinxattableend\end{savenotes}

\begin{sphinxadmonition}{note}{}\unskip
\sphinxstylestrong{Constructive Interference}

Constructive interference between wave occurs when the phase difference between the waves is a multiple of \(2\pi\).
\begin{equation*}
\begin{split}\Delta \phi=2\pi m\end{split}
\end{equation*}
where \(m\) is an integer number.
\end{sphinxadmonition}

\sphinxstylestrong{Destructive Interference}
\begin{itemize}
\item {} 
\(\Delta \phi=\pi,3\pi,(2m-1)\pi\): \(I=0\), where m is an integer

\end{itemize}


\begin{savenotes}\sphinxattablestart
\centering
\begin{tabulary}{\linewidth}[t]{|T|}
\hline
\sphinxstyletheadfamily 
\sphinxincludegraphics[width=0.330\linewidth]{{w_0}.png} \sphinxincludegraphics[width=0.330\linewidth]{{w_pi}.png} \sphinxincludegraphics[width=0.330\linewidth]{{sum_destr}.png}
\\
\hline
\sphinxstylestrong{Fig.:} Destructive interference of two wave (left, middle) and the sumn of the two wave amplitudes.
\\
\hline
\end{tabulary}
\par
\sphinxattableend\end{savenotes}

\begin{sphinxadmonition}{note}{}\unskip
\sphinxstylestrong{Destructive Interference}

Destructive interference between wave occurs when the phase difference between the waves is an odd multiple of \(2\pi\).
\begin{equation*}
\begin{split}\Delta \phi=(2m-1)\pi\end{split}
\end{equation*}
where \(m\) is an integer number.
\end{sphinxadmonition}

\sphinxstylestrong{Sum of both Intensities}
\begin{itemize}
\item {} 
\(\Delta \phi=\pi/2,3\pi/2,(2m-1)\pi/2\): \(I=2I_0\), where m is an integer

\end{itemize}

\begin{sphinxadmonition}{note}{}\unskip
\sphinxstylestrong{Phase Difference and Path Difference}

Note that a phase difference can be caused by a difference in the path length the two waves have to travel. This pathlength difference \(\Delta s\) then correponds to a phase amngle difference of \(k\Delta s=2\pi\Delta s /\lambda\), where \(k\) is the wavenumber. A pathlength difference, which is a multiple of \(\lambda\), will thus correspond to a phase angle difference, which is a multiple of \(2\pi\).
\end{sphinxadmonition}

Interference may give rise to distinct patterns in space, which can be used for lithography purposes, for example. If we superpose two plane waves with different directions of the wave propagation, we will in general obtain a stripe like intensity pattern as depicted below.


\begin{savenotes}\sphinxattablestart
\centering
\begin{tabulary}{\linewidth}[t]{|T|}
\hline
\sphinxstyletheadfamily 
\sphinxincludegraphics[width=0.800\linewidth]{{plane_plane}.png}
\\
\hline
\sphinxstylestrong{Fig.:} Interference of two plane waves propagating under an angle of 45\(^{\circ}\). The two left graphs show the original waves. The two right the total amplitude and the intensity pattern.
\\
\hline
\end{tabulary}
\par
\sphinxattableend\end{savenotes}

The interference of the two plane wave shown above leads also to stripe pattern on a screen (think of the intensity pattern at z=10 µm for example).

If we use a spherical and a plane wave, we will find a circular shaped interference intensity due to the curvature of the wavefronts of the spherical wave.


\begin{savenotes}\sphinxattablestart
\centering
\begin{tabulary}{\linewidth}[t]{|T|}
\hline
\sphinxstyletheadfamily 
\sphinxincludegraphics[width=0.900\linewidth]{{sphere_plane}.png}
\\
\hline
\sphinxstylestrong{Fig.:} Interference of a spherical wave and a plane wave (left amplitude, right intensity).
\\
\hline
\end{tabulary}
\par
\sphinxattableend\end{savenotes}

The following section was created from \sphinxcode{\sphinxupquote{source/snippets/Wave Explorer.ipynb}}.


\section{EXP3 Snippets: Wave Explorer}
\label{\detokenize{snippets/Wave Explorer:EXP3-Snippets:-Wave-Explorer}}\label{\detokenize{snippets/Wave Explorer::doc}}
This notebook provides function and plots to explore plane waves and spherical waves. You may need some experience in Python programming.

{
\sphinxsetup{VerbatimColor={named}{nbsphinx-code-bg}}
\sphinxsetup{VerbatimBorderColor={named}{nbsphinx-code-border}}
\begin{sphinxVerbatim}[commandchars=\\\{\}]
\llap{\color{nbsphinxin}[2]:\,\hspace{\fboxrule}\hspace{\fboxsep}}\PYG{k+kn}{import} \PYG{n+nn}{numpy} \PYG{k}{as} \PYG{n+nn}{np}
\PYG{k+kn}{import} \PYG{n+nn}{matplotlib}\PYG{n+nn}{.}\PYG{n+nn}{pyplot} \PYG{k}{as} \PYG{n+nn}{plt}
\PYG{k+kn}{from} \PYG{n+nn}{time} \PYG{k+kn}{import} \PYG{n}{sleep}\PYG{p}{,}\PYG{n}{time}
\PYG{k+kn}{from} \PYG{n+nn}{ipycanvas} \PYG{k+kn}{import} \PYG{n}{MultiCanvas}\PYG{p}{,} \PYG{n}{hold\PYGZus{}canvas}\PYG{p}{,}\PYG{n}{Canvas}
\PYG{k+kn}{import} \PYG{n+nn}{matplotlib} \PYG{k}{as} \PYG{n+nn}{mpl}
\PYG{k+kn}{import} \PYG{n+nn}{matplotlib}\PYG{n+nn}{.}\PYG{n+nn}{cm} \PYG{k}{as} \PYG{n+nn}{cm}


\PYG{o}{\PYGZpc{}}\PYG{k}{matplotlib} inline
\PYG{o}{\PYGZpc{}}\PYG{k}{config} InlineBackend.figure\PYGZus{}format = \PYGZsq{}retina\PYGZsq{}

\PYG{c+c1}{\PYGZsh{} default values for plotting}
\PYG{n}{plt}\PYG{o}{.}\PYG{n}{rcParams}\PYG{o}{.}\PYG{n}{update}\PYG{p}{(}\PYG{p}{\PYGZob{}}\PYG{l+s+s1}{\PYGZsq{}}\PYG{l+s+s1}{font.size}\PYG{l+s+s1}{\PYGZsq{}}\PYG{p}{:} \PYG{l+m+mi}{12}\PYG{p}{,}
                     \PYG{l+s+s1}{\PYGZsq{}}\PYG{l+s+s1}{axes.titlesize}\PYG{l+s+s1}{\PYGZsq{}}\PYG{p}{:} \PYG{l+m+mi}{18}\PYG{p}{,}
                     \PYG{l+s+s1}{\PYGZsq{}}\PYG{l+s+s1}{axes.labelsize}\PYG{l+s+s1}{\PYGZsq{}}\PYG{p}{:} \PYG{l+m+mi}{16}\PYG{p}{,}
                     \PYG{l+s+s1}{\PYGZsq{}}\PYG{l+s+s1}{axes.labelpad}\PYG{l+s+s1}{\PYGZsq{}}\PYG{p}{:} \PYG{l+m+mi}{14}\PYG{p}{,}
                     \PYG{l+s+s1}{\PYGZsq{}}\PYG{l+s+s1}{lines.linewidth}\PYG{l+s+s1}{\PYGZsq{}}\PYG{p}{:} \PYG{l+m+mi}{1}\PYG{p}{,}
                     \PYG{l+s+s1}{\PYGZsq{}}\PYG{l+s+s1}{lines.markersize}\PYG{l+s+s1}{\PYGZsq{}}\PYG{p}{:} \PYG{l+m+mi}{10}\PYG{p}{,}
                     \PYG{l+s+s1}{\PYGZsq{}}\PYG{l+s+s1}{xtick.labelsize}\PYG{l+s+s1}{\PYGZsq{}} \PYG{p}{:} \PYG{l+m+mi}{16}\PYG{p}{,}
                     \PYG{l+s+s1}{\PYGZsq{}}\PYG{l+s+s1}{ytick.labelsize}\PYG{l+s+s1}{\PYGZsq{}} \PYG{p}{:} \PYG{l+m+mi}{16}\PYG{p}{,}
                     \PYG{l+s+s1}{\PYGZsq{}}\PYG{l+s+s1}{xtick.top}\PYG{l+s+s1}{\PYGZsq{}} \PYG{p}{:} \PYG{k+kc}{True}\PYG{p}{,}
                     \PYG{l+s+s1}{\PYGZsq{}}\PYG{l+s+s1}{xtick.direction}\PYG{l+s+s1}{\PYGZsq{}} \PYG{p}{:} \PYG{l+s+s1}{\PYGZsq{}}\PYG{l+s+s1}{in}\PYG{l+s+s1}{\PYGZsq{}}\PYG{p}{,}
                     \PYG{l+s+s1}{\PYGZsq{}}\PYG{l+s+s1}{ytick.right}\PYG{l+s+s1}{\PYGZsq{}} \PYG{p}{:} \PYG{k+kc}{True}\PYG{p}{,}
                     \PYG{l+s+s1}{\PYGZsq{}}\PYG{l+s+s1}{ytick.direction}\PYG{l+s+s1}{\PYGZsq{}} \PYG{p}{:} \PYG{l+s+s1}{\PYGZsq{}}\PYG{l+s+s1}{in}\PYG{l+s+s1}{\PYGZsq{}}\PYG{p}{,}\PYG{p}{\PYGZcb{}}\PYG{p}{)}
\end{sphinxVerbatim}
}


\subsection{Plane Waves}
\label{\detokenize{snippets/Wave Explorer:Plane-Waves}}
A plane wave is a solution of the homogeneous wave equation and is given in its complex form by

\begin{equation}
U=Ae^{-i\vec{k}\cdot \vec{r}}e^{i\omega t}
\end{equation}

where the two exponentials contain an spatial and a temporal phase. \(A\) denotes the amplitude of the plane wave. The plane is defined by the shape of the wavefront which is given by \(\vec{k}\cdot \vec{r}=const\), which is just the definition of a plane perpendicular to \(\vec{k}\).

A wave is a physical quantity which oscillates in space and time. Its energy current density is related to the square magnitude of the amplitude. We will include in the following the spatial and the temporal phase. For plotting just the spatial variation of the wave, you may just use the spatial part of the equation

\begin{equation}
U(\vec{r})=Ae^{-i\vec{k}\cdot \vec{r}}
\end{equation}

But since we also want to see the wave propagate, we will directly include also the temporal dependence on our function. In all of the computational examples below we set the amplitude of the wave \(A=1\).

The propagation of the wave is defined by wavevector \(\vec{k}\). In vacuum, the wavevector is just real valued

\begin{equation}
\vec{k}_{0}=
\begin{pmatrix}
k_{0x} \\
k_{0y}\\
k_{0z}\\
\end{pmatrix}
\end{equation}

The wavevector is providing the direction in which the wavefronts propagate. It is also proportional to the momentum of the wave, which will be important if we consider the refraction process a bit later. The magnitude of the wavevector is related to the wavelength \(\lambda\) and called the wavenumber.

\begin{equation}
k_{0}=\frac{2\pi}{\lambda_{0}}=\frac{\omega}{c_{0}}
\end{equation}

At the same time, its magnitude is also given by the frequency of the light devided by the wave vector. The latter is called a dispersion relation.

With a few lines of Python code, we will explore planes waves as well as their propagation.

{
\sphinxsetup{VerbatimColor={named}{nbsphinx-code-bg}}
\sphinxsetup{VerbatimBorderColor={named}{nbsphinx-code-border}}
\begin{sphinxVerbatim}[commandchars=\\\{\}]
\llap{\color{nbsphinxin}[3]:\,\hspace{\fboxrule}\hspace{\fboxsep}}\PYG{k}{def} \PYG{n+nf}{plane\PYGZus{}wave}\PYG{p}{(}\PYG{n}{k}\PYG{p}{,}\PYG{n}{omega}\PYG{p}{,}\PYG{n}{r}\PYG{p}{,}\PYG{n}{t}\PYG{p}{)}\PYG{p}{:}
    \PYG{k}{return}\PYG{p}{(}\PYG{n}{np}\PYG{o}{.}\PYG{n}{exp}\PYG{p}{(}\PYG{l+m+mi}{1}\PYG{n}{j}\PYG{o}{*}\PYG{p}{(}\PYG{n}{np}\PYG{o}{.}\PYG{n}{dot}\PYG{p}{(}\PYG{n}{k}\PYG{p}{,}\PYG{n}{r}\PYG{p}{)}\PYG{o}{\PYGZhy{}}\PYG{n}{omega}\PYG{o}{*}\PYG{n}{t}\PYG{p}{)}\PYG{p}{)}\PYG{p}{)}
\end{sphinxVerbatim}
}

Lets have a look at waves and wave propagation. We want to create a wave, which has a wavelength of 532 nm in vacuum.

{
\sphinxsetup{VerbatimColor={named}{nbsphinx-code-bg}}
\sphinxsetup{VerbatimBorderColor={named}{nbsphinx-code-border}}
\begin{sphinxVerbatim}[commandchars=\\\{\}]
\llap{\color{nbsphinxin}[4]:\,\hspace{\fboxrule}\hspace{\fboxsep}}\PYG{n}{wavelength}\PYG{o}{=}\PYG{l+m+mf}{532e\PYGZhy{}9}
\PYG{n}{k0}\PYG{o}{=}\PYG{l+m+mi}{2}\PYG{o}{*}\PYG{n}{np}\PYG{o}{.}\PYG{n}{pi}\PYG{o}{/}\PYG{n}{wavelength}
\PYG{n}{c}\PYG{o}{=}\PYG{l+m+mi}{299792458}
\PYG{n}{omega0}\PYG{o}{=}\PYG{n}{k0}\PYG{o}{*}\PYG{n}{c}
\end{sphinxVerbatim}
}

It shall propagate along the z\sphinxhyphen{}direction and we wull have a look at the x\sphinxhyphen{}z plane.

{
\sphinxsetup{VerbatimColor={named}{nbsphinx-code-bg}}
\sphinxsetup{VerbatimBorderColor={named}{nbsphinx-code-border}}
\begin{sphinxVerbatim}[commandchars=\\\{\}]
\llap{\color{nbsphinxin}[5]:\,\hspace{\fboxrule}\hspace{\fboxsep}}\PYG{n}{vec}\PYG{o}{=}\PYG{n}{np}\PYG{o}{.}\PYG{n}{array}\PYG{p}{(}\PYG{p}{[}\PYG{l+m+mf}{0.}\PYG{p}{,}\PYG{l+m+mf}{0.}\PYG{p}{,}\PYG{l+m+mf}{1.}\PYG{p}{]}\PYG{p}{)}
\PYG{n}{vec}\PYG{o}{=}\PYG{n}{vec}\PYG{o}{/}\PYG{n}{np}\PYG{o}{.}\PYG{n}{sqrt}\PYG{p}{(}\PYG{n}{np}\PYG{o}{.}\PYG{n}{dot}\PYG{p}{(}\PYG{n}{vec}\PYG{p}{,}\PYG{n}{vec}\PYG{p}{)}\PYG{p}{)}

\PYG{n}{k}\PYG{o}{=}\PYG{n}{k0}\PYG{o}{*}\PYG{n}{vec}
\end{sphinxVerbatim}
}

We can plot the electric field in the x\sphinxhyphen{}z plane by defining a grid of points (x,z). This is done by the \sphinxstyleemphasis{meshgrid} function of \sphinxstyleemphasis{numpy}. The meshgrid returns a 2\sphinxhyphen{}dimensional array for each coordinate. Have a look at the values in the meshgrid.

{
\sphinxsetup{VerbatimColor={named}{nbsphinx-code-bg}}
\sphinxsetup{VerbatimBorderColor={named}{nbsphinx-code-border}}
\begin{sphinxVerbatim}[commandchars=\\\{\}]
\llap{\color{nbsphinxin}[6]:\,\hspace{\fboxrule}\hspace{\fboxsep}}\PYG{n}{x}\PYG{o}{=}\PYG{n}{np}\PYG{o}{.}\PYG{n}{linspace}\PYG{p}{(}\PYG{o}{\PYGZhy{}}\PYG{l+m+mf}{2.5e\PYGZhy{}6}\PYG{p}{,}\PYG{l+m+mf}{2.5e\PYGZhy{}6}\PYG{p}{,}\PYG{l+m+mi}{300}\PYG{p}{)}
\PYG{n}{z}\PYG{o}{=}\PYG{n}{np}\PYG{o}{.}\PYG{n}{linspace}\PYG{p}{(}\PYG{l+m+mi}{0}\PYG{p}{,}\PYG{l+m+mf}{5e\PYGZhy{}6}\PYG{p}{,}\PYG{l+m+mi}{300}\PYG{p}{)}

\PYG{n}{X}\PYG{p}{,}\PYG{n}{Z}\PYG{o}{=}\PYG{n}{np}\PYG{o}{.}\PYG{n}{meshgrid}\PYG{p}{(}\PYG{n}{x}\PYG{p}{,}\PYG{n}{z}\PYG{p}{)}
\PYG{n}{r}\PYG{o}{=}\PYG{n}{np}\PYG{o}{.}\PYG{n}{array}\PYG{p}{(}\PYG{p}{[}\PYG{n}{X}\PYG{p}{,}\PYG{l+m+mi}{0}\PYG{p}{,}\PYG{n}{Z}\PYG{p}{]}\PYG{p}{,}\PYG{n}{dtype}\PYG{o}{=}\PYG{n+nb}{object}\PYG{p}{)}
\end{sphinxVerbatim}
}

In the last lines, we defined an array of X,0,Z, where X and Z are already 2\sphinxhyphen{}dimensional array. This finally gives an array 3D vectors, which we can use to calculate the electric field at any point in space. If we want to plot the electric field, we have to calculate the real part of the complex values, as the electric field is a physical quantity, which is always real. There is not much to see for a plane wave in the intensity plot, as the intensity of a plane wave is constant in space. Yet, if
you want to plot it, you have to calculate the magnitude square of the electric field, e.g.

\begin{equation}
I\propto |E|^{2}
\end{equation}

{
\sphinxsetup{VerbatimColor={named}{nbsphinx-code-bg}}
\sphinxsetup{VerbatimBorderColor={named}{nbsphinx-code-border}}
\begin{sphinxVerbatim}[commandchars=\\\{\}]
\llap{\color{nbsphinxin}[7]:\,\hspace{\fboxrule}\hspace{\fboxsep}}\PYG{n}{plt}\PYG{o}{.}\PYG{n}{figure}\PYG{p}{(}\PYG{n}{figsize}\PYG{o}{=}\PYG{p}{(}\PYG{l+m+mi}{12}\PYG{p}{,}\PYG{l+m+mi}{5}\PYG{p}{)}\PYG{p}{)}
\PYG{n}{field}\PYG{o}{=}\PYG{n}{plane\PYGZus{}wave}\PYG{p}{(}\PYG{n}{k}\PYG{p}{,}\PYG{n}{omega0}\PYG{p}{,}\PYG{n}{r}\PYG{p}{,}\PYG{l+m+mi}{0}\PYG{p}{)}
\PYG{n}{extent} \PYG{o}{=} \PYG{n}{np}\PYG{o}{.}\PYG{n}{min}\PYG{p}{(}\PYG{n}{z}\PYG{p}{)}\PYG{o}{*}\PYG{l+m+mf}{1e6}\PYG{p}{,} \PYG{n}{np}\PYG{o}{.}\PYG{n}{max}\PYG{p}{(}\PYG{n}{z}\PYG{p}{)}\PYG{o}{*}\PYG{l+m+mf}{1e6}\PYG{p}{,}\PYG{n}{np}\PYG{o}{.}\PYG{n}{min}\PYG{p}{(}\PYG{n}{x}\PYG{p}{)}\PYG{o}{*}\PYG{l+m+mf}{1e6}\PYG{p}{,} \PYG{n}{np}\PYG{o}{.}\PYG{n}{max}\PYG{p}{(}\PYG{n}{x}\PYG{p}{)}\PYG{o}{*}\PYG{l+m+mf}{1e6}

\PYG{n}{plt}\PYG{o}{.}\PYG{n}{title}\PYG{p}{(}\PYG{l+s+s1}{\PYGZsq{}}\PYG{l+s+s1}{electric field}\PYG{l+s+s1}{\PYGZsq{}}\PYG{p}{)}
\PYG{n}{plt}\PYG{o}{.}\PYG{n}{imshow}\PYG{p}{(}\PYG{n}{np}\PYG{o}{.}\PYG{n}{real}\PYG{p}{(}\PYG{n}{field}\PYG{o}{.}\PYG{n}{transpose}\PYG{p}{(}\PYG{p}{)}\PYG{p}{)}\PYG{p}{,}\PYG{n}{extent}\PYG{o}{=}\PYG{n}{extent}\PYG{p}{,}\PYG{n}{vmin}\PYG{o}{=}\PYG{o}{\PYGZhy{}}\PYG{l+m+mi}{1}\PYG{p}{,}\PYG{n}{vmax}\PYG{o}{=}\PYG{l+m+mi}{1}\PYG{p}{,}\PYG{n}{cmap}\PYG{o}{=}\PYG{l+s+s1}{\PYGZsq{}}\PYG{l+s+s1}{seismic}\PYG{l+s+s1}{\PYGZsq{}}\PYG{p}{)}
\PYG{n}{plt}\PYG{o}{.}\PYG{n}{xlabel}\PYG{p}{(}\PYG{l+s+s1}{\PYGZsq{}}\PYG{l+s+s1}{z\PYGZhy{}position [µm]}\PYG{l+s+s1}{\PYGZsq{}}\PYG{p}{)}
\PYG{n}{plt}\PYG{o}{.}\PYG{n}{ylabel}\PYG{p}{(}\PYG{l+s+s1}{\PYGZsq{}}\PYG{l+s+s1}{x\PYGZhy{}position [µm]}\PYG{l+s+s1}{\PYGZsq{}}\PYG{p}{)}


\PYG{n}{plt}\PYG{o}{.}\PYG{n}{show}\PYG{p}{(}\PYG{p}{)}
\end{sphinxVerbatim}
}

\hrule height -\fboxrule\relax
\vspace{\nbsphinxcodecellspacing}

\makeatletter\setbox\nbsphinxpromptbox\box\voidb@x\makeatother

\begin{nbsphinxfancyoutput}

\noindent\sphinxincludegraphics[width=347\sphinxpxdimen,height=354\sphinxpxdimen]{{snippets_Wave_Explorer_16_0}.png}

\end{nbsphinxfancyoutput}


\subsubsection{Plane wave propagation}
\label{\detokenize{snippets/Wave Explorer:Plane-wave-propagation}}
The above graph shows a static snapshot of the plane wave at a time \(t=0\). We know, however, that a plane wave is propagating in space and time. Since we know how to animate things, we may do that using the \sphinxcode{\sphinxupquote{ipycanvas}} module.

{
\sphinxsetup{VerbatimColor={named}{nbsphinx-code-bg}}
\sphinxsetup{VerbatimBorderColor={named}{nbsphinx-code-border}}
\begin{sphinxVerbatim}[commandchars=\\\{\}]
\llap{\color{nbsphinxin}[8]:\,\hspace{\fboxrule}\hspace{\fboxsep}}\PYG{n}{x}\PYG{o}{=}\PYG{n}{np}\PYG{o}{.}\PYG{n}{linspace}\PYG{p}{(}\PYG{o}{\PYGZhy{}}\PYG{l+m+mf}{2.5e\PYGZhy{}6}\PYG{p}{,}\PYG{l+m+mf}{2.5e\PYGZhy{}6}\PYG{p}{,}\PYG{l+m+mi}{300}\PYG{p}{)}
\PYG{n}{z}\PYG{o}{=}\PYG{n}{np}\PYG{o}{.}\PYG{n}{linspace}\PYG{p}{(}\PYG{l+m+mi}{0}\PYG{p}{,}\PYG{l+m+mf}{5e\PYGZhy{}6}\PYG{p}{,}\PYG{l+m+mi}{300}\PYG{p}{)}

\PYG{n}{X}\PYG{p}{,}\PYG{n}{Z}\PYG{o}{=}\PYG{n}{np}\PYG{o}{.}\PYG{n}{meshgrid}\PYG{p}{(}\PYG{n}{x}\PYG{p}{,}\PYG{n}{z}\PYG{p}{)}
\PYG{n}{r}\PYG{o}{=}\PYG{n}{np}\PYG{o}{.}\PYG{n}{array}\PYG{p}{(}\PYG{p}{[}\PYG{n}{X}\PYG{p}{,}\PYG{l+m+mi}{0}\PYG{p}{,}\PYG{n}{Z}\PYG{p}{]}\PYG{p}{,}\PYG{n}{dtype}\PYG{o}{=}\PYG{n+nb}{object}\PYG{p}{)}
\end{sphinxVerbatim}
}

{
\sphinxsetup{VerbatimColor={named}{nbsphinx-code-bg}}
\sphinxsetup{VerbatimBorderColor={named}{nbsphinx-code-border}}
\begin{sphinxVerbatim}[commandchars=\\\{\}]
\llap{\color{nbsphinxin}[9]:\,\hspace{\fboxrule}\hspace{\fboxsep}}\PYG{n}{canvas} \PYG{o}{=} \PYG{n}{Canvas}\PYG{p}{(}\PYG{n}{width}\PYG{o}{=}\PYG{l+m+mi}{300}\PYG{p}{,} \PYG{n}{height}\PYG{o}{=}\PYG{l+m+mi}{300}\PYG{p}{,}\PYG{n}{sync\PYGZus{}image\PYGZus{}data}\PYG{o}{=}\PYG{k+kc}{True}\PYG{p}{)}
\PYG{n}{display}\PYG{p}{(}\PYG{n}{canvas}\PYG{p}{)}
\end{sphinxVerbatim}
}

{

\kern-\sphinxverbatimsmallskipamount\kern-\baselineskip
\kern+\FrameHeightAdjust\kern-\fboxrule
\vspace{\nbsphinxcodecellspacing}

\sphinxsetup{VerbatimColor={named}{white}}
\sphinxsetup{VerbatimBorderColor={named}{nbsphinx-code-border}}
\begin{sphinxVerbatim}[commandchars=\\\{\}]
Canvas(height=300, sync\_image\_data=True, width=300)
\end{sphinxVerbatim}
}

To do the animation I use a little trick to get the same color map as in the matplotlib plotting. The function below uses the matplotlib color map \sphinxcode{\sphinxupquote{seismic}} and the corresponding mapping of values with a given minimum \sphinxcode{\sphinxupquote{vmin}} and maximum \sphinxcode{\sphinxupquote{vmax}} value. The mapping is done in the animation function with \sphinxcode{\sphinxupquote{c=m.to\_rgba(tmp)}}.

{
\sphinxsetup{VerbatimColor={named}{nbsphinx-code-bg}}
\sphinxsetup{VerbatimBorderColor={named}{nbsphinx-code-border}}
\begin{sphinxVerbatim}[commandchars=\\\{\}]
\llap{\color{nbsphinxin}[10]:\,\hspace{\fboxrule}\hspace{\fboxsep}}\PYG{c+c1}{\PYGZsh{}normalize the color map to a certain value range}
\PYG{n}{norm} \PYG{o}{=} \PYG{n}{mpl}\PYG{o}{.}\PYG{n}{colors}\PYG{o}{.}\PYG{n}{Normalize}\PYG{p}{(}\PYG{n}{vmin}\PYG{o}{=}\PYG{o}{\PYGZhy{}}\PYG{l+m+mi}{1}\PYG{p}{,} \PYG{n}{vmax}\PYG{o}{=}\PYG{l+m+mi}{1}\PYG{p}{)}

\PYG{c+c1}{\PYGZsh{}call the color map}
\PYG{n}{cmap} \PYG{o}{=} \PYG{n}{cm}\PYG{o}{.}\PYG{n}{seismic}

\PYG{c+c1}{\PYGZsh{} do the mapping of values to color values.}
\PYG{n}{m} \PYG{o}{=} \PYG{n}{cm}\PYG{o}{.}\PYG{n}{ScalarMappable}\PYG{p}{(}\PYG{n}{norm}\PYG{o}{=}\PYG{n}{norm}\PYG{p}{,} \PYG{n}{cmap}\PYG{o}{=}\PYG{n}{cmap}\PYG{p}{)}
\end{sphinxVerbatim}
}

This is our animation function, where I provide time and the wavevector as arguments, such that we may change both parameters easily.

{
\sphinxsetup{VerbatimColor={named}{nbsphinx-code-bg}}
\sphinxsetup{VerbatimBorderColor={named}{nbsphinx-code-border}}
\begin{sphinxVerbatim}[commandchars=\\\{\}]
\llap{\color{nbsphinxin}[11]:\,\hspace{\fboxrule}\hspace{\fboxsep}}\PYG{k}{def} \PYG{n+nf}{animate}\PYG{p}{(}\PYG{n}{k}\PYG{p}{,}\PYG{n}{time}\PYG{p}{)}\PYG{p}{:}
    \PYG{k}{for} \PYG{n}{t} \PYG{o+ow}{in} \PYG{n}{time}\PYG{p}{:}
        \PYG{n}{field}\PYG{o}{=}\PYG{n}{plane\PYGZus{}wave}\PYG{p}{(}\PYG{n}{k}\PYG{p}{,}\PYG{n}{omega0}\PYG{p}{,}\PYG{n}{r}\PYG{p}{,}\PYG{n}{t}\PYG{p}{)}
        \PYG{n}{tmp}\PYG{o}{=}\PYG{n}{np}\PYG{o}{.}\PYG{n}{real}\PYG{p}{(}\PYG{n}{field}\PYG{o}{.}\PYG{n}{transpose}\PYG{p}{(}\PYG{p}{)}\PYG{p}{)}
        \PYG{n}{c}\PYG{o}{=}\PYG{n}{m}\PYG{o}{.}\PYG{n}{to\PYGZus{}rgba}\PYG{p}{(}\PYG{n}{tmp}\PYG{p}{)}
        \PYG{k}{with} \PYG{n}{hold\PYGZus{}canvas}\PYG{p}{(}\PYG{n}{canvas}\PYG{p}{)}\PYG{p}{:}
            \PYG{n}{canvas}\PYG{o}{.}\PYG{n}{put\PYGZus{}image\PYGZus{}data}\PYG{p}{(}\PYG{n}{c}\PYG{p}{[}\PYG{p}{:}\PYG{p}{,}\PYG{p}{:}\PYG{p}{,}\PYG{p}{:}\PYG{l+m+mi}{3}\PYG{p}{]}\PYG{o}{*}\PYG{l+m+mi}{255}\PYG{p}{,}\PYG{l+m+mi}{0}\PYG{p}{,}\PYG{l+m+mi}{0}\PYG{p}{)}
            \PYG{c+c1}{\PYGZsh{}canvas.put\PYGZus{}image\PYGZus{}data(data*255,0,0)}
        \PYG{n}{sleep}\PYG{p}{(}\PYG{l+m+mf}{0.02}\PYG{p}{)}
\end{sphinxVerbatim}
}

With the call below, you may animate the wave now with different refractive indices.

{
\sphinxsetup{VerbatimColor={named}{nbsphinx-code-bg}}
\sphinxsetup{VerbatimBorderColor={named}{nbsphinx-code-border}}
\begin{sphinxVerbatim}[commandchars=\\\{\}]
\llap{\color{nbsphinxin}[12]:\,\hspace{\fboxrule}\hspace{\fboxsep}}\PYG{n}{eta}\PYG{o}{=}\PYG{l+m+mi}{1}
\PYG{n}{kappa}\PYG{o}{=}\PYG{l+m+mi}{0}
\PYG{n}{n}\PYG{o}{=}\PYG{n}{eta}\PYG{o}{+}\PYG{n}{kappa}\PYG{o}{*}\PYG{l+m+mi}{1}\PYG{n}{j}

\PYG{n}{k}\PYG{o}{=}\PYG{n}{n}\PYG{o}{*}\PYG{n}{k0}\PYG{o}{*}\PYG{n}{vec}
\PYG{n}{time}\PYG{o}{=} \PYG{n}{np}\PYG{o}{.}\PYG{n}{linspace}\PYG{p}{(}\PYG{l+m+mi}{0}\PYG{p}{,}\PYG{l+m+mf}{5e\PYGZhy{}14}\PYG{p}{,}\PYG{l+m+mi}{500}\PYG{p}{)}
\PYG{n}{animate}\PYG{p}{(}\PYG{n}{k}\PYG{p}{,}\PYG{n}{time}\PYG{p}{)}
\end{sphinxVerbatim}
}


\begin{savenotes}\sphinxattablestart
\centering
\begin{tabulary}{\linewidth}[t]{|T|}
\hline
\sphinxstyletheadfamily 
\sphinxincludegraphics[width=0.500\linewidth]{{plane_prop1}.mov}
\\
\hline
\sphinxstylestrong{Fig.:} . Propagating spherical waves for positive and negative wavenumber.
\\
\hline
\end{tabulary}
\par
\sphinxattableend\end{savenotes}


\subsection{Spherical Waves}
\label{\detokenize{snippets/Wave Explorer:Spherical-Waves}}
A spherical wave is as well described by two exponentials containing the spatial and temporal dependence of the wave. The only difference is, that the wavefronts shall describe spheres instead of planes. We therefore need \(|\vec{k}||\vec{r}|=k r=const\). The product of the magntitudes of the wavevector and the distance from the source are constant. If we further generalize the position of the source to \(\vec{r}_{0}\) we can write a spherical wave by

\begin{equation}
U=\frac{A}{|\vec{r}-\vec{r}_{0}|}e^{-i k|\vec{r}-\vec{r}_{0}|} e^{i\omega t}
\end{equation}

Note that we have to introduce an additional scaling of the amplitude with the inverse distance of the source. This is due to energy conservation, as we require that all the energy that flows through all spheres around the source is constant.

{
\sphinxsetup{VerbatimColor={named}{nbsphinx-code-bg}}
\sphinxsetup{VerbatimBorderColor={named}{nbsphinx-code-border}}
\begin{sphinxVerbatim}[commandchars=\\\{\}]
\llap{\color{nbsphinxin}[13]:\,\hspace{\fboxrule}\hspace{\fboxsep}}\PYG{k}{def} \PYG{n+nf}{spherical\PYGZus{}wave}\PYG{p}{(}\PYG{n}{k}\PYG{p}{,}\PYG{n}{omega}\PYG{p}{,}\PYG{n}{r}\PYG{p}{,}\PYG{n}{r0}\PYG{p}{,}\PYG{n}{t}\PYG{p}{)}\PYG{p}{:}
    \PYG{n}{k}\PYG{o}{=}\PYG{n}{np}\PYG{o}{.}\PYG{n}{linalg}\PYG{o}{.}\PYG{n}{norm}\PYG{p}{(}\PYG{n}{k}\PYG{p}{)}\PYG{o}{*}\PYG{n}{np}\PYG{o}{.}\PYG{n}{sign}\PYG{p}{(}\PYG{n}{k}\PYG{p}{[}\PYG{l+m+mi}{2}\PYG{p}{]}\PYG{p}{)}
    \PYG{n}{d}\PYG{o}{=}\PYG{n}{np}\PYG{o}{.}\PYG{n}{linalg}\PYG{o}{.}\PYG{n}{norm}\PYG{p}{(}\PYG{n}{r}\PYG{o}{\PYGZhy{}}\PYG{n}{r0}\PYG{p}{)}
    \PYG{k}{return}\PYG{p}{(} \PYG{n}{np}\PYG{o}{.}\PYG{n}{exp}\PYG{p}{(}\PYG{l+m+mi}{1}\PYG{n}{j}\PYG{o}{*}\PYG{p}{(}\PYG{n}{k}\PYG{o}{*}\PYG{n}{d}\PYG{o}{\PYGZhy{}}\PYG{n}{omega}\PYG{o}{*}\PYG{n}{t}\PYG{p}{)}\PYG{p}{)}\PYG{o}{/}\PYG{n}{d}\PYG{p}{)}
\end{sphinxVerbatim}
}

Lets have a look at the electric field of the spherical wave. Below is some code plotting the electric field is space. The source is at the origin and the plot nicely shows, that the amplitude decays with the distance.

{
\sphinxsetup{VerbatimColor={named}{nbsphinx-code-bg}}
\sphinxsetup{VerbatimBorderColor={named}{nbsphinx-code-border}}
\begin{sphinxVerbatim}[commandchars=\\\{\}]
\llap{\color{nbsphinxin}[14]:\,\hspace{\fboxrule}\hspace{\fboxsep}}\PYG{n}{plt}\PYG{o}{.}\PYG{n}{figure}\PYG{p}{(}\PYG{n}{figsize}\PYG{o}{=}\PYG{p}{(}\PYG{l+m+mi}{5}\PYG{p}{,}\PYG{l+m+mi}{5}\PYG{p}{)}\PYG{p}{)}

\PYG{n}{x}\PYG{o}{=}\PYG{n}{np}\PYG{o}{.}\PYG{n}{linspace}\PYG{p}{(}\PYG{o}{\PYGZhy{}}\PYG{l+m+mf}{5e\PYGZhy{}6}\PYG{p}{,}\PYG{l+m+mf}{5e\PYGZhy{}6}\PYG{p}{,}\PYG{l+m+mi}{300}\PYG{p}{)}
\PYG{n}{z}\PYG{o}{=}\PYG{n}{np}\PYG{o}{.}\PYG{n}{linspace}\PYG{p}{(}\PYG{o}{\PYGZhy{}}\PYG{l+m+mf}{5e\PYGZhy{}6}\PYG{p}{,}\PYG{l+m+mf}{5e\PYGZhy{}6}\PYG{p}{,}\PYG{l+m+mi}{300}\PYG{p}{)}

\PYG{n}{X}\PYG{p}{,}\PYG{n}{Z}\PYG{o}{=}\PYG{n}{np}\PYG{o}{.}\PYG{n}{meshgrid}\PYG{p}{(}\PYG{n}{x}\PYG{p}{,}\PYG{n}{z}\PYG{p}{)}
\PYG{n}{r}\PYG{o}{=}\PYG{n}{np}\PYG{o}{.}\PYG{n}{array}\PYG{p}{(}\PYG{p}{[}\PYG{n}{X}\PYG{p}{,}\PYG{l+m+mi}{0}\PYG{p}{,}\PYG{n}{Z}\PYG{p}{]}\PYG{p}{,}\PYG{n}{dtype}\PYG{o}{=}\PYG{n+nb}{object}\PYG{p}{)}

\PYG{n}{wavelength}\PYG{o}{=}\PYG{l+m+mf}{532e\PYGZhy{}9}
\PYG{n}{k0}\PYG{o}{=}\PYG{l+m+mi}{2}\PYG{o}{*}\PYG{n}{np}\PYG{o}{.}\PYG{n}{pi}\PYG{o}{/}\PYG{n}{wavelength}
\PYG{n}{c}\PYG{o}{=}\PYG{l+m+mi}{299792458}
\PYG{n}{omega0}\PYG{o}{=}\PYG{n}{k0}\PYG{o}{*}\PYG{n}{c}

\PYG{n}{k}\PYG{o}{=}\PYG{n}{k0}\PYG{o}{*}\PYG{n}{np}\PYG{o}{.}\PYG{n}{array}\PYG{p}{(}\PYG{p}{[}\PYG{l+m+mi}{0}\PYG{p}{,}\PYG{l+m+mi}{0}\PYG{p}{,}\PYG{l+m+mf}{1.}\PYG{p}{]}\PYG{p}{)}
\PYG{n}{r0}\PYG{o}{=}\PYG{n}{np}\PYG{o}{.}\PYG{n}{array}\PYG{p}{(}\PYG{p}{[}\PYG{l+m+mi}{0}\PYG{p}{,}\PYG{l+m+mi}{0}\PYG{p}{,}\PYG{l+m+mi}{0}\PYG{p}{]}\PYG{p}{)}

\PYG{n}{field}\PYG{o}{=}\PYG{n}{spherical\PYGZus{}wave}\PYG{p}{(}\PYG{n}{k}\PYG{p}{,}\PYG{n}{omega0}\PYG{p}{,}\PYG{n}{r}\PYG{p}{,}\PYG{n}{r0}\PYG{p}{,}\PYG{l+m+mi}{0}\PYG{p}{)}

\PYG{n}{extent} \PYG{o}{=} \PYG{n}{np}\PYG{o}{.}\PYG{n}{min}\PYG{p}{(}\PYG{n}{z}\PYG{p}{)}\PYG{o}{*}\PYG{l+m+mf}{1e6}\PYG{p}{,} \PYG{n}{np}\PYG{o}{.}\PYG{n}{max}\PYG{p}{(}\PYG{n}{z}\PYG{p}{)}\PYG{o}{*}\PYG{l+m+mf}{1e6}\PYG{p}{,}\PYG{n}{np}\PYG{o}{.}\PYG{n}{min}\PYG{p}{(}\PYG{n}{x}\PYG{p}{)}\PYG{o}{*}\PYG{l+m+mf}{1e6}\PYG{p}{,} \PYG{n}{np}\PYG{o}{.}\PYG{n}{max}\PYG{p}{(}\PYG{n}{x}\PYG{p}{)}\PYG{o}{*}\PYG{l+m+mf}{1e6}
\PYG{n}{plt}\PYG{o}{.}\PYG{n}{imshow}\PYG{p}{(}\PYG{n}{np}\PYG{o}{.}\PYG{n}{real}\PYG{p}{(}\PYG{n}{field}\PYG{o}{.}\PYG{n}{transpose}\PYG{p}{(}\PYG{p}{)}\PYG{p}{)}\PYG{p}{,}\PYG{n}{extent}\PYG{o}{=}\PYG{n}{extent}\PYG{p}{,}\PYG{n}{vmin}\PYG{o}{=}\PYG{o}{\PYGZhy{}}\PYG{l+m+mf}{5e6}\PYG{p}{,}\PYG{n}{vmax}\PYG{o}{=}\PYG{l+m+mf}{5e6}\PYG{p}{,}\PYG{n}{cmap}\PYG{o}{=}\PYG{l+s+s1}{\PYGZsq{}}\PYG{l+s+s1}{seismic}\PYG{l+s+s1}{\PYGZsq{}}\PYG{p}{)}

\PYG{n}{plt}\PYG{o}{.}\PYG{n}{xlabel}\PYG{p}{(}\PYG{l+s+s1}{\PYGZsq{}}\PYG{l+s+s1}{z [µm]}\PYG{l+s+s1}{\PYGZsq{}}\PYG{p}{)}
\PYG{n}{plt}\PYG{o}{.}\PYG{n}{ylabel}\PYG{p}{(}\PYG{l+s+s1}{\PYGZsq{}}\PYG{l+s+s1}{x [µm]}\PYG{l+s+s1}{\PYGZsq{}}\PYG{p}{)}
\PYG{n}{plt}\PYG{o}{.}\PYG{n}{show}\PYG{p}{(}\PYG{p}{)}
\end{sphinxVerbatim}
}

\hrule height -\fboxrule\relax
\vspace{\nbsphinxcodecellspacing}

\makeatletter\setbox\nbsphinxpromptbox\box\voidb@x\makeatother

\begin{nbsphinxfancyoutput}

\noindent\sphinxincludegraphics[width=342\sphinxpxdimen,height=334\sphinxpxdimen]{{snippets_Wave_Explorer_32_0}.png}

\end{nbsphinxfancyoutput}

The line plots below show that the field amplitude rapidly decays and the intensity follows a \(1/r^2\) law as expected. The slight deiviation at small distances is an artifact from our discretization. We used the image above to extract the line plot and therefore never exactly hit \(r=0\).

{
\sphinxsetup{VerbatimColor={named}{nbsphinx-code-bg}}
\sphinxsetup{VerbatimBorderColor={named}{nbsphinx-code-border}}
\begin{sphinxVerbatim}[commandchars=\\\{\}]
\llap{\color{nbsphinxin}[15]:\,\hspace{\fboxrule}\hspace{\fboxsep}}\PYG{n}{plt}\PYG{o}{.}\PYG{n}{figure}\PYG{p}{(}\PYG{n}{figsize}\PYG{o}{=}\PYG{p}{(}\PYG{l+m+mi}{12}\PYG{p}{,}\PYG{l+m+mi}{5}\PYG{p}{)}\PYG{p}{)}
\PYG{n}{plt}\PYG{o}{.}\PYG{n}{subplot}\PYG{p}{(}\PYG{l+m+mi}{1}\PYG{p}{,}\PYG{l+m+mi}{2}\PYG{p}{,}\PYG{l+m+mi}{1}\PYG{p}{)}
\PYG{n}{plt}\PYG{o}{.}\PYG{n}{plot}\PYG{p}{(}\PYG{n}{z}\PYG{o}{*}\PYG{l+m+mf}{1e6}\PYG{p}{,}\PYG{n}{np}\PYG{o}{.}\PYG{n}{real}\PYG{p}{(}\PYG{n}{field}\PYG{o}{.}\PYG{n}{transpose}\PYG{p}{(}\PYG{p}{)}\PYG{p}{[}\PYG{l+m+mi}{150}\PYG{p}{,}\PYG{p}{:}\PYG{p}{]}\PYG{p}{)}\PYG{p}{)}
\PYG{n}{plt}\PYG{o}{.}\PYG{n}{xlabel}\PYG{p}{(}\PYG{l+s+s1}{\PYGZsq{}}\PYG{l+s+s1}{z in [µm]}\PYG{l+s+s1}{\PYGZsq{}}\PYG{p}{)}
\PYG{n}{plt}\PYG{o}{.}\PYG{n}{ylabel}\PYG{p}{(}\PYG{l+s+s1}{\PYGZsq{}}\PYG{l+s+s1}{electric field [a.u.]}\PYG{l+s+s1}{\PYGZsq{}}\PYG{p}{)}


\PYG{n}{plt}\PYG{o}{.}\PYG{n}{subplot}\PYG{p}{(}\PYG{l+m+mi}{1}\PYG{p}{,}\PYG{l+m+mi}{2}\PYG{p}{,}\PYG{l+m+mi}{2}\PYG{p}{)}
\PYG{n}{plt}\PYG{o}{.}\PYG{n}{loglog}\PYG{p}{(}\PYG{n}{z}\PYG{o}{*}\PYG{l+m+mf}{1e6}\PYG{p}{,}\PYG{l+m+mi}{1}\PYG{o}{/}\PYG{p}{(}\PYG{n}{z}\PYG{o}{*}\PYG{o}{*}\PYG{l+m+mi}{2}\PYG{p}{)}\PYG{p}{,}\PYG{l+s+s1}{\PYGZsq{}}\PYG{l+s+s1}{k\PYGZhy{}\PYGZhy{}}\PYG{l+s+s1}{\PYGZsq{}}\PYG{p}{,}\PYG{n}{label}\PYG{o}{=}\PYG{l+s+s1}{\PYGZsq{}}\PYG{l+s+s1}{\PYGZdl{}1/r\PYGZca{}2\PYGZdl{}}\PYG{l+s+s1}{\PYGZsq{}}\PYG{p}{)}
\PYG{n}{plt}\PYG{o}{.}\PYG{n}{loglog}\PYG{p}{(}\PYG{n}{z}\PYG{o}{*}\PYG{l+m+mf}{1e6}\PYG{p}{,}\PYG{n}{np}\PYG{o}{.}\PYG{n}{abs}\PYG{p}{(}\PYG{n}{field}\PYG{o}{.}\PYG{n}{transpose}\PYG{p}{(}\PYG{p}{)}\PYG{p}{[}\PYG{l+m+mi}{150}\PYG{p}{,}\PYG{p}{:}\PYG{p}{]}\PYG{p}{)}\PYG{o}{*}\PYG{o}{*}\PYG{l+m+mi}{2}\PYG{p}{,}\PYG{n}{color}\PYG{o}{=}\PYG{l+s+s1}{\PYGZsq{}}\PYG{l+s+s1}{k}\PYG{l+s+s1}{\PYGZsq{}}\PYG{p}{,}\PYG{n}{alpha}\PYG{o}{=}\PYG{l+m+mf}{0.2}\PYG{p}{,}\PYG{n}{lw}\PYG{o}{=}\PYG{l+m+mi}{4}\PYG{p}{,}\PYG{n}{label}\PYG{o}{=}\PYG{l+s+s1}{\PYGZsq{}}\PYG{l+s+s1}{intensity}\PYG{l+s+s1}{\PYGZsq{}}\PYG{p}{)}
\PYG{n}{plt}\PYG{o}{.}\PYG{n}{xlabel}\PYG{p}{(}\PYG{l+s+s1}{\PYGZsq{}}\PYG{l+s+s1}{z in [µm]}\PYG{l+s+s1}{\PYGZsq{}}\PYG{p}{)}
\PYG{n}{plt}\PYG{o}{.}\PYG{n}{xlim}\PYG{p}{(}\PYG{l+m+mf}{2e\PYGZhy{}2}\PYG{p}{,}\PYG{p}{)}
\PYG{n}{plt}\PYG{o}{.}\PYG{n}{ylabel}\PYG{p}{(}\PYG{l+s+s1}{\PYGZsq{}}\PYG{l+s+s1}{intensity [a.u.]}\PYG{l+s+s1}{\PYGZsq{}}\PYG{p}{)}
\PYG{n}{plt}\PYG{o}{.}\PYG{n}{legend}\PYG{p}{(}\PYG{p}{)}
\PYG{n}{plt}\PYG{o}{.}\PYG{n}{tight\PYGZus{}layout}\PYG{p}{(}\PYG{p}{)}
\PYG{n}{plt}\PYG{o}{.}\PYG{n}{show}\PYG{p}{(}\PYG{p}{)}
\end{sphinxVerbatim}
}

\hrule height -\fboxrule\relax
\vspace{\nbsphinxcodecellspacing}

\makeatletter\setbox\nbsphinxpromptbox\box\voidb@x\makeatother

\begin{nbsphinxfancyoutput}

\noindent\sphinxincludegraphics[width=853\sphinxpxdimen,height=347\sphinxpxdimen]{{snippets_Wave_Explorer_34_0}.png}

\end{nbsphinxfancyoutput}


\subsubsection{Spherical wave propagation}
\label{\detokenize{snippets/Wave Explorer:Spherical-wave-propagation}}
We can also visualize the animation our spherical wave to check for the direction of the wave propagation.

{
\sphinxsetup{VerbatimColor={named}{nbsphinx-code-bg}}
\sphinxsetup{VerbatimBorderColor={named}{nbsphinx-code-border}}
\begin{sphinxVerbatim}[commandchars=\\\{\}]
\llap{\color{nbsphinxin}[16]:\,\hspace{\fboxrule}\hspace{\fboxsep}}\PYG{n}{norm} \PYG{o}{=} \PYG{n}{mpl}\PYG{o}{.}\PYG{n}{colors}\PYG{o}{.}\PYG{n}{Normalize}\PYG{p}{(}\PYG{n}{vmin}\PYG{o}{=}\PYG{o}{\PYGZhy{}}\PYG{l+m+mf}{5e6}\PYG{p}{,} \PYG{n}{vmax}\PYG{o}{=}\PYG{l+m+mf}{5e6}\PYG{p}{)}
\PYG{n}{cmap} \PYG{o}{=} \PYG{n}{cm}\PYG{o}{.}\PYG{n}{seismic}
\PYG{n}{m} \PYG{o}{=} \PYG{n}{cm}\PYG{o}{.}\PYG{n}{ScalarMappable}\PYG{p}{(}\PYG{n}{norm}\PYG{o}{=}\PYG{n}{norm}\PYG{p}{,} \PYG{n}{cmap}\PYG{o}{=}\PYG{n}{cmap}\PYG{p}{)}
\end{sphinxVerbatim}
}

{
\sphinxsetup{VerbatimColor={named}{nbsphinx-code-bg}}
\sphinxsetup{VerbatimBorderColor={named}{nbsphinx-code-border}}
\begin{sphinxVerbatim}[commandchars=\\\{\}]
\llap{\color{nbsphinxin}[17]:\,\hspace{\fboxrule}\hspace{\fboxsep}}\PYG{n}{canvas} \PYG{o}{=} \PYG{n}{Canvas}\PYG{p}{(}\PYG{n}{width}\PYG{o}{=}\PYG{l+m+mi}{300}\PYG{p}{,} \PYG{n}{height}\PYG{o}{=}\PYG{l+m+mi}{300}\PYG{p}{,}\PYG{n}{sync\PYGZus{}image\PYGZus{}data}\PYG{o}{=}\PYG{k+kc}{True}\PYG{p}{)}
\PYG{n}{display}\PYG{p}{(}\PYG{n}{canvas}\PYG{p}{)}
\end{sphinxVerbatim}
}

{

\kern-\sphinxverbatimsmallskipamount\kern-\baselineskip
\kern+\FrameHeightAdjust\kern-\fboxrule
\vspace{\nbsphinxcodecellspacing}

\sphinxsetup{VerbatimColor={named}{white}}
\sphinxsetup{VerbatimBorderColor={named}{nbsphinx-code-border}}
\begin{sphinxVerbatim}[commandchars=\\\{\}]
Canvas(height=300, sync\_image\_data=True, width=300)
\end{sphinxVerbatim}
}

{
\sphinxsetup{VerbatimColor={named}{nbsphinx-code-bg}}
\sphinxsetup{VerbatimBorderColor={named}{nbsphinx-code-border}}
\begin{sphinxVerbatim}[commandchars=\\\{\}]
\llap{\color{nbsphinxin}[18]:\,\hspace{\fboxrule}\hspace{\fboxsep}}\PYG{k}{def} \PYG{n+nf}{animate}\PYG{p}{(}\PYG{n}{k}\PYG{p}{,}\PYG{n}{time}\PYG{p}{)}\PYG{p}{:}
    \PYG{k}{for} \PYG{n}{t} \PYG{o+ow}{in} \PYG{n}{time}\PYG{p}{:}
        \PYG{n}{field}\PYG{o}{=}\PYG{n}{spherical\PYGZus{}wave}\PYG{p}{(}\PYG{n}{k}\PYG{p}{,}\PYG{n}{omega0}\PYG{p}{,}\PYG{n}{r}\PYG{p}{,}\PYG{n}{r0}\PYG{p}{,}\PYG{n}{t}\PYG{p}{)}
        \PYG{n}{data}\PYG{o}{=}\PYG{n}{np}\PYG{o}{.}\PYG{n}{zeros}\PYG{p}{(}\PYG{p}{[}\PYG{l+m+mi}{300}\PYG{p}{,}\PYG{l+m+mi}{300}\PYG{p}{,}\PYG{l+m+mi}{3}\PYG{p}{]}\PYG{p}{)}
        \PYG{n}{tmp}\PYG{o}{=}\PYG{n}{np}\PYG{o}{.}\PYG{n}{real}\PYG{p}{(}\PYG{n}{field}\PYG{o}{.}\PYG{n}{transpose}\PYG{p}{(}\PYG{p}{)}\PYG{p}{)}
        \PYG{n}{c}\PYG{o}{=}\PYG{n}{m}\PYG{o}{.}\PYG{n}{to\PYGZus{}rgba}\PYG{p}{(}\PYG{n}{tmp}\PYG{p}{)}
        \PYG{k}{with} \PYG{n}{hold\PYGZus{}canvas}\PYG{p}{(}\PYG{n}{canvas}\PYG{p}{)}\PYG{p}{:}
            \PYG{n}{canvas}\PYG{o}{.}\PYG{n}{put\PYGZus{}image\PYGZus{}data}\PYG{p}{(}\PYG{n}{c}\PYG{p}{[}\PYG{p}{:}\PYG{p}{,}\PYG{p}{:}\PYG{p}{,}\PYG{p}{:}\PYG{l+m+mi}{3}\PYG{p}{]}\PYG{o}{*}\PYG{l+m+mi}{255}\PYG{p}{,}\PYG{l+m+mi}{0}\PYG{p}{,}\PYG{l+m+mi}{0}\PYG{p}{)}
        \PYG{n}{sleep}\PYG{p}{(}\PYG{l+m+mf}{0.02}\PYG{p}{)}
\end{sphinxVerbatim}
}

{
\sphinxsetup{VerbatimColor={named}{nbsphinx-code-bg}}
\sphinxsetup{VerbatimBorderColor={named}{nbsphinx-code-border}}
\begin{sphinxVerbatim}[commandchars=\\\{\}]
\llap{\color{nbsphinxin}[19]:\,\hspace{\fboxrule}\hspace{\fboxsep}}\PYG{n}{time}\PYG{o}{=} \PYG{n}{np}\PYG{o}{.}\PYG{n}{linspace}\PYG{p}{(}\PYG{l+m+mi}{0}\PYG{p}{,}\PYG{l+m+mf}{2e\PYGZhy{}14}\PYG{p}{,}\PYG{l+m+mi}{400}\PYG{p}{)}
\PYG{n}{animate}\PYG{p}{(}\PYG{n}{k}\PYG{p}{,}\PYG{n}{time}\PYG{p}{)}
\end{sphinxVerbatim}
}


\begin{savenotes}\sphinxattablestart
\centering
\begin{tabulary}{\linewidth}[t]{|T|}
\hline
\sphinxstyletheadfamily 
\sphinxincludegraphics[width=0.400\linewidth]{{out1}.mov} \sphinxincludegraphics[width=0.400\linewidth]{{in1}.mov}
\\
\hline
\sphinxstylestrong{Fig.:} . Propagating spherical waves for positive and negative wavenumber.
\\
\hline
\end{tabulary}
\par
\sphinxattableend\end{savenotes}


\section{Lecture Contents}
\label{\detokenize{lectures/L9/overview_9:lecture-contents}}\label{\detokenize{lectures/L9/overview_9::doc}}
In Lecture 9 we continue to explore different interference phenomena including different interferometers and thin film interference.

\noindent\sphinxincludegraphics[width=600\sphinxpxdimen]{{slides12}.png}

Lecture 9 slides for download \sphinxcode{\sphinxupquote{pdf}}

The following section was created from \sphinxcode{\sphinxupquote{source/notebooks/L9/Interference.ipynb}}.


\section{Coherence}
\label{\detokenize{notebooks/L9/Interference:Coherence}}\label{\detokenize{notebooks/L9/Interference::doc}}


One of the important properties e have assumed in the calculation of interference during the last lecture has been coherence. Coherence denotes the phase relation between two waves.


\begin{savenotes}\sphinxattablestart
\centering
\begin{tabulary}{\linewidth}[t]{|T|}
\hline
\sphinxstyletheadfamily 
\sphinxincludegraphics[width=0.900\linewidth]{{coherence}.png}
\\
\hline
\sphinxstylestrong{Fig.:} Two waves of different frequency over time.
\\
\hline
\end{tabulary}
\par
\sphinxattableend\end{savenotes}

The above image shows the timetrace of the amplitude of two wave with slightly different frequency. Due to the frequency, the waves run out of phase and have acquired a phase different of \(\pi\) after \(40\) fs.

The temporal coherence of two waves is now defined by the time it takes for the two waves to obtain a phase difference of \(2\pi\). The phase difference between two wave of frequency \(\nu_1\) and \(\nu_2\) is given by
\begin{equation*}
\begin{split}\Delta \phi = 2\pi (\nu_2-\nu_1)(t-t_0)\end{split}
\end{equation*}
Here \(t_0\) refers to the time, when thw two waves were perfectly in sync. Lets assume that the two frequencies are seperarated from a central frequency \(\nu_0\) such that
\begin{equation*}
\begin{split}\nu_1=\nu_0-\Delta \nu/2\end{split}
\end{equation*}\begin{equation*}
\begin{split}\nu_2=\nu_0+\Delta \nu/2\end{split}
\end{equation*}
Inserting this into the first equation yields
\begin{equation*}
\begin{split}\Delta \phi = 2\pi \Delta \nu \Delta t\end{split}
\end{equation*}
with \(\Delta t=t-t_0\). We can now define the coherence time as the time interval over which the phase shift \(\Delta \phi\) grows to \(2\pi\), i.e. \(\Delta \phi=2\pi\). The coherence time is thus
\begin{equation*}
\begin{split}\tau_{c}=\Delta t =\frac{1}{\Delta \nu}\end{split}
\end{equation*}
Thus the temporal coherence and the frequency distribution of the light are intrisincly connected. Monochromatic light has \(\Delta nu=0\) and thus the coherence time is infinitely long. Light with a wide spectrum (white light for example) therefore has and extremly short coherence time.

The coherence time is also connected to a coherence length. The coherence length \(L_c\) is given by the distance light travels within the coherence time \(\tau_c\), i.e.
\begin{equation*}
\begin{split}L_c=c\tau_c\end{split}
\end{equation*}
\begin{sphinxadmonition}{note}{}\unskip
\sphinxstylestrong{Coherence}

Two waves are called coherent, if they exihibit a fixed phase relation over time. A constant phase relation allows the observation of interference effects.

The coherence time \(\tau_c\) is given by
\begin{equation*}
\begin{split}\tau_{c}=\Delta t =\frac{1}{\Delta \nu}\end{split}
\end{equation*}
The coherence length \(L_c\) is given by
\begin{equation*}
\begin{split}L_c=c\tau_c\end{split}
\end{equation*}\end{sphinxadmonition}


\section{Interferometers}
\label{\detokenize{notebooks/L9/Interference:Interferometers}}
Interferometers are measurement devices that slit a light wave into two waves, put them on different path to reunite them later. Due to the short wavelength in the optical range, interference is an extremely senstive tool to measures small differences in the optical path length. This is used in a number of application culminating in the detection of gravitational waves on large length scales and single molecules on the other extreme. The examples below all use the \sphinxstylestrong{interference of two partial
waves}.


\subsection{Michelson Interferometer}
\label{\detokenize{notebooks/L9/Interference:Michelson-Interferometer}}
The Michelson interferometer is probably the most famous one due to the experiments by Michelson on the existance of th ether (a medium in which light was supposed to propagate) and nowadays also by the LIGO experiments for the detection of gravitational waves.The Michelson interferometer splits in incoming light wave with a beamsplitter (BS) into two equally intense waves that are directed towards two mirrors (M).


\begin{savenotes}\sphinxattablestart
\centering
\begin{tabulary}{\linewidth}[t]{|T|}
\hline
\sphinxstyletheadfamily 
\sphinxincludegraphics[width=0.400\linewidth]{{michelson}.png}
\\
\hline
\sphinxstylestrong{Fig.:} Michelson interferometer.
\\
\hline
\end{tabulary}
\par
\sphinxattableend\end{savenotes}

The light is reflected by the mirrors and joined again at the beamsplitter to propagate towards the screen. If the incident wave is a perfewct plane wave and the interferometer is aligned perfectly with both arms having the same length, you will observe a bright screen, which, when one are is elongated by half a wavelength is turin dark to show destructive interference. Yet, this perfect alignment does not exist neither a perfect plane wave. Therefore the pattern observed at the screen is either
a stripe or ring pattern. Note that if the screen would be completely dark due to desctructive interference, all light would be reflected back towards the laser.


\subsection{Mach\sphinxhyphen{}Zehnder Interferometer}
\label{\detokenize{notebooks/L9/Interference:Mach-Zehnder-Interferometer}}
Another important interferometer is the Mach Zehnder interferometer, which is sketched below. There two beamsplitters separate and unite the beams which finally end on two screens. The screens do both display interference patterns, which are complementary, meaning that the regions where the screens are bright on one side are dark on the other screen. This type of interfereometer is often used in integrated circuits and quantum information technology.


\begin{savenotes}\sphinxattablestart
\centering
\begin{tabulary}{\linewidth}[t]{|T|}
\hline
\sphinxstyletheadfamily 
\sphinxincludegraphics[width=0.400\linewidth]{{mach}.png}
\\
\hline
\sphinxstylestrong{Fig.:} Mach Zehnder interferometer.
\\
\hline
\end{tabulary}
\par
\sphinxattableend\end{savenotes}



\sphinxincludegraphics[width=0.320\linewidth]{{machz_lecture}.png} \sphinxincludegraphics[width=0.320\linewidth]{{machz_lecture_pat}.png} \sphinxincludegraphics[width=0.320\linewidth]{{machz_lecture_vortex}.png}



\sphinxstylestrong{Fig.:} Mach Zehnder interferometer used in the lecture. The \sphinxstylestrong{left} image shows the general layout (compare with sketch). The \sphinxstylestrong{middle} image shows the interference pattern obtained. Note that the two interference patterns are complementary. The pattern is circular as the incident wave is not a plane wave, but a Gaussian beam. The \sphinxstylestrong{right} image show a vortex like pattern, if a special waveplate is inserted that gives a special phase delay.






\subsection{Sagnac Interferometer}
\label{\detokenize{notebooks/L9/Interference:Sagnac-Interferometer}}
The Sagnac interferometer is a special type of interferometer which uses a clockwise and counter\sphinxhyphen{}clockwise propagating beam in a system as shown below.


\begin{savenotes}\sphinxattablestart
\centering
\begin{tabulary}{\linewidth}[t]{|T|}
\hline
\sphinxstyletheadfamily 
\sphinxincludegraphics[width=0.500\linewidth]{{sagnac}.png}
\\
\hline
\sphinxstylestrong{Fig.:} Sagnac interferometer.
\\
\hline
\end{tabulary}
\par
\sphinxattableend\end{savenotes}

The two counter propagating rotation directions of light propagation allow us to rotate at the frequency \(\Omega\) the interferometer around an axis perpendicular on the interferometer plane. In this case both light waves travel different distance.

The distance is either longer or shorter by
\begin{equation*}
\begin{split}dx=\Omega r dt\end{split}
\end{equation*}
if we consider a short time intervall \(dt\) and a circular interferometer with a radius \(r\). The time interval is given by \(dt=dl/c\). To obtain the path length we have to intergrate over one round
\begin{equation*}
\begin{split}x=\int dx = \frac{\omega}{c}2A\end{split}
\end{equation*}
where \(A\) corresponds to the area encircled by the light path on a radius \(r\).

The final phase shift of the two wave propagating in the different direction is then
\begin{equation*}
\begin{split}\Delta \phi =\frac{2\pi}{\lambda}2 x=\frac{8\pi}{\lambda c}\Omega\end{split}
\end{equation*}
If the axis of rotation has now the angle \(\theta\) with the normal to the plane \(A\), the phase shift becomes
\begin{equation*}
\begin{split}\Delta \phi =\frac{2\pi}{\lambda}2 x=\frac{8\pi}{\lambda c}\Omega\end{split}
\end{equation*}
Thus the Sagnac interferometer can be used as a gyroscope. This is actually its main application, for example in military airplanes.


\section{Double Slit Interference}
\label{\detokenize{notebooks/L9/Interference:Double-Slit-Interference}}
The double slit is also an example for a two\sphinxhyphen{}wave interference. We consider two tiny slits in a screen, which lets two waves pass. The two waves are of course coherent and in phase right at the slit. The transmitted waves can travel in all directions. The two arrows in the sketch below, demnote two propagation directions and the correponding wavefronts. The two arrows thus represent two wavevectors. The overall situation is actually a bit more complicated and we will address this with the help
of spherical waves later in the section about Huygens waves.


\begin{savenotes}\sphinxattablestart
\centering
\begin{tabulary}{\linewidth}[t]{|T|}
\hline
\sphinxstyletheadfamily 
\sphinxincludegraphics[width=400\sphinxpxdimen]{{double_slit}.png}
\\
\hline
\sphinxstylestrong{Fig.:} Double Slit .
\\
\hline
\end{tabulary}
\par
\sphinxattableend\end{savenotes}

When you consider the image, you recognize that the light from the lower slit has to travel a distance \(\Delta s\) farther as compared to the wave from the upper slit. This path length difference is given by
\begin{equation*}
\begin{split}\Delta s =d\sin(\theta)\end{split}
\end{equation*}
where \(d\) is the distance between the two slits and \(\theta\) the angle the two directions make with the horizontal direction.

We can now decide under which conditions we observe constructure or destructive interference.

\sphinxstylestrong{constructive interference}
\begin{equation*}
\begin{split}d\sin(\theta)=m\lambda, m=0,1,\ldots\end{split}
\end{equation*}
For constructive interference the waves have to have a path length difference, which is an integer multiple \(m\) of the wavelength.

\sphinxstylestrong{destructive interference}
\begin{equation*}
\begin{split}d\sin(\theta)=\left (m+\frac{1}{2}\right )\lambda, m=0,1,\ldots\end{split}
\end{equation*}
For destructive interference the wave have a pathlength difference of an odd multiple of half the wavelength as given above.

Having the angles for the constructive and destructive interference on the screen, we would also like to have the intensity distribution, which we can calculate from two spherical waves

The total amplitude of the two waves is given by
\begin{equation*}
\begin{split}U=U_1+ U_2 = A e^{-ikr}+A e^{-ikr+i\phi}\end{split}
\end{equation*}
For simplicity we have given thw two waves the same amplitude \(A\). Also the distance dependence of the amplitude of a spherical wave is put into the amplitude so that we don’t have to write it explicity every time.

The intensity is then the magnitude square if the total amplitude
\begin{equation*}
\begin{split}I=|U|^2=4I_{0}\cos^2(\phi/2)\end{split}
\end{equation*}
Given that we know the pathlength difference \(\Delta s\) we can also express the intensity as a function of the angle \(\theta\) at which we observe the interference on the screen.
\begin{equation*}
\begin{split}I=4I_0 \cos^2\left ( \frac{k \Delta s}{2}\right)=4I_0 \cos^2\left ( \frac{\pi d \sin(\theta)}{\lambda}\right)\end{split}
\end{equation*}
The two graphs below show the intensity as a function of \(\sin(\theta)\), for two different doubles slits at different distance.



\sphinxincludegraphics[width=800\sphinxpxdimen]{{double_slit_int}.png}



\sphinxstylestrong{Fig.:} Double slit interference pattern for two different wavelength (532 nm, 700 nm) and two different slit separations (left: 3 µm, right: 5 µm)





The larger the slit separation \(d\), the closer are the neighboring peaks for a single color. Also if we shorten the wavelength \(\lambda\), the individual peaks of constructive interference come closer.


\section{Lecture Contents}
\label{\detokenize{lectures/L10/overview_10:lecture-contents}}\label{\detokenize{lectures/L10/overview_10::doc}}
In Lecture 10 we look at multiple wave interference, anit\sphinxhyphen{}reflection coatings and explore the diffraction of light including Huygens principle.

\noindent\sphinxincludegraphics[width=600\sphinxpxdimen]{{slides1}.png}

Lecture 10 slides for download \sphinxcode{\sphinxupquote{pdf}}

The following section was created from \sphinxcode{\sphinxupquote{source/notebooks/L10/Interference.ipynb}}.


\section{Thin Film Interference}
\label{\detokenize{notebooks/L10/Interference:Thin-Film-Interference}}\label{\detokenize{notebooks/L10/Interference::doc}}


The reflection and transmission of waves on a thin film can also be regarded as an interference of two waves. A light wave is incident on a thin film as depicted below. A part of the wave is reflected on the first boundary (1). Another part is transmitted through the first boundary and reflected at the second boundary to be transmitted in the same direction (2) as the first reflected part. Note that the lines and arrows denote the direction of the wavevector \(\vec{k}\) of the partial waves.


\begin{savenotes}\sphinxattablestart
\centering
\begin{tabulary}{\linewidth}[t]{|T|}
\hline
\sphinxstyletheadfamily 
\sphinxincludegraphics[width=0.500\linewidth]{{thin_film}.png}
\\
\hline
\sphinxstylestrong{Fig.:} Interference on a thin film considering two partial waves.
\\
\hline
\end{tabulary}
\par
\sphinxattableend\end{savenotes}

This picture of a single reflection at each of the interfaces is certainly a simplification and in general we would have to consider an infinite number of reflections. But if we assume that the reflection is weak (e.g. 4\% as for the air/glass interface), then the next reflection would just diminish the 4\% to 4\% of 4\%, which is really weak. So the two wave interference may be a good model for weak reflections.

For the geometry shown in the figure above, we have just have a medium with \(n_1\) surrounding the film with \(n_2\). For this situation, we may calculate the path difference \(\Delta s\), the waves 1 and 2 have to travel.

We find
\begin{equation*}
\begin{split}\Delta s=\frac{2nd}{\cos(\beta)}-2d\tan(\beta)\sin(\alpha)\end{split}
\end{equation*}
where the first term contains the distance traveled by wave 2 inside the water film. The second term is the additional distance wave 1 has to travel after being reflected which is indicated by the dotted line inbetween.

We can now introduce Snells law into the calculation, as Snells law must be also valid in wave optics and also in electromagnetic optics. With \(n_1\sin(\alpha)=n_2\sin(\beta)\) and setting \(n_1=1\) and \(n_2=n\) we can rewrite the path difference as
\begin{equation*}
\begin{split}\Delta s =\frac{2nd}{\cos(\beta)}-\frac{2nd\sin^2(\beta)}{\cos(\beta)}=2n d \cos(\beta)=2d\sqrt{n^2-\sin^2(\alpha)}\end{split}
\end{equation*}
With that we can now also calculate the phase shift \(\Delta \phi\) that is introduced through the path difference, which reads
\begin{equation*}
\begin{split}\Delta \phi=\frac{2\pi}{\lambda}\Delta s +\pi\end{split}
\end{equation*}
You may now wonder about the second part \(\pi\), which is not coming from a path difference. This addition phase shift occurs due to the reflection at an interface with higher refractive index. For such a reflection, there is always a phase jump of \(\phi\), which we have to consider. This is not occuring for the reflection at the second boundary, as we there go from a higher refractive index to a lower one.

\begin{sphinxadmonition}{note}{}\unskip
\sphinxstylestrong{Phase Jump at Boundaries}

Wave may experience phase jumps when being reflected.

A light wave will experience a phase jump of \(\pi\) when being reflected by a medium of higher refractive index.

A light wave will experience no phase jump when being reflected by a medium of lower refractive index.

The physical reasons will be covered when we deal with the Fresnel formulas in electromagnetic optics.
\end{sphinxadmonition}

To get to know the properties of thin film interference a bit better we consider the normal incidence \(\alpha=0\), which leaves us with
\begin{equation*}
\begin{split}\Delta \phi=\frac{2\pi}{\lambda}2dn+\pi\end{split}
\end{equation*}
In case we are searching for constructive interference, this phase shift should correspond to an integer multiple of \(2\pi\), e.g. \(\Delta \phi =m2\pi\). From the last equation we see already, that for \(d=0\), we have in principle a residual phase shift of \(\pi\), meaning that there is only destructive interference. Yet a film thickness of zeor does not really make sense.

We would like to discuss two different situations in the following in an example. For that we either look at the thickness under which a constructive interference at a wavelength of \(\lambda\) occurs, or we ask what kind of wavelength do show constructive interference for a fixed thickness.

\sphinxstylestrong{Fixed Wavelength}

For a fixed wavelength of \(\lambda\) we obtain a corresponding thickness for the constructive interference of
\begin{equation*}
\begin{split}d=\frac{(2m-1)\lambda}{4n}\end{split}
\end{equation*}
\sphinxstylestrong{Fixed Thickness}

For a fixed thickness of \(d\) we obtain constructive interference at
\begin{equation*}
\begin{split}\lambda_{max}=\frac{4nd}{2m-1}\end{split}
\end{equation*}
We can now have a look at two examples.

\sphinxstylestrong{Example 1 \sphinxhyphen{} d=100 nm}

If we look at a film thickness of \(d=100\) nm and a film of \(n=1.33\), which corresponds to water we obbtain constructive interference for
\begin{equation*}
\begin{split}\lambda_{max}=\frac{4\cdot 100\, {\rm nm} \cdot 1.33}{2m-1}\end{split}
\end{equation*}
or
\begin{equation*}
\begin{split}\lambda_{max}=\frac{ 532 {\rm nm}}{2m-1}\end{split}
\end{equation*}
which yields for different values of \(m\)
\begin{itemize}
\item {} 
\(m=1\): 532 nm

\item {} 
\(m=2\): 177 nm

\item {} 
\(m=3\): 106 nm

\end{itemize}

and so on. We see therefore that the longest wavelength to create constructive interference is \(532\) nm, which is green light. The next longest wavelength is 177 nm, which is not visible anymore, so the reflection of a \(d=100\) nm film would look green. The left plot in the figure below shows the intensity distribution over wavelength where you recognize that the maximum is very broad.


\begin{savenotes}\sphinxattablestart
\centering
\begin{tabulary}{\linewidth}[t]{|T|}
\hline
\sphinxstyletheadfamily 
\sphinxincludegraphics[width=0.400\linewidth]{{100nm_film}.png} \sphinxincludegraphics[width=0.400\linewidth]{{10nm_film}.png}
\\
\hline
\sphinxstylestrong{Fig.:} Reflection from a 100 nm (left) and a 10 nm (right)thin water film.
\\
\hline
\end{tabulary}
\par
\sphinxattableend\end{savenotes}

An interesting effect is appearing, when the thickness of the water film get very thin. We may ask, when is no constructive interference observed. We therefore set the wavelength of the constructive interefence to \(\lambda_{max}=400\, {\rm nm}\) and calculate the film thickness for which this occurs.
\begin{equation*}
\begin{split}d=\frac{(2m-1)\lambda_{max}}{2n}\approx 75\, {\rm nm}\end{split}
\end{equation*}
So for film thickness of water thinner than 75 nm, there is no constructive interference of the reflected light from the two boundaries in the visible region anymore. There will be still a reflection but no specific color. If the film gets even thinner, the intensity of the reflected light is further diminished by desctructive interference and whe obtain no reflection as shown on the right side of the above figure for a \(d=10\) nm film. Such thin films, which do not show any reflection are
called \sphinxstylestrong{Newton black films}. You might have seen them, if you look closer at soap bubbles. They will show regions, which look like holes, but of course there are no holes in soap bubbles.

If the film gets thicker, e.g. \(d=1\) µm or even \(d=100\) µm, more than one constructive interference fits into the visible wavelength range. Due to that, the film may appear to have mixed colors or even look white. Below are the diagrams for those film thicknesses.


\begin{savenotes}\sphinxattablestart
\centering
\begin{tabulary}{\linewidth}[t]{|T|}
\hline
\sphinxstyletheadfamily 
\sphinxincludegraphics[width=0.400\linewidth]{{1µm_film}.png} \sphinxincludegraphics[width=0.400\linewidth]{{100µm_film}.png}
\\
\hline
\sphinxstylestrong{Fig.:} Reflection from a 1 µm (left) and a 100 µm (right)thin water film.
\\
\hline
\end{tabulary}
\par
\sphinxattableend\end{savenotes}


\begin{savenotes}\sphinxattablestart
\centering
\begin{tabulary}{\linewidth}[t]{|T|}
\hline
\sphinxstyletheadfamily 
\sphinxincludegraphics[width=0.450\linewidth]{{soap_film_lecture}.png} \sphinxincludegraphics[width=0.450\linewidth]{{soap_film_lecture1}.png}
\\
\hline
\sphinxstylestrong{Fig.:} Experimental demonstration of the reflection of white light by a thin soap film.
\\
\hline
\end{tabulary}
\par
\sphinxattableend\end{savenotes}


\section{Multiple Wave Interference}
\label{\detokenize{notebooks/L10/Interference:Multiple-Wave-Interference}}
So far we looked at the interference of two waves, which was a simplification as I mentioned already earlier. Commonly there will be a multitude of partial waves contribute to the oberved intereference. This is what we would like to have a look at now. We will do that in a quite general fashion, as the resulting formulas will appear several times again for different problems.

Nevertheless we will make a difference between
\begin{itemize}
\item {} 
multiwave interference of waves with the same amplitude

\item {} 
multiwave interference of waves with decreasing amplitude

\end{itemize}

Especially the latter is often occuring, if we have multiple reflections and each reflection is only a fraction of the incident amplitude.


\subsection{Multiple Wave Interference with Constant Amplitude}
\label{\detokenize{notebooks/L10/Interference:Multiple-Wave-Interference-with-Constant-Amplitude}}
In the case of constant amplitude (for example realized by a grating, which we talk about later), the total wave amplitude is given according to the picture below by
\begin{equation*}
\begin{split}U=U_1+U_2+U_1+U_3+\ldots+U_M\end{split}
\end{equation*}
where we sum the amplitude over \(M\) partial waves. Between the neighboring waves (e.g. \(U_1\) and \(U_2\)), we will assume a phase difference (because of a path length difference for example), which we denote as \(\phi\).

The amplitude of the p\sphinxhyphen{}th wave is then given by
\begin{equation*}
\begin{split}U_p=\sqrt{I_0}e^{i(p-1)\phi}\end{split}
\end{equation*}
with the index \(p\) being an interger \(p=1,2,\ldots,M\), \(h=e^{i\phi}\) and \(\sqrt{I_0}\) as the amplitude of each individual wave. The total amplitude \(U\) can be then expressed as
\begin{equation*}
\begin{split}U=\sqrt{I_0}\left (1+h+h^2+\ldots +h^{M-1}\right)\end{split}
\end{equation*}
which is a geometric sum. We can apply the sum formula for geometric sums to obtain
\begin{equation*}
\begin{split}U=\sqrt{I_0}\frac{1-h^M}{1-h}=\sqrt{I_0}\frac{1-e^{iM\phi}}{1-e^{i\phi}}\end{split}
\end{equation*}
We now have to calculate the intensity of the total amplitude
\begin{equation*}
\begin{split}I=|U|^2=I_{0}\left | \frac{e^{-iM\phi/2}-e^{iM\phi/2}}{e^{-i\phi/2}-e^{i\phi/2}}\right |^2\end{split}
\end{equation*}
which we can further simplify to give
\begin{equation*}
\begin{split}I=I_{0}\frac{\sin^2(M\phi/2)}{\sin^2(\phi/2)}\end{split}
\end{equation*}

\begin{savenotes}\sphinxattablestart
\centering
\begin{tabulary}{\linewidth}[t]{|T|}
\hline
\sphinxstyletheadfamily 
\sphinxincludegraphics[width=0.900\linewidth]{{multi_wave}.png}
\\
\hline
\sphinxstylestrong{Fig.:} Multiple wave interefence. Left: Phase construction of a multiwave intereference with M equal amplitude waves. Middle: Intensity distribution obtained as a function of the phase shift \(\phi\). Right: Examplary sketch of the interference generation.
\\
\hline
\end{tabulary}
\par
\sphinxattableend\end{savenotes}

The result is therefore an oscillating function. The numerator \(\sin^2(M\phi/2)\) shows and oscillation frequency, which is by a factor of \(M\) higher than the one in the denominator \(\sin^2\phi/2)\). Special situations occur, whenever the numerator and the denominator become zero. This will happen whenever
\begin{equation*}
\begin{split}\phi=m 2\pi\end{split}
\end{equation*}
where \(m\) is an integer and denotes the interference order, i.e. the number of wavelength that neighboring partial waves have as path length difference. In this case, the intensity distributiion will give us
\begin{equation*}
\begin{split}I=I_0 \frac{0}{0}\end{split}
\end{equation*}
and we have to determine the limit with the help of l’Hospitals rule. The outcome of this calculation is, that
\begin{equation*}
\begin{split}I(\phi=m2\pi)=M^2 I_0\end{split}
\end{equation*}
which can be also realized when using the small angle approximation for the sine functions.

Since he numerator is much faster oscillating then the denominator, we may also encounter the situation, where the numerator is zero, but the denomintor is not. These situations result in the additional minima between the primary maxima. We will have exactly \(M-1\) minima between the global maxima and \(M-2\) primary maxima. We will come back to these details when we talk about the diffraction grating in the next section.


\subsection{Multiple Wave Interference with Decreasing Amplitude}
\label{\detokenize{notebooks/L10/Interference:Multiple-Wave-Interference-with-Decreasing-Amplitude}}
We will turn our attention now to a slight modification of the previous multiwave interference. We will introduce a decreasing amplitude of the individual waves. The first wave shall have an amplitude \(U_1=\sqrt{I_0}\). The next wave, however, will not only be phase shifted but also have a smaller amplitude.
\begin{equation*}
\begin{split}U_2=h U_1\end{split}
\end{equation*}
where \(h=re^{i\phi}\) with \(|h|=r<1\). \(r\) can be regarded as a reflection coefficient, which deminishes the amplitude of the incident wave. According to that the intensity is reduced by
\begin{equation*}
\begin{split}I_2=|U_2|^2=|h U_1|^2=r^2 I_1\end{split}
\end{equation*}
The intensity of the incident wave is multiplied by a factor \(r^2\), while the amplitude is multiplied by \(r\). Note that the phase factor \(e^{i\phi}\) is removed when taking the square of this complex number.

\begin{sphinxadmonition}{note}{}\unskip
\sphinxstylestrong{Intensity at Boundaries}

The amplitude of the reflected wave is diminished by a factor \(r\le 1\), which is called the reflection coefficient. The intensity is diminished by a factor \(R=|r|^2\le1\), which is the \sphinxstylestrong{reflectance}.

In the absence of absorption, reflectance \(R\) and \sphinxstylestrong{transmittance} \(T\) add to one due to energy conservation.
\begin{equation*}
\begin{split}R+T=1\end{split}
\end{equation*}\end{sphinxadmonition}

Consequently, the third wave would be now \(U_3=hU_2=h^2U_1\). The total amplitude is thus
\begin{equation*}
\begin{split}U=U_1+U_2+U_3+\ldots+U_M = \sqrt{I_0}(1+h+h^2+\ldots)\end{split}
\end{equation*}

\begin{savenotes}\sphinxattablestart
\centering
\begin{tabulary}{\linewidth}[t]{|T|}
\hline
\sphinxstyletheadfamily 
\sphinxincludegraphics[width=0.700\linewidth]{{multi_wave_decreasing}.png}
\\
\hline
\sphinxstylestrong{Fig.:} Multiple wave interefence with decreasing amplitudes. Left: Phase construction of a multiwave intereference with M equal amplitude waves. Right: Intensity distribution obtained as a function of the phase shift \(\phi\).
\\
\hline
\end{tabulary}
\par
\sphinxattableend\end{savenotes}

This yields again
\begin{equation*}
\begin{split}U=\sqrt{I_0}\frac{(1-h^M)}{1-h}=\frac{\sqrt{I_0}}{1-r e^{i\phi}}\end{split}
\end{equation*}
Calculating the intensity of the waves is giving
\begin{equation*}
\begin{split}I=|U|^2=\frac{I_{0}}{|1-re^{i\phi}|^2}=\frac{I_0}{(1-r)^2+4r\sin^2(\phi/2)}\end{split}
\end{equation*}
which is also known as the \sphinxstylestrong{Airy function}. This function can be further simplified by the following abbrevations
\begin{equation*}
\begin{split}I_{\rm max}=\frac{I_0}{(1-r)^2}\end{split}
\end{equation*}
and
\begin{equation*}
\begin{split}\mathcal{F}=\frac{\pi \sqrt{r}}{1-r}\end{split}
\end{equation*}
where the latter is called the \sphinxstyleemphasis{Finesse}. With those abbrevations, we obtain
\begin{equation*}
\begin{split}I=\frac{I_{\rm max}}{1+4\left(\frac{\mathcal{F}}{\pi}\right)\sin^{2}(\phi/2)}\end{split}
\end{equation*}
for the interference of multiple waves with decreasing amplitude.

This intensity distribution has a different shape than the one we obtained for multiple waves with the same amplitude.


\begin{savenotes}\sphinxattablestart
\centering
\begin{tabulary}{\linewidth}[t]{|T|}
\hline
\sphinxstyletheadfamily 
\sphinxincludegraphics[width=0.600\linewidth]{{perot}.png}
\\
\hline
\sphinxstylestrong{Fig.:} Multiple wave interference with decreasing amplitude. The graph shows the intensity distribution over the phase angle \(\phi\) for different values of the Finesse \(\mathcal{F}\).
\\
\hline
\end{tabulary}
\par
\sphinxattableend\end{savenotes}

We clearly observe that with increasing Finesse the intensity maxima, which occur at multiples fo \(\pi\) get much narrower. In addition the regions between the maxima show better contrast and fopr higher Finesse we get complete destructive interference.


\section{Lecture Contents}
\label{\detokenize{lectures/L11/overview_11:lecture-contents}}\label{\detokenize{lectures/L11/overview_11::doc}}
In Lecture 11 we look at multi\sphinxhyphen{}wave interference again, anti\sphinxhyphen{}reflection coatings and explore the diffraction of light including Huygens principle.

\noindent\sphinxincludegraphics[width=600\sphinxpxdimen]{{slides2}.png}

Lecture 11 slides for download \sphinxcode{\sphinxupquote{pdf}}

The following section was created from \sphinxcode{\sphinxupquote{source/notebooks/L11/Interference.ipynb}}.


\section{Fabry Perot Interferometer}
\label{\detokenize{notebooks/L11/Interference:Fabry-Perot-Interferometer}}\label{\detokenize{notebooks/L11/Interference::doc}}


We will now have a look at some applications of the multiwave interference also with changing amplitude. One of the is the Fabry\sphinxhyphen{}Perot interferometer, which essentially consists of two mirrors, which are brought to a distance \(d\) as shown below.


\begin{savenotes}\sphinxattablestart
\centering
\begin{tabulary}{\linewidth}[t]{|T|}
\hline
\sphinxstyletheadfamily 
\sphinxincludegraphics[width=0.500\linewidth]{{fabry_perot}.png}
\\
\hline
\sphinxstylestrong{Fig.:} Fabry\sphinxhyphen{}Perot Interferometer.
\\
\hline
\end{tabulary}
\par
\sphinxattableend\end{savenotes}

If light enters the Fabry Perot interferometer with an amplitude \(A_0\). It is transmitted with a reduced amplitude. The initial amplitude has to be multiplied with the transmission factor \(t_1\). It travels further until it hits the sescond mirror, where it is reflected and transmitted as well. The reduced amplitude due to reflection is expressed by \(r_2\), while the transmission is introducing another factor \(t_2\). This means that the first transmitted wave has the
amplitude
\begin{equation*}
\begin{split}U_1=A_0t_1 t_2\end{split}
\end{equation*}
The second transmitted wave now has a decreased amplitude as compared to the first one and reads
\begin{equation*}
\begin{split}U_2=A_0t_1 t_2 r_1 r_2 e^{i\phi}=U_1 r_1 r_2 e^{i\phi}\end{split}
\end{equation*}
We could now continue like that, but we see already at this point the similarity to the multiwave interference with decreasing amplitude we have talked about in the last lecture. We would observe interference of all transmitted waves

We just have to insert the corresponding expression, i.e. \(\sqrt{I_0}=A_0 t_1 t_2\) and \(h=r_1r_2e^{i\phi}\) or \(r=r_1 r_2\). We may even insert the phase shift
\begin{equation*}
\begin{split}\phi=\frac{2\pi}{\lambda} \Delta s = \frac{2\pi}{\lambda} 2d\cos(\theta)\end{split}
\end{equation*}
Our final formula would thus look like
\begin{equation*}
\begin{split}I=|U|^2=\frac{I_{0}}{|1-re^{i\phi}|^2}=\frac{I_0}{(1-r)^2+4r\sin^2\left (\frac{2\pi}{\lambda} d\cos(\theta)\right)}\end{split}
\end{equation*}
and gives now the intensity as calculated during the last lecture.


\begin{savenotes}\sphinxattablestart
\centering
\begin{tabulary}{\linewidth}[t]{|T|}
\hline
\sphinxstyletheadfamily 
\sphinxincludegraphics[width=0.600\linewidth]{{perot1}.png}
\\
\hline
\sphinxstylestrong{Fig.:} Fabry Perot Interferometer.
\\
\hline
\end{tabulary}
\par
\sphinxattableend\end{savenotes}

Yet, we know now that \(r=r_1 r_2\) is the factor by which the amplitude decreases, which also enters the Finesse
\begin{equation*}
\begin{split}\mathcal{F}=\frac{\pi \sqrt{r}}{1-r}\end{split}
\end{equation*}
The better the reflectivity of the mirrors (which means \(r\rightarrow 1\)), the higher is the Finess and the sharper are the interference peaks observed in the Fabry\sphinxhyphen{}Perot interferometer.

Let us discuss the form of the interference peeks and the distance a bit more in detail. We have plotted so far the intensity as a function of the phase angle \(\phi\). This phase angle is given by
\begin{equation*}
\begin{split}\phi=\frac{4\pi d}{\lambda}\cos(\theta)\end{split}
\end{equation*}
We would like to have a look at the transmission in the case of normal incidence \(\theta=0\), so that \(\phi=4\pi d/\lambda\). To obtain constructive interference in transmission, the phase angle needs to be an integer multiple of \(2\pi\), i.e. \(\phi=m 2\pi\). This means that for a given thickness \(d\) and a specific integer value \(m\), the transmission is maximum for
\begin{equation*}
\begin{split}\lambda_m=\frac{2d}{m}\end{split}
\end{equation*}
We may then calculate the distance of the constructive interference maxima for neighboring values of the interference order \(m\). A quick calculation gives
\begin{equation*}
\begin{split}\delta \lambda= \lambda_{m}-\lambda_{m+1}=\frac{\lambda_m}{m+1}\end{split}
\end{equation*}
We may also do the calculation in the frequency space. A wavelength \(\lambda_m\) corresponds via the relation \(\nu=c/\lambda\) to \(\nu_m=\frac{c\, m}{2d}\). The difference between neighboring frequencies is then
\begin{equation*}
\begin{split}\delta \nu=\nu_{m+1}-\nu_m=\frac{c}{2d}\end{split}
\end{equation*}
This quantity \(\delta \nu\) or \(\delta \lambda\) is called \sphinxstylestrong{free spectral range} as it indicates the “space” we have until the next frequency delivers a constructive interference maximum for the next integer \(m\). In this range, we would like to resolve two different frequencies. So let us assume we would like to resolve two frequencies \(\nu_a\) and \(\nu_b\) with the same order of the interference \(m\), then we have to find a criterium, that tells us, that we can
still seperate the two frequencies. We use as a measure the width of the individual peaks. We measure the width as the full width at half maximum (FWHM), this means, we look for the phase angle \(\phi_{1/2}\) at which we have \(I/I_{\rm max}=1/2\), i.e.
\begin{equation*}
\begin{split}\frac{I}{I_{\rm max}}=\frac{1}{2}=\frac{1}{1+\left( \frac{\mathcal{F}}{\pi}\right )^2\phi^{2}_{1/2} }\end{split}
\end{equation*}
using the small angle approximation \(\sin(\phi)\approx\phi\).This results in an angle \(\phi_{1/2}= \frac{\pi}{\mathcal{F}}\) for the half width. The full width is thus twice this values, i.e. \(2\frac{\pi}{\mathcal{F}}\). This full width corresponds also to a width in frequency space \(\Delta \nu\), which with our conversion formula \(\phi=4\pi\nu d/c\) yields
\begin{equation*}
\begin{split}\Delta \nu=\frac{c}{2d\mathcal{F}}=\frac{\delta \nu}{\mathcal{F}}\end{split}
\end{equation*}
This actually directly gives a definition for the Finesse.
\begin{equation*}
\begin{split}\mathcal{F}=\frac{\delta \nu}{\Delta \nu}=\frac{\lambda}{\delta \lambda}\end{split}
\end{equation*}
This means that the Finesse gives us a measure for the spectral resolution of a Fabry\sphinxhyphen{}Perot\sphinxhyphen{}Interferometer. It tells us by how much the free spectral range \$:nbsphinx\sphinxhyphen{}math:\sphinxtitleref{delta \textasciigrave{}:nbsphinx\sphinxhyphen{}math:}nu \sphinxtitleref{\$ is bigger than the width :math:}Delta nu\textasciigrave{} of the interference maximum.

More generally this is measured by the resolving power \(\mathcal{R}\), which is given for the Fabry\sphinxhyphen{}Perot\sphinxhyphen{}Interferometer
\begin{equation*}
\begin{split}R=m\mathcal{F}\end{split}
\end{equation*}
where \(m\) is again the interference order \(m\), we defined above.


\begin{savenotes}\sphinxattablestart
\centering
\begin{tabulary}{\linewidth}[t]{|T|}
\hline
\sphinxstyletheadfamily 
\sphinxincludegraphics[width=0.600\linewidth]{{perot_spectral}.png}
\\
\hline
\sphinxstylestrong{Fig.:} Fabry Perot Interferometer.
\\
\hline
\end{tabulary}
\par
\sphinxattableend\end{savenotes}



\sphinxincludegraphics[width=0.450\linewidth]{{fabry_perot_lecture1}.png} \sphinxincludegraphics[width=0.450\linewidth]{{fabry_perot_lecture2}.png}



\sphinxstylestrong{Fig.:} Fabry Perot Interferometer and interferebnce pattern observed in the lecture.






\section{Newton Rings}
\label{\detokenize{notebooks/L11/Interference:Newton-Rings}}


\sphinxincludegraphics[width=0.800\linewidth]{{newton_rings_lecture}.png}



\sphinxstylestrong{Fig.:} Observation of Newton Rings using white light in the lecture.






\section{Bragg Reflection}
\label{\detokenize{notebooks/L11/Interference:Bragg-Reflection}}
The Bragg reflection is a typical example in which we assume that the amplitude of the waves interfering is constant. Remember the image below, which we already discussed in the previous lecture



\sphinxincludegraphics[width=0.600\linewidth]{{Bragg_reflection}.png}



\sphinxstylestrong{Fig.:} Bragg reflection.





We obtained earlier, that constructive interference will occur if the path difference \(\Delta s\) of neighboring waves (e.g. \(U_1\) and \(U_2\)) in just an integer multiple of the wavelength \(\Delta s=m\lambda\). For the picture above, we may easily calculate the pathlength difference (the distance between the blue lines \(\Delta s_1+\Delta s_2\)), which is given by the angle of incidence \(\theta\) (note that it is measured to the plane and not to the normal here!)
and the distance between the horizontal planes \(d\). According to that we find that
\begin{equation*}
\begin{split}\Delta s=\Delta s_1+\Delta s_2 =2 d\sin(\theta)\end{split}
\end{equation*}
\begin{sphinxadmonition}{note}{}\unskip
\sphinxstylestrong{Bragg reflection}

A Bragg reflection occurs on a periodic arrangement of reflecting objects with a spacing \(d\) under an incident angle \(\theta\) to the plane if
\begin{equation*}
\begin{split}\Delta s=2 d\sin(\theta)=m\lambda\end{split}
\end{equation*}
which is known as the \sphinxstylestrong{Bragg equation}.
\end{sphinxadmonition}

This law of Bragg reflection is the foundation for the structural analysis using X\sphinxhyphen{}rays, for example. If we assume that the periodic arrangement of objects is a lattice of atoms, the the reflection is microscopically related to the scattering of the X\sphinxhyphen{}rays on the atoms in the lattice. The atoms thereby act as sources of spherical waves, which in the far\sphinxhyphen{}field (infinite distance) can be treated as plane waves interfering. If we observe therefore a constructive interference from such a lattice
structure under an angle \(\theta\), we may deduce the lattice constant \(d\) when we know the wavelength \(\lambda\).



\sphinxincludegraphics[width=0.600\linewidth]{{Lattice_reflection}.png}



\sphinxstylestrong{Fig.:} Bragg reflection on a lattice.





In a real atomic lattice, atoms are arranged in more complex periodic structures than shown in 2 dimensions above. In X\sphinxhyphen{}ray scattering pattern is therefore a chracteristic dot pattern on the surface of a sphere. You will extend this picture later in the course on \sphinxstyleemphasis{Solid State Physics}. More generally you may see X\sphinxhyphen{}ray scattering also as diffraction on a 3\sphinxhyphen{}dimensional grating.



\sphinxincludegraphics[width=0.600\linewidth]{{xray}.png}



\sphinxstylestrong{Fig.:} One of the first recorded X\sphinxhyphen{}ray scattering patterns from a platinum lattice.





The image below shows the experiment from the lecture, where the periodic structure is given by a stack of acrylic sheets, with constant separation. A microwave is incident to the stack. Each sheet is reflecting a tiny fraction of the wave so that the transmitted wave by a sheet is almost the original one.



\sphinxincludegraphics[width=0.750\linewidth]{{multi_wave_lecture}.png}



\sphinxstylestrong{Fig.:} Microwave reflection from a stack of acrylic glass sheets.





As a result we obtained for certain angles constructive interference in reflection, which is a Bragg reflection as well. Note that we used here Microwaves as radiation, as these waves have a wavelength of several centimeters. We may therefore arrange the objects at a distance of centimeters (have a look at the Bragg equation again).


\section{Bragg Mirrors, Anti\sphinxhyphen{}Reflection Coating and Photonic Crystals}
\label{\detokenize{notebooks/L11/Interference:Bragg-Mirrors,-Anti-Reflection-Coating-and-Photonic-Crystals}}

\subsection{Bragg Mirror}
\label{\detokenize{notebooks/L11/Interference:Bragg-Mirror}}
We may also prepare structure, which are periodic and act on optical waves, i.e. on waves in the visible regions. There we create stacks of materials with different refractive index. At each interface waves are reflected. As compared to the X\sphinxhyphen{}ray situation, light is also refracted before it propagates further to the next interface. During propagation not only the path length but the optical path length is of importance as the wavelength is changed when the refractive index is changed.



\sphinxincludegraphics[width=0.400\linewidth]{{multilayer_optics}.png}



\sphinxstylestrong{Fig.:} Periodic optical multilayer structure of layers with different refractive index.





Using such structures high reflectivities can be achieved in certain wavelength ranges even though the materials used are transparent. The propagation through such a multilayer stack can be formalized into matrix operations, which can simplify the calculation. Note that also phase jumps at the boundaries upon reflection have to be taken into accout. The overall calculations go beyond our scope here.



{\color{red}\bfseries{}|ae2a7fd464364c72bae130a0f9ce017a||65254dbff45e4d31938bc23ffaecb354|}



\sphinxstylestrong{Fig.:} (Left) Reflection spectrum of a Bragg mirror as demonstrated in the lecture. (Right) Reflection spectrum of two Bragg mirrors stacked on top of each other with an intermediate layer causing a “defect” in the reflection (little dip), which can be used as a cavity, e.g. for semiconductor lasers.






\subsection{Anti\sphinxhyphen{}Reflection Coating}
\label{\detokenize{notebooks/L11/Interference:Anti-Reflection-Coating}}
While we do not consider the periodic multilayer structures, we may consider another important applications. We wish to calculate the thickness and the optimal refractive index of a single layer that we would like to deposit on glass to reduce the reflection of the glass air interface. Such an antireflection coating is quite common, though, most antireflection coatings involve multiple layers.



\sphinxincludegraphics[width=0.400\linewidth]{{anti_reflection}.png}



\sphinxstylestrong{Fig.:} Setup for our calculation of he anti\sphinxhyphen{}reflection coating.





The sketch above shows the setup we would like to consider to calculate the thickness \(d_1\) of the layer as well as its refractive index \(n_2\) such that we get zero reflection from the underlying glass \(n_3\). At each of the interfaces, we have a reflectance \(R_n=|r_n|^2\), where \(r_n\) are the reflection coefficients. We would like to use this setup and calculate the anti\sphinxhyphen{}reflection layer properties for normal incidence.

\sphinxstylestrong{Thickness of the coating}

To calculate the thickness of the coating, we just have to consider the destructive interference condition. For this we first need to assume \(n_1<n_2<n_3\), since that assures that there is the same phase jump of \(\pi\) at the first and the second interface, so we do not need to consider it. The destructive interference is then realized by
\begin{equation*}
\begin{split}2k d_1=(2m+1)\pi\end{split}
\end{equation*}
Here \(k\) is the wavenumber in the thin film and \(m=0,1,\ldots\) is an integer. Just inserting \(k=2\pi/\lambda\) yields
\begin{equation*}
\begin{split}\frac{4\pi}{\lambda} d_1=(2m+1)\pi\end{split}
\end{equation*}
or finally the thickness of the layer
\begin{equation*}
\begin{split}d_1=(2m+1)\frac{\lambda}{4}\end{split}
\end{equation*}
We are free to chose the order of the destructive interference \(m\) so we just choose \(m=0\). In this case, we obtain a film thickness of
\begin{equation*}
\begin{split}d_1=\frac{\lambda}{4}\end{split}
\end{equation*}
Note that the wavelength \(\lambda\) corresponds to the wavelength of light in the material with the refractive index \(n_2\) and not the original wavelength in vacuum.

\sphinxstylestrong{Refractive index of the coating}

To get real destructive interference, we need the two reflected waves we are considering to cancel completely also in intensity. This means that
\begin{equation*}
\begin{split}R_1 I_0=(1-R_1)R_2 I_0\end{split}
\end{equation*}
Typically \(R_1\) and \(R_2\) are small (only a few percent are reflected). Thus we may assume that \(R_1 R_2\approx 0\) and therefroe
\begin{equation*}
\begin{split}R_1 I_0\approx R_2 I_0\end{split}
\end{equation*}
or \(R_1\approxR_2\). We next need to relate the refractive indices to the reflection coefficient. We will derive the exact formulas later in the electromagnetic wave chapter, but for now we may use the reflectance for normal incidence when the light is coming from the medium with \(n_1\) to the medium \(n_2\). This is given by
\begin{equation*}
\begin{split}R_1=\left |\frac{n_2-n_1}{n_1+n_2}\right|^2\end{split}
\end{equation*}
Our anti\sphinxhyphen{}reflection condition for the intensities is thus
\begin{equation*}
\begin{split}\frac{n_2-n_1}{n_1+n_2}=\frac{n_3-n_2}{n_2+n_3}\end{split}
\end{equation*}
from which you find by transformation
\begin{equation*}
\begin{split}n2=\sqrt{n_1 n_3}\end{split}
\end{equation*}
The refractive index for a anti\sphinxhyphen{}reflection coating is thus given as the square root of the product of the neighboring materials refractive indices. Thus for example, if we use \(n_1=1\) (air) and \(n_3=1.5\) (glass), then we obtain \(n_2=1.22\) for the appropriate coating refractive index. Unfortunately refractive indices rely on materials and are not completely arbitrary. Therefore, we have to choose one of the robust materials, that is closest to this value, which is in this case
\(Mg F_2\) with \(n_2=1.38\), which brings down reflection far enough. Better performance, also across larger ranges of wavelength, is obtained with multilayer structures.



\sphinxincludegraphics[width=0.800\linewidth]{{anti_reflex_coating}.png}



\sphinxstylestrong{Fig.:} Glass surface with three different segements of surface treatment. One segment is uncoated glass, the other two segements are coeated with an anti\sphinxhyphen{}reflection coeating. In these regions the reflections is clearly reduced. .






\subsection{Photonic Crystals}
\label{\detokenize{notebooks/L11/Interference:Photonic-Crystals}}
Photonic crystals take the idea of periodic structures now to three dimensions. If we create a periodic structure much like and atomic lattice just out of dielectric units, which interact with light, we obtain a complex interference structure depending on the direction of wave propagation. This creates directions under which propagation is leading to complete transmission and others where there is complete reflections. This relation between wavenumber and frequency for which a propagation is
leadint to transmission is called the dispersion relation and this is plotted in a band diagram. Such a band structure diagram is completely equivalent to what you may know about the propagation of electrons in an atomic lattice.



\sphinxincludegraphics[width=0.800\linewidth]{{butterfly_wings}.png}



\sphinxstylestrong{Fig.:} Butterfly wing reflection corresponding to a 3D Bragg reflection from a complex structured dielectric lattice.





The figure above shows such periodic dielectric structures, which are found in nature. The are used to create an impression of color, where a pigment is actually not really suitable, e.g. because coloring by light absorbing molecules may spoil the heat balance in butterfly wings.


\section{Lecture Contents}
\label{\detokenize{lectures/L12/overview_12:lecture-contents}}\label{\detokenize{lectures/L12/overview_12::doc}}
In Lecture 12 we explore the diffraction of light including Huygens principle. We study the diffraction of a slit, a circular aperture as well as of a grating.

\noindent\sphinxincludegraphics[width=600\sphinxpxdimen]{{slides3}.png}

Lecture 11 slides for download \sphinxcode{\sphinxupquote{pdf}}

The following section was created from \sphinxcode{\sphinxupquote{source/notebooks/L12/Diffraction.ipynb}}.


\section{Diffraction}
\label{\detokenize{notebooks/L12/Diffraction:Diffraction}}\label{\detokenize{notebooks/L12/Diffraction::doc}}


{
\sphinxsetup{VerbatimColor={named}{nbsphinx-code-bg}}
\sphinxsetup{VerbatimBorderColor={named}{nbsphinx-code-border}}
\begin{sphinxVerbatim}[commandchars=\\\{\}]
\llap{\color{nbsphinxin}[2]:\,\hspace{\fboxrule}\hspace{\fboxsep}}\PYG{k+kn}{import} \PYG{n+nn}{numpy} \PYG{k}{as} \PYG{n+nn}{np}
\PYG{k+kn}{import} \PYG{n+nn}{matplotlib}\PYG{n+nn}{.}\PYG{n+nn}{pyplot} \PYG{k}{as} \PYG{n+nn}{plt}
\PYG{k+kn}{from} \PYG{n+nn}{time} \PYG{k+kn}{import} \PYG{n}{sleep}\PYG{p}{,}\PYG{n}{time}
\PYG{k+kn}{from} \PYG{n+nn}{ipycanvas} \PYG{k+kn}{import} \PYG{n}{MultiCanvas}\PYG{p}{,} \PYG{n}{hold\PYGZus{}canvas}\PYG{p}{,}\PYG{n}{Canvas}
\PYG{k+kn}{import} \PYG{n+nn}{matplotlib} \PYG{k}{as} \PYG{n+nn}{mpl}
\PYG{k+kn}{import} \PYG{n+nn}{matplotlib}\PYG{n+nn}{.}\PYG{n+nn}{cm} \PYG{k}{as} \PYG{n+nn}{cm}


\PYG{o}{\PYGZpc{}}\PYG{k}{matplotlib} inline
\PYG{o}{\PYGZpc{}}\PYG{k}{config} InlineBackend.figure\PYGZus{}format = \PYGZsq{}retina\PYGZsq{}

\PYG{c+c1}{\PYGZsh{} default values for plotting}
\PYG{n}{plt}\PYG{o}{.}\PYG{n}{rcParams}\PYG{o}{.}\PYG{n}{update}\PYG{p}{(}\PYG{p}{\PYGZob{}}\PYG{l+s+s1}{\PYGZsq{}}\PYG{l+s+s1}{font.size}\PYG{l+s+s1}{\PYGZsq{}}\PYG{p}{:} \PYG{l+m+mi}{12}\PYG{p}{,}
                     \PYG{l+s+s1}{\PYGZsq{}}\PYG{l+s+s1}{axes.titlesize}\PYG{l+s+s1}{\PYGZsq{}}\PYG{p}{:} \PYG{l+m+mi}{18}\PYG{p}{,}
                     \PYG{l+s+s1}{\PYGZsq{}}\PYG{l+s+s1}{axes.labelsize}\PYG{l+s+s1}{\PYGZsq{}}\PYG{p}{:} \PYG{l+m+mi}{16}\PYG{p}{,}
                     \PYG{l+s+s1}{\PYGZsq{}}\PYG{l+s+s1}{axes.labelpad}\PYG{l+s+s1}{\PYGZsq{}}\PYG{p}{:} \PYG{l+m+mi}{14}\PYG{p}{,}
                     \PYG{l+s+s1}{\PYGZsq{}}\PYG{l+s+s1}{lines.linewidth}\PYG{l+s+s1}{\PYGZsq{}}\PYG{p}{:} \PYG{l+m+mi}{1}\PYG{p}{,}
                     \PYG{l+s+s1}{\PYGZsq{}}\PYG{l+s+s1}{lines.markersize}\PYG{l+s+s1}{\PYGZsq{}}\PYG{p}{:} \PYG{l+m+mi}{10}\PYG{p}{,}
                     \PYG{l+s+s1}{\PYGZsq{}}\PYG{l+s+s1}{xtick.labelsize}\PYG{l+s+s1}{\PYGZsq{}} \PYG{p}{:} \PYG{l+m+mi}{16}\PYG{p}{,}
                     \PYG{l+s+s1}{\PYGZsq{}}\PYG{l+s+s1}{ytick.labelsize}\PYG{l+s+s1}{\PYGZsq{}} \PYG{p}{:} \PYG{l+m+mi}{16}\PYG{p}{,}
                     \PYG{l+s+s1}{\PYGZsq{}}\PYG{l+s+s1}{xtick.top}\PYG{l+s+s1}{\PYGZsq{}} \PYG{p}{:} \PYG{k+kc}{True}\PYG{p}{,}
                     \PYG{l+s+s1}{\PYGZsq{}}\PYG{l+s+s1}{xtick.direction}\PYG{l+s+s1}{\PYGZsq{}} \PYG{p}{:} \PYG{l+s+s1}{\PYGZsq{}}\PYG{l+s+s1}{in}\PYG{l+s+s1}{\PYGZsq{}}\PYG{p}{,}
                     \PYG{l+s+s1}{\PYGZsq{}}\PYG{l+s+s1}{ytick.right}\PYG{l+s+s1}{\PYGZsq{}} \PYG{p}{:} \PYG{k+kc}{True}\PYG{p}{,}
                     \PYG{l+s+s1}{\PYGZsq{}}\PYG{l+s+s1}{ytick.direction}\PYG{l+s+s1}{\PYGZsq{}} \PYG{p}{:} \PYG{l+s+s1}{\PYGZsq{}}\PYG{l+s+s1}{in}\PYG{l+s+s1}{\PYGZsq{}}\PYG{p}{,}\PYG{p}{\PYGZcb{}}\PYG{p}{)}
\end{sphinxVerbatim}
}


\subsection{Huygens Principle}
\label{\detokenize{notebooks/L12/Diffraction:Huygens-Principle}}
The Huygens principle states, that each point of a wavefront is the source of a spherical wave in forward direction. This means nothing else, that any wave can be expanded into a superposition of spherical waves, which is the fundamental of Mie scattering for example. Yet, the overall statement of this principle seems a bit unphysical. Classically, accelerated charges are the source of electromagnetic waves. If there is no accelerated charge, there is no wave. Yet, the Huygens principle is in
accord with quantum field theory.



\sphinxincludegraphics[width=0.400\linewidth]{{huygens}.png}



\sphinxstylestrong{Fig.:} Huygens principle for a plane wave incident with a wave vector \(\vec{k}\).





The sketch above illustrates the Huygens principle, but we may also check that numerically. The graphs below were calculated to illustrate Huygens principle using spherical waves placed in a line close to each other.



\sphinxincludegraphics[width=1.000\linewidth]{{huygens_num}.png}



\sphinxstylestrong{Fig.:} Huygens principle used to create a plane wave from a set of spherical waves. The graph on the left shows the amplitude of a single spherical wave of a wavelength \(\lambda=532\) nm. If we arange 500 spherical wave sources along the x\sphinxhyphen{}direction very densely and they all are in phase since the represent the phase of the incident plane wave at \(z=0\), we can recreate the plane wavefronts for \(z>0\) (middle) and the constant intensity (right).





We have already dealt with all of the mathematics describing this situation. We have earlier looked at the interference of multiple waves with the same amplitude. In this case we have \(N\) spherical waves arranged along the x\sphinxhyphen{}axis at \(z=0\) with neighboring sources at a distance \(d\). If we consider the interference at infinite distance, the phase difference between the neighboring waves is given by \(\Delta \phi=2\pi/\lambda d\sin(\theta)\), where \(\theta\) is the angle
with the z\sphinxhyphen{}direction. Thus all \(N\) waves give according to our previous calculations an intensity pattern
\begin{equation*}
\begin{split}I=I_0 \frac{\sin^2\left (N\frac{\pi}{\lambda}d\sin(\theta)\right)}{\sin^2\left (\frac{\pi}{\lambda}d\sin(\theta)\right)}\end{split}
\end{equation*}
This is again our starting point for the next calculations concernign a single slit, which confines the range of Huygens waves emitted to a region along the x\sphinxhyphen{}direction. This interference of Huygens waves is termed diffraction, yet the physical principle behind is interference. Our second calculation will have a look at many slits arranged along the x\sphinxhyphen{}direction, which is then called a diffraction grating. The latter is an indispensible tool for spectroscopy.


\subsection{Single Slit diffraction}
\label{\detokenize{notebooks/L12/Diffraction:Single-Slit-diffraction}}
We would like to apply this formula to study the diffraction of an incident plane wave (wavevector \(k\)) on a single slit of width \(b\). Along this slit opening we place a number of Huygens sources. The sketch below has 3 Huygens sources. We may generalize that to \(N\) sources.



\sphinxincludegraphics[width=400\sphinxpxdimen]{{single_slit}.png}



\sphinxstylestrong{Fig.:}





We will devide the slit into segments of width \(\Delta b\) such that we have \(N=b/\Delta b\) Huygens sources with an amplitude \(A_0=\sqrt{I_0}\) along the slit. We may then apply the formula from teh previous section for the multi\sphinxhyphen{}wave interference using \(d=\Delta b\). We obtain
\begin{equation*}
\begin{split}I=I_0 \frac{\sin^2\left (\pi\frac{b}{\lambda}\sin(\theta)\right)}{\sin^2\left (\pi\frac{\Delta b}{\lambda}\sin(\theta)\right)}\end{split}
\end{equation*}
where we can replace the \(\Delta b\) in the denominator by \(\Delta b=N/b\) to yield
\begin{equation*}
\begin{split}I=I_0 \frac{\sin^2\left (\pi\frac{b}{\lambda}\sin(\theta)\right)}{\sin^2\left (\pi\frac{b}{N\lambda}\sin(\theta)\right)}\end{split}
\end{equation*}
Substituting \(x=\pi \frac{b}{\lambda}\sin(\theta)\)

then gives
\begin{equation*}
\begin{split}I=I_0\frac{\sin^2(x)}{\sin^2(x/N)}\end{split}
\end{equation*}
Since we do not have a finite number \(N\) sources in the slit, we have to consider the limit of the above formula for \(N\rightarrow\infty\), which we have to calculate for the denominator. There with increasing \(N\) the argument of the sine function becomes very small and we can use our small angle expansion of the sine function to yield
\begin{equation*}
\begin{split}\sin^2\left (\frac{x}{N}\right) \approx \frac{x^2}{N^2}\end{split}
\end{equation*}
and therefore
\begin{equation*}
\begin{split}I(\theta)=I_s\frac{\sin^2\left (\pi \frac{b}{\lambda}\sin(\theta)\right)}{\left( \pi \frac{b}{\lambda}\sin(\theta)\right)^2}\end{split}
\end{equation*}
where \(I_s=N^2I_0\). The function \(sin(x)/x\) is often abbrevated as \(\sinc(x)\), which is the sinus cardinalis. Let’s have a look at some of the properties of the intensity distribution with the help of some python code.

\begin{sphinxadmonition}{note}{}\unskip
\sphinxstylestrong{Single Slit Diffraction}

The intensity distribution generated by the diffraction of monochromatic light on a single slit and observed in the far field is given by
\begin{equation*}
\begin{split}I(\theta)=I_s\frac{\sin^2\left (\pi \frac{b}{\lambda}\sin(\theta)\right)}{\left( \pi \frac{b}{\lambda}\sin(\theta)\right)^2}\end{split}
\end{equation*}
where \(\lambda\) is the wavelength of the light and \(b\) the width of the slit. The angle of observation is given by \(\theta\).
\end{sphinxadmonition}

\sphinxstylestrong{Function definition}

{
\sphinxsetup{VerbatimColor={named}{nbsphinx-code-bg}}
\sphinxsetup{VerbatimBorderColor={named}{nbsphinx-code-border}}
\begin{sphinxVerbatim}[commandchars=\\\{\}]
\llap{\color{nbsphinxin}[93]:\,\hspace{\fboxrule}\hspace{\fboxsep}}\PYG{k}{def} \PYG{n+nf}{single\PYGZus{}slit}\PYG{p}{(}\PYG{n}{d}\PYG{p}{,}\PYG{n}{theta}\PYG{p}{,}\PYG{n}{wl}\PYG{p}{)}\PYG{p}{:}
    \PYG{n}{d}\PYG{o}{=}\PYG{n}{np}\PYG{o}{.}\PYG{n}{pi}\PYG{o}{*}\PYG{n}{d}\PYG{o}{/}\PYG{n}{wl}\PYG{o}{*}\PYG{n}{np}\PYG{o}{.}\PYG{n}{sin}\PYG{p}{(}\PYG{n}{theta}\PYG{p}{)}
    \PYG{k}{return}\PYG{p}{(}\PYG{p}{(}\PYG{n}{np}\PYG{o}{.}\PYG{n}{sin}\PYG{p}{(}\PYG{n}{d}\PYG{p}{)}\PYG{o}{/}\PYG{n}{d}\PYG{p}{)}\PYG{o}{*}\PYG{o}{*}\PYG{l+m+mi}{2}\PYG{p}{)}
\end{sphinxVerbatim}
}

\sphinxstylestrong{Parameter definition}

{
\sphinxsetup{VerbatimColor={named}{nbsphinx-code-bg}}
\sphinxsetup{VerbatimBorderColor={named}{nbsphinx-code-border}}
\begin{sphinxVerbatim}[commandchars=\\\{\}]
\llap{\color{nbsphinxin}[96]:\,\hspace{\fboxrule}\hspace{\fboxsep}}\PYG{n}{wl}\PYG{o}{=}\PYG{l+m+mf}{532e\PYGZhy{}9}
\PYG{n}{b}\PYG{o}{=}\PYG{l+m+mf}{5e\PYGZhy{}6}
\PYG{n}{theta}\PYG{o}{=}\PYG{n}{np}\PYG{o}{.}\PYG{n}{linspace}\PYG{p}{(}\PYG{o}{\PYGZhy{}}\PYG{n}{np}\PYG{o}{.}\PYG{n}{pi}\PYG{o}{/}\PYG{l+m+mi}{6}\PYG{p}{,}\PYG{n}{np}\PYG{o}{.}\PYG{n}{pi}\PYG{o}{/}\PYG{l+m+mi}{6}\PYG{p}{,}\PYG{l+m+mi}{1000}\PYG{p}{)}
\end{sphinxVerbatim}
}

\sphinxstylestrong{Plotting}

{
\sphinxsetup{VerbatimColor={named}{nbsphinx-code-bg}}
\sphinxsetup{VerbatimBorderColor={named}{nbsphinx-code-border}}
\begin{sphinxVerbatim}[commandchars=\\\{\}]
\llap{\color{nbsphinxin}[95]:\,\hspace{\fboxrule}\hspace{\fboxsep}}\PYG{n}{plt}\PYG{o}{.}\PYG{n}{figure}\PYG{p}{(}\PYG{n}{figsize}\PYG{o}{=}\PYG{p}{(}\PYG{l+m+mi}{15}\PYG{p}{,}\PYG{l+m+mi}{5}\PYG{p}{)}\PYG{p}{)}
\PYG{n}{plt}\PYG{o}{.}\PYG{n}{subplot}\PYG{p}{(}\PYG{l+m+mi}{1}\PYG{p}{,}\PYG{l+m+mi}{2}\PYG{p}{,}\PYG{l+m+mi}{1}\PYG{p}{)}
\PYG{n}{plt}\PYG{o}{.}\PYG{n}{plot}\PYG{p}{(}\PYG{n}{np}\PYG{o}{.}\PYG{n}{sin}\PYG{p}{(}\PYG{n}{theta}\PYG{p}{)}\PYG{p}{,}\PYG{n}{single\PYGZus{}slit}\PYG{p}{(}\PYG{n}{b}\PYG{p}{,}\PYG{n}{theta}\PYG{p}{,}\PYG{l+m+mf}{532e\PYGZhy{}9}\PYG{p}{)}\PYG{p}{,}\PYG{l+s+s1}{\PYGZsq{}}\PYG{l+s+s1}{green}\PYG{l+s+s1}{\PYGZsq{}}\PYG{p}{,}\PYG{n}{lw}\PYG{o}{=}\PYG{l+m+mi}{2}\PYG{p}{,}\PYG{n}{label}\PYG{o}{=}\PYG{l+s+s2}{\PYGZdq{}}\PYG{l+s+s2}{532 nm}\PYG{l+s+s2}{\PYGZdq{}}\PYG{p}{)}
\PYG{n}{plt}\PYG{o}{.}\PYG{n}{plot}\PYG{p}{(}\PYG{n}{np}\PYG{o}{.}\PYG{n}{sin}\PYG{p}{(}\PYG{n}{theta}\PYG{p}{)}\PYG{p}{,}\PYG{n}{single\PYGZus{}slit}\PYG{p}{(}\PYG{n}{b}\PYG{p}{,}\PYG{n}{theta}\PYG{p}{,}\PYG{l+m+mf}{700e\PYGZhy{}9}\PYG{p}{)}\PYG{p}{,}\PYG{l+s+s1}{\PYGZsq{}}\PYG{l+s+s1}{red}\PYG{l+s+s1}{\PYGZsq{}}\PYG{p}{,}\PYG{n}{lw}\PYG{o}{=}\PYG{l+m+mi}{2}\PYG{p}{,}\PYG{n}{label}\PYG{o}{=}\PYG{l+s+s2}{\PYGZdq{}}\PYG{l+s+s2}{700 nm}\PYG{l+s+s2}{\PYGZdq{}}\PYG{p}{)}
\PYG{n}{plt}\PYG{o}{.}\PYG{n}{annotate}\PYG{p}{(}\PYG{l+s+s1}{\PYGZsq{}}\PYG{l+s+s1}{first minimum}\PYG{l+s+s1}{\PYGZsq{}}\PYG{p}{,}\PYG{n}{xy}\PYG{o}{=}\PYG{p}{(}\PYG{l+m+mf}{0.2}\PYG{p}{,} \PYG{l+m+mi}{0}\PYG{p}{)}\PYG{p}{,}\PYG{n}{xytext}\PYG{o}{=}\PYG{p}{(}\PYG{l+m+mf}{0.2}\PYG{p}{,} \PYG{l+m+mf}{0.25}\PYG{p}{)}\PYG{p}{,}\PYG{n}{arrowprops}\PYG{o}{=}\PYG{n+nb}{dict}\PYG{p}{(}\PYG{n}{facecolor}\PYG{o}{=}\PYG{l+s+s1}{\PYGZsq{}}\PYG{l+s+s1}{black}\PYG{l+s+s1}{\PYGZsq{}}\PYG{p}{,} \PYG{n}{width}\PYG{o}{=}\PYG{l+m+mf}{0.5}\PYG{p}{,}\PYG{n}{headwidth}\PYG{o}{=}\PYG{l+m+mi}{6}\PYG{p}{)}\PYG{p}{)}
\PYG{n}{plt}\PYG{o}{.}\PYG{n}{xlabel}\PYG{p}{(}\PYG{l+s+sa}{r}\PYG{l+s+s1}{\PYGZsq{}}\PYG{l+s+s1}{\PYGZdl{}}\PYG{l+s+s1}{\PYGZbs{}}\PYG{l+s+s1}{sin(}\PYG{l+s+s1}{\PYGZbs{}}\PYG{l+s+s1}{theta)\PYGZdl{}}\PYG{l+s+s1}{\PYGZsq{}}\PYG{p}{)}
\PYG{n}{plt}\PYG{o}{.}\PYG{n}{ylabel}\PYG{p}{(}\PYG{l+s+s1}{\PYGZsq{}}\PYG{l+s+s1}{\PYGZdl{}I/I\PYGZus{}s\PYGZdl{}}\PYG{l+s+s1}{\PYGZsq{}}\PYG{p}{)}
\PYG{n}{plt}\PYG{o}{.}\PYG{n}{legend}\PYG{p}{(}\PYG{p}{)}
\PYG{n}{plt}\PYG{o}{.}\PYG{n}{subplot}\PYG{p}{(}\PYG{l+m+mi}{1}\PYG{p}{,}\PYG{l+m+mi}{2}\PYG{p}{,}\PYG{l+m+mi}{2}\PYG{p}{)}
\PYG{n}{plt}\PYG{o}{.}\PYG{n}{plot}\PYG{p}{(}\PYG{n}{np}\PYG{o}{.}\PYG{n}{sin}\PYG{p}{(}\PYG{n}{theta}\PYG{p}{)}\PYG{p}{,}\PYG{n}{single\PYGZus{}slit}\PYG{p}{(}\PYG{n}{b}\PYG{o}{/}\PYG{l+m+mi}{2}\PYG{p}{,}\PYG{n}{theta}\PYG{p}{,}\PYG{l+m+mf}{532e\PYGZhy{}9}\PYG{p}{)}\PYG{p}{,}\PYG{l+s+s1}{\PYGZsq{}}\PYG{l+s+s1}{green}\PYG{l+s+s1}{\PYGZsq{}}\PYG{p}{,}\PYG{n}{lw}\PYG{o}{=}\PYG{l+m+mi}{2}\PYG{p}{,}\PYG{n}{label}\PYG{o}{=}\PYG{l+s+s2}{\PYGZdq{}}\PYG{l+s+s2}{532 nm}\PYG{l+s+s2}{\PYGZdq{}}\PYG{p}{)}
\PYG{n}{plt}\PYG{o}{.}\PYG{n}{plot}\PYG{p}{(}\PYG{n}{np}\PYG{o}{.}\PYG{n}{sin}\PYG{p}{(}\PYG{n}{theta}\PYG{p}{)}\PYG{p}{,}\PYG{n}{single\PYGZus{}slit}\PYG{p}{(}\PYG{n}{b}\PYG{o}{/}\PYG{l+m+mi}{2}\PYG{p}{,}\PYG{n}{theta}\PYG{p}{,}\PYG{l+m+mf}{700e\PYGZhy{}9}\PYG{p}{)}\PYG{p}{,}\PYG{l+s+s1}{\PYGZsq{}}\PYG{l+s+s1}{red}\PYG{l+s+s1}{\PYGZsq{}}\PYG{p}{,}\PYG{n}{lw}\PYG{o}{=}\PYG{l+m+mi}{2}\PYG{p}{,}\PYG{n}{label}\PYG{o}{=}\PYG{l+s+s2}{\PYGZdq{}}\PYG{l+s+s2}{700 nm}\PYG{l+s+s2}{\PYGZdq{}}\PYG{p}{)}
\PYG{n}{plt}\PYG{o}{.}\PYG{n}{xlabel}\PYG{p}{(}\PYG{l+s+sa}{r}\PYG{l+s+s1}{\PYGZsq{}}\PYG{l+s+s1}{\PYGZdl{}}\PYG{l+s+s1}{\PYGZbs{}}\PYG{l+s+s1}{sin(}\PYG{l+s+s1}{\PYGZbs{}}\PYG{l+s+s1}{theta)\PYGZdl{}}\PYG{l+s+s1}{\PYGZsq{}}\PYG{p}{)}
\PYG{n}{plt}\PYG{o}{.}\PYG{n}{ylabel}\PYG{p}{(}\PYG{l+s+s1}{\PYGZsq{}}\PYG{l+s+s1}{\PYGZdl{}I/I\PYGZus{}s\PYGZdl{}}\PYG{l+s+s1}{\PYGZsq{}}\PYG{p}{)}
\PYG{n}{plt}\PYG{o}{.}\PYG{n}{legend}\PYG{p}{(}\PYG{p}{)}
\PYG{n}{plt}\PYG{o}{.}\PYG{n}{show}\PYG{p}{(}\PYG{p}{)}
\end{sphinxVerbatim}
}

\hrule height -\fboxrule\relax
\vspace{\nbsphinxcodecellspacing}

\makeatletter\setbox\nbsphinxpromptbox\box\voidb@x\makeatother

\begin{nbsphinxfancyoutput}

\noindent\sphinxincludegraphics[width=910\sphinxpxdimen,height=335\sphinxpxdimen]{{notebooks_L12_Diffraction_17_0}.png}

\end{nbsphinxfancyoutput}

The two plots show the characteristic diffraction pattern of a single slit, which is an oscillating function, decaying in amplitude. The oscillation is defined by the numerator and the decaying amplitude by the denominator. The left graph displays the diffraction pattern as a function of \(\sin(\theta)\) for two different wavelength for a slit width of \(b=5\, {\rm \mu m}\). The right graph displays the diffraction of the same wavelength for only half the slit width
\(b=2.5\, {\rm \mu m}\). Looking at these graphs, we can make the following observations
\begin{itemize}
\item {} 
longer wavelength show a wider diffraction pattern

\item {} 
small slit width creates a wider diffraction pattern

\end{itemize}

We can also quantify these observation by calculating the minima of the oscillating denominator. The denomintor becomes zero whenever the argument of the sine square function is \(\pi\). From this it easily follows that
\begin{equation*}
\begin{split}\sin(\theta)=\frac{\lambda}{b}\end{split}
\end{equation*}
to have a minimum in the intensity of slit diffraction. This type of dependency, i.e. wavelength divided by the dimension of the diffracting object, is valid in general even though it migh show up with diffreent numerical prefactors for other diffracting geometries.



\sphinxincludegraphics[width=600\sphinxpxdimen]{{slit_diff_min}.png}



\sphinxstylestrong{Fig.:} Diffraction patterns as a function of the sine of the diffraction angle. The minima of the diffraction pattern in this plot are at integer multiples of \(\lambda/b\).







\sphinxincludegraphics[width=0.480\linewidth]{{single_slit1}.png} \sphinxincludegraphics[width=0.480\linewidth]{{single_slit2}.png}



\sphinxstylestrong{Fig.:} Diffraction patterns on a single slit as observed in the lecture. The left image shows the diffraction patter for red light, while the right image combines two different wavelength (red, blue), where one clearly recognizes the wider diffraction peaks for the longer red wavelength.






\subsection{Circular Aperture}
\label{\detokenize{notebooks/L12/Diffraction:Circular-Aperture}}
A similar but a bit more involved calculation can be done for a circular aperture. The result of this calculation is
\begin{equation*}
\begin{split}I(\theta)=I_0\left( \frac{2J_1(x)}{x} \right )^2\end{split}
\end{equation*}
with
\begin{equation*}
\begin{split}x=\frac{2\pi R}{\lambda}\sin(\theta)\end{split}
\end{equation*}
where \(J_1\) is the \sphinxhref{https://en.wikipedia.org/wiki/Bessel\_function}{Bessel function} of first kind, \(R\) is the radius of the aperture. The Bessel function is similar to the sine function and oscillating function but has its zeros at \(x_1=1.22 \pi\), \(x_2=2.23 \pi\) and so on.



\sphinxincludegraphics[width=0.900\linewidth]{{circ_ap}.png}



\sphinxstylestrong{Fig.:} Diffraction pattern of a circular aperture of radius \(5\) µm. Note that the intensity scale is saturated. The diffraction rings would otherwise not be visible. The minima of the diffraction pattern in this plot are at integer multiples of \(0.61\lambda/R\).





According to the zeros of the numerator, we can again obtain the sine of the angle under which the first minimum in the intensity occurs in radial direction from the center of the diffraction pattern. We just have to set
\begin{equation*}
\begin{split}x_1=1.22\pi=\frac{2\pi R}{\lambda}\sin(\theta_1)\end{split}
\end{equation*}
which yields
\begin{equation*}
\begin{split}\sin(\theta_1)=0.61\frac{\lambda}{R}\end{split}
\end{equation*}
We recognize the same dependence as before: wavelength divided by the dimension of the diffracting element. The width of the diffraction peak in the center is well defined by this first minimum. This minimum therefore defines the radius of the so called \sphinxstylestrong{Airy disc}. In the language of microscopy, it defines the size of a resolution element which is termed \sphinxstylestrong{resel}.

\sphinxstylestrong{Diffraction on the Eye\sphinxhyphen{}Iris}

The above description gives us some nice insight into diffraction effects which occur on the aperture of the eye \sphinxhyphen{} the iris. So lets do some quick estimate.

The aperture of the iris has a radius of \(R=2.5\) mm giving us for a wavelength of \(\lambda=532\) nm the sine of the diffraction angle to be
\begin{equation*}
\begin{split}\sin(\theta_1)=0.61\frac{532\, {\rm nm}}{2.5\, {\rm mm}}=0.00013\end{split}
\end{equation*}
If we multiply this value with the diameter of the eye (roughly 2 cm), we obtain the size of the \sphinxstylestrong{Airy disc} on the “screen”, which is the retina.
\begin{equation*}
\begin{split}2\,{\rm cm}\sin(\theta_1)=2.59\, {\mu m}\end{split}
\end{equation*}
So the width of the diffraction pattern of light on the retina is roughly 2.59 µm, which corresponds well to the distance between light sensitive cell in the eye around the fovea. In this region we find densities of aroiund \(150.000\) cells per square millimeter, which gives a distance of 2.58 µm. It is not usefull to spend a larger amount of cells on a region of an Airy disc, as it does not increase the resolution but requires the cells to be more sensitive, as they have to share the
intensity.


\subsubsection{Resolution of an Optical Microscope}
\label{\detokenize{notebooks/L12/Diffraction:Resolution-of-an-Optical-Microscope}}
We may spin this idea a bit further and use the diffraction pattern of a circular aperature to define the resolution of a microscope.


\section{Lecture Contents}
\label{\detokenize{lectures/L13/overview_13:lecture-contents}}\label{\detokenize{lectures/L13/overview_13::doc}}
In Lecture 12 we explore the diffraction of light including Huygens principle. We study the diffraction of a slit, a circular aperture as well as of a grating.

\noindent\sphinxincludegraphics[width=600\sphinxpxdimen]{{slides4}.png}

Lecture 11 slides for download \sphinxcode{\sphinxupquote{pdf}}


\section{Lecture Contents}
\label{\detokenize{lectures/L14/overview_14:lecture-contents}}\label{\detokenize{lectures/L14/overview_14::doc}}
In Lecture 12 we explore the diffraction of light including Huygens principle. We study the diffraction of a slit, a circular aperture as well as of a grating.

\noindent\sphinxincludegraphics[width=600\sphinxpxdimen]{{lectures/L14/img/slides}.png}

Lecture 11 slides for download \sphinxcode{\sphinxupquote{pdf}}


\chapter{Indices and tables}
\label{\detokenize{index:indices-and-tables}}\begin{itemize}
\item {} 
\DUrole{xref,std,std-ref}{genindex}

\item {} 
\DUrole{xref,std,std-ref}{modindex}

\item {} 
\DUrole{xref,std,std-ref}{search}

\end{itemize}



\renewcommand{\indexname}{Index}
\printindex
\end{document}